% Appendix F

\chapter{No inversion symmetry of mean field Hamiltonian, 1 wire} % Main appendix title

\label{Appendix.noinversionsymmetry} % For referencing this appendix elsewhere, use \ref{AppendixA}
\chead{}
\lhead{Appendix F. \emph{No inversion symmetry of mean field Hamiltonian}} % This is for the header on each page - perhaps a shortened title

In this appendix we show, that the single wire system has no inversion symmetry. We define the parity transformation, $P$, in the following manner in second quantization: 
\begin{equation}
P^2 = \mathbb{I}, \hspace{0.5cm} P\begin{bmatrix} \psi_F(x) \\ \psi^\dagger_F(x) \end{bmatrix} P = U_P\begin{bmatrix} \psi_F(-x) \\ \psi^\dagger_F(-x) \end{bmatrix}.
\label{eq.definversion}
\end{equation}
The first expression states, that the parity transformation is assumed to be unitary and hermitian: $P^\dagger = P$. Notice, that we allow for some nontrivial mixing of particles and holes. The above definition sets a restriction on the $2 \times 2$ matrix $U_P$. Firstly, since $P^2 = \mathbb{I}$ we also get that $U_P^2 = \mathbb{I}$, so that $U_P$ is unitary and hermitian. Secondly, since the entries of the Nambu spinor are not independent, we get:
\begin{align}
P\psi_F(x)P &= U_P^{11} \psi_F(-x) + U_P^{12} \psi^\dagger(-x), \nonumber \\
P\psi_F(x)P &= U_P^{21*} \psi^\dagger_F(-x) + U_P^{22*} \psi(-x). \nonumber
\end{align}
The second line is obtained by taking the hermitian conjugate of the second line in the definition. Since the coefficients to the same operator must be identical we get: $U_P^{11} = U_P^{22*}, U_P^{12} = U_P^{21*}$. The second requirement is already fulfilled when $U_P$ is hermitian, but the first one is a further restriction to the matrix. Hence, we have:
\begin{equation}
U_P = \begin{bmatrix} U_P^{11} & U_P^{12} \\ U_P^{12*} & U_P^{11*}\end{bmatrix}. 
\label{eq.UP}
\end{equation}
In momentum space the above definition means, that:
\begin{equation}
P\begin{bmatrix} c_k \\ c^\dagger_k \end{bmatrix} P = U_P\begin{bmatrix} c_{-k} \\ c^\dagger_{-k} \end{bmatrix}.
\end{equation}
Now to investigate whether the Hamiltonian has an inversion symmetry, we should first write the Hamiltonian in a form depending on the above spinor. Gauge transforming $\Delta_k$ to be real as in chapter \ref{Chapter7} and transforming the Hamiltonian yields:
\begin{align}
H_{FF} &= \frac{1}{2}\sum_k \begin{bmatrix} c_k^\dagger & c_{-k}\end{bmatrix}\mathcal{H}_{FF,k} \begin{bmatrix} c_k \\ c^\dagger_{-k}\end{bmatrix} \nonumber \\
	   &= \frac{1}{4}\sum_k \tilde{C}^\dagger_k \tilde{\mathcal{H}}_{FF,k} \tilde{C}_k, \hspace{0.5cm} \tilde{\mathcal{H}}_{FF,k} = \epsilon_k \rho_0 \otimes \tau_3 - \Delta_k \rho_2 \otimes \tau_2, \hspace{0.5cm} \tilde{C}_k = \begin{bmatrix} c_k \\ c^\dagger_{k} \\ c_{-k} \\ c^\dagger_{-k}\end{bmatrix}, 
\end{align}
where we let $\rho_i$ be the Pauli matrices in $k\to -k$ space and $\tau_i$ the matrices in particle-hole space. Now: $P\tilde{C}_kP = \tilde{U}_P \tilde{C}_k$ with $\tilde{U}_P = \rho_1\otimes U_P$. This means, that to have an inversion symmetry we need:
\begin{equation}
PH_{FF}P = H_{FF} \Rightarrow \tilde{U}_P\tilde{\mathcal{H}}_{FF,k}\tilde{U}_P = \tilde{\mathcal{H}}_{FF,k}. \nonumber 
\end{equation}
Since $\rho_1$ of course commutes with $\rho_0$ we need $U_P$ to commute with $\tau_3$. Hence, $U_P = \tau_0$ or $U_P = \tau_3$. Further, since $\rho_1$ anticommutes with $\rho_2$ we need $U_P$ to anticommute with $\tau_2$. In total, this means that $U_P = \tau_3$ is the \textit{only} possibility. This transformation does fulfill the requirement above, however it does not have the right structure. Specifically from equation \ref{eq.UP} we get, that $1 = U_P^{11} = U_P^{22*} = -1$. A clear contradiction. Hence, we get that the system has no inversion symmetry. 

