% Appendix D

\chapter{Quasiparticle distribution} % Main appendix title

\label{Appendix.distribution.quasiparticles} 
\chead{Appendix D. \emph{Quasiparticle distribution, 1 wire}}
\lhead{Part V. Appendices} 
In this appendix we calculate the distribution function for fermionic particles, that are not number-conserved. This distribution function is used for the quasiparticles of the single and double wire system. 

Let $H = \sum_i E_i a^\dagger_i a_i$ be the Hamiltonian with $a^\dagger_i$ a fermionic creation operator:
\begin{equation}
\{a_i, a^\dagger_j\} = \delta_{i,j}, 
\label{eq.appendix.fermioniccommutationrelation}
\end{equation}
and $i$ the quantum number. We assume, that the number of these fermions are \textit{not} conserved. This means, that the partition function takes the form of the \textit{canonical} partition function: $Z = \tr\left[\text{e}^{-\beta H}\right]$. In terms of thermodynamics this means, that every single fermion is in thermal (and not diffusive) equilibrium with all the others, hence working as a heat reservoir. Since the Hamiltonian is diagonal, we can calculate the partition function for each $i$ by replacing $H$ with the $i$'th (diagonal) term. We then get:
\begin{equation}
Z_i = \tr\left[\text{e}^{-\beta H_i}\right] = \tr\left[\text{e}^{-\beta E_ia^\dagger_i a_i }\right] = 1 + \text{e}^{-\beta E_i}. \nonumber
\end{equation}     
For the calculation of the trace, we use the single particle complete basis $\{\ket{0}, a^\dagger_i\ket{0}\}$, with $\ket{0}$ the ground state: $a_i\ket{0} = 0$. This is all a rather involved way of saying, that the fermion can either be absent, $\ket{0}$, and have zero energy or be present, $a^\dagger_i\ket{0}$, and have energy $E_i$. The total partition function is then simply $Z = \prod_i Z_i$. The mean number of fermions with quantum number $i$ is given by $\braket{a^\dagger_ia_i} = \tr\left[\text{e}^{-\beta H}a^\dagger_ia_i\right]/Z$. For the calculation of the trace we need a complete basis. The basis consists of states with any number of fermions present with all possible quantum numbers. These states can all be written as: $\prod_{j\in J} a^\dagger_j \ket{0}$, where $J$ is an arbitrary set of quantum numbers. Since $a^\dagger_ia_i$ counts the number of fermions present with quantum number $i$, we only need sets $J$ with $i \in J$. For these states we get:
\begin{equation}
\text{e}^{-\beta H} a^\dagger_i a_i\prod_{j\in J} a^\dagger_j \ket{0} = \text{e}^{-\beta H}\prod_{j\in J} a^\dagger_i \ket{0} = \text{e}^{-\beta \sum_{j \in J} E_j} \prod_{j \in J} a^\dagger_i \ket{0}. \nonumber
\end{equation}
The desired trace is then the sum of $\text{e}^{-\beta \sum_{j \in J} E_j}$ for all combinations of $J$ containing $i$:
\begin{equation}
\tr\left[\text{e}^{-\beta H}a^\dagger_ia_i\right] = \sum_J \text{e}^{-\beta \sum_{j \in J} E_j} = \text{e}^{-\beta E_i}\prod_{j \neq i} \left[1 + \text{e}^{-\beta E_j}\right] = \text{e}^{-\beta E_i} \frac{Z}{Z_i}. \nonumber 
\end{equation}
The second equality is most easily verified by writing out the product. By doing this one explicitly sees, that all combinations of energies are present. Finally: 
\begin{equation}
\braket{a^\dagger_ia_i} = \frac{\tr\left[\text{e}^{-\beta H}a^\dagger_ia_i\right]}{Z} = \frac{\text{e}^{-\beta E_i }}{Z_i} = \frac{\text{e}^{-\beta E_i}}{ 1 + \text{e}^{-\beta E_i} } = \frac{1}{ \text{e}^{\beta E_i} + 1 } = f(E_i),
\end{equation}
with $f(E)$ the Fermi-Dirac distribution. This is the average number of $a$ particles in the thermalized state of the system. We emphasize that this calculation is a rather formal approach. In a simpler manner we have, that the average number of fermions with quantum number $i$ is given by: $\sum_n n P_i(n) = \frac{1}{Z_i}\sum_{n = 0}^{1} \text{e}^{ -n\beta E_i } = f( E_i )$, with the probability of occupation with $n$ fermions in the $i$'th state given by $\text{e}^{-n\beta E_i}/Z_i$. In any regard the above explicitly shows, how we equivalently can formulate this using the second quantized operators. 

The present appendix hereby in detail shows, that fermionic free particles with energy $E_i$, that are not number-conserved, are distributed according to $f(E_i)$. For the single wire system we make the identification: $a_i \to \gamma_k$. For the double wire system: $a_i \to \gamma_{j,k}$, where $j$ specifies the wire and $k$ is the momentum. 
