% Appendix C

\chapter{Quasiparticle distribution, 1 wire} % Main appendix title

\label{Appendix.distribution.quasiparticles} % For referencing this appendix elsewhere, use \ref{AppendixA}
\chead{}
\lhead{Appendix C. \emph{Quasiparticle distribution, 1 wire}} % This is for the header on each page - perhaps a shortened title
In this appendix we in detail describe how to take the thermal average for the single wire within the mean field approximation. We use this to show, that the quasiparticles are Fermi-Dirac distributed. Throughout we neglect the ground state grand energy $E_0 = \frac{1}{2}\sum_k \left[\varepsilon_k - \Delta_k \braket {c^\dagger_k c^\dagger_{-k}}\right]$.

Throughout the thesis we keep the mean number of fermions, $\braket{N_F}$, constant. However, we do not fix the number of quasiparticles ($\gamma$). This means, that the partition function takes the form of the \textit{canonical} partition function: $Z = \tr\left[\text{e}^{-\beta H_{FF}}\right]$. In terms of thermodynamics this means, that every single quasiparticle is in thermal (and not diffusive) equilibrium with all the others, hence working as a heat reservoir. Since the Hamiltonian is diagonal in the quasiparticles $\gamma_k$, we can calculate the partition function for each $k$ by replacing $H_{FF}$ with the $k$'th (diagonal) term. We then get:
\begin{equation}
Z_k = \tr\left[\text{e}^{-\beta H_{FF,k}}\right] = \tr\left[\text{e}^{-\beta E_{F,k}\gamma^\dagger_k\gamma_k }\right] = 1 + \text{e}^{-\beta E_{F,k}}. \nonumber
\end{equation}     
For the calculation of the trace, we use the single particle complete basis $\{\ket{\text{S}}_0, \gamma^\dagger_k\ket{\text{S}}_0\}$. This is all a rather involved way of saying, that the quasiparticle can either be absent, $\ket{\text{S}}_0$, and have zero energy or present, $\gamma^\dagger_k\ket{\text{S}}_0$, and have energy $E_{F,k}$. The total partition function is then simply $Z = \prod_k Z_k$. The mean number of quasiparticles with momentum $k$ is given by $\braket{\gamma^\dagger_k\gamma_k} = \tr\left[\text{e}^{-\beta H_{FF}}\gamma^\dagger_k\gamma_k\right]/Z$. For the calculation of the trace we need a complete basis. The basis consists of states with any number of quasiparticles present with all possible momenta. These states can all be written as: $\prod_{q\in K} \gamma^\dagger_q \ket{\text{S}}_0$, where $K$ is an arbitrary set of momenta. Since $\gamma^\dagger_k\gamma_k$ counts the number of quasiparticles present with momentum $k$, we only need sets $K$ with $k \in K$. For these states we get:
\begin{equation}
\text{e}^{-\beta H_{FF}} \gamma^\dagger_k \gamma_k\prod_{q\in K} \gamma^\dagger_q \ket{\text{S}}_0 = \text{e}^{-\beta H_{FF}}\prod_{q\in K} \gamma^\dagger_q \ket{\text{S}}_0  = \text{e}^{-\beta \sum_{q \in K} E_{F,q}} \prod_{q\in K} \gamma^\dagger_q \ket{\text{S}}_0. \nonumber
\end{equation}
The desired trace is then the sum of $\text{e}^{-\beta \sum_{q \in K} E_{F,q}}$ for all combinations of $K$ containing $k$:
\begin{equation}
\tr\left[\text{e}^{-\beta H_{FF}}\gamma^\dagger_k\gamma_k\right] = \sum_K \text{e}^{-\beta \sum_{q \in K} E_{F,q}} = \text{e}^{-\beta E_{F,k}}\prod_{q \neq k} \left[1 + \text{e}^{-\beta E_{F,q}}\right] = \text{e}^{-\beta E_{F,k}} \frac{Z}{Z_k}. \nonumber 
\end{equation}
The second equality is most easily verified by writing out the product. By doing this one explicitly sees, that all combinations of energies are present. Finally: 
\begin{equation}
\braket{\gamma^\dagger_k\gamma_k} = \frac{\tr\left[\text{e}^{-\beta H_{FF}}\gamma^\dagger_k\gamma_k\right]}{Z} = \frac{\text{e}^{-\beta E_{F,k}}}{Z_k} = \frac{\text{e}^{-\beta E_{F,k}}}{1 + \text{e}^{-\beta E_{F,k}}} = f(E_{F,k}),
\end{equation}
with $f(E)$ the Fermi-Dirac distribution. This is the average number of $\gamma_k$ particles in the thermalized state of the system. We emphasize that this calculation is a rather formal approach. In a simpler manner we have, that the average number of quasiparticles with momentum $k$ is given by: $\sum_n n P_k(n) = \frac{1}{Z_k}\sum_{n = 0}^{1} \text{e}^{ -n\beta E_{F,k} } = f( E_{F,k} )$, with the probability of occupation with $n$ quasiparticles in the $k$'th state given by $\text{e}^{-n\beta E_{F,k}}/Z_k$. In any regard the above explicitly shows, how we equivalently can formulate this using the second quantized operators. 

By going from the $c$ to $\gamma$ operators we can calculate all desired averages.
