% Appendix E

\chapter{Fluctuations in the number of fermions} % Main appendix title

\label{Appendix.fermionnumberfluctuation} 
\fancyhead[LO, RE]{Part V. \emph{Appendices}}
\chead{E. \emph{Fermion number fluctuations}}

In this appendix we discuss the variance of the fermion particle number in the $T = 0$ limit. 

As noted in the derivation of the mean field Hamiltonian, the resulting Hamiltonian does not conserve the number of fermions: $[N_{j,F}, H_{FF}] \neq 0 $, where $N_{j,F} = \sum_k c^\dagger_{j,k} c_{j,k}$ is the number operator for fermions in wire $j$. As a consequence there will be a variance $\braket{(N_{j,F}-\braket{N_{j,F}})^2} \neq 0$. Calculating the variance gives a neat demonstration of how one can utilize Wick's theorem. To keep the notation simple we do the calculation for the fermions in wire $1$ and suppress the $1$ subscript in $N_{1,F}$. We get:
\begin{equation}
\braket{(N_F - {\braket{N_F}})^2} = \braket{N_F^2} - \braket{N_F}^2 = \sum_{k,q} \braket{c^\dagger_{1,k}c_{1,k}c^\dagger_{1,q}c_{1,q}} - \braket{c^\dagger_{1,k}c_{1,k}}\braket{c^\dagger_{1,q}c_{1,q}}. \nonumber
\end{equation} 
The main challenge is thus to compute $\braket{c^\dagger_{1,k}c_{1,k}c^\dagger_{1,q}c_{1,q}}$. Since the mean field Hamiltonian is quadratic, we can use Wick's theorem to reduce this four-body mean to a sum of two-body means:
\begin{equation}
\braket{c^\dagger_{1,k}c_{1,k}c^\dagger_{1,q}c_{1,q}} = \braket{c^\dagger_{1,k}c_{1,k}}\braket{c^\dagger_{1,q}c_{1,q}} - \braket{c^\dagger_{1,k}c^\dagger_{1,q}}\braket{c_{1,k}c_{1,q}} + \braket{c^\dagger_{1,k}c_{1,q}}\braket{c_{1,k}c^\dagger_{1,q}}. \nonumber
\end{equation}
The sign $(-1)^{n}$ of the terms are thus given by the number times, $n$, we have to exchange two operators next to each other to get to the term in question \cite[pp. 198-202]{BruusFlensberg}. The first term we recognise as the part coming from $\braket{N_F}^2$. Hence, we only need to calculate the two latter terms. By changing to the quasiparticle $\gamma$-operators we can evaluate these directly. The calculation is rather lengthy and not very illuminating. It is therefore skipped. In the zero temperature limit we get:
\begin{equation}
\braket{(N_F - {\braket{N_F}})^2} = \frac{1}{4}\sum_k\left[ \frac{|\Delta^{11}_k + \Delta^{12}_k|^2}{(E^{+}_{F,k})^2} + \frac{1}{2E^{-}_{F,k}E^{+}_{F,k}}\left(E^{-}_{F,k}E^{+}_{F,k} - \left(\varepsilon_k^2 + (\Delta^{11}_k)^2 + |\Delta^{12}_k|^2\right) \right) \right].
\end{equation}
Firstly, when both inter- and intrawire pairings are zero both terms in the sum vanish, which means the variance is zero. This illustrates the fact, that it is the presence of the pairings, that makes the variance nonzero. Secondly, we notice that the last term is only present when $E^{-}_{F,k} \neq E^{+}_{F,k}$, which means when $\Delta^{12}_k$ is \textit{not} purely imaginary. For $\Delta^{11}_k = 0$ we recover the standard variance result of the original BCS theory for $s$-wave pairing (and $2N_F$ fermions) \cite[pp. 50-52]{Tinkham}. Finally, converting the sum to an integral in the thermodynamic limit gives us the relative variance:
\begin{equation}
\frac{\braket{(N_F - \braket{N_F})^2}}{\braket{N_F}^2} = \frac{1}{8\braket{N_F}}\int d\tilde{k} \; \left[ \frac{|\Delta^{11}_{\tilde{k}} + \Delta^{12}_{\tilde{k}}|^2}{(E^{+}_{F,\tilde{k}})^2} + \frac{1}{2E^{-}_{F,\tilde{k}}E^{+}_{F,\tilde{k}}}\left(E^{-}_{F,\tilde{k}}E^{+}_{F,\tilde{k}} - \left(\varepsilon_{\tilde{k}}^2 + (\Delta^{11}_{\tilde{k}})^2 + |\Delta^{12}_{\tilde{k}}|^2\right) \right) \right]. \nonumber
\end{equation}
Here we write $k = k_F \tilde{k}$. We use, that the spacing in momentum space is $\Delta \tilde{k} = \frac{2}{\braket{N_F}}$. This illustrates an important aspect, namely that the relative variance $\braket{(N_F - \braket{N_F})^2}/\braket{N_F}^2$ decreases as $1/\braket{N_F}$. Hence, the number of fermions becomes increasingly ill-defined, but it does so in a slow way, namely so that the relative variance approaches zero in the thermodynamic limit. This is all completely analogous with the standard BCS theory for $s$-wave pairing \cite[pp. 50-52]{Tinkham}. 


