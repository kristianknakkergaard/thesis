\documentclass[10pt]{article}
\usepackage[T1]{fontenc}
\usepackage[english, danish]{babel}
\usepackage[utf8]{inputenc}
\usepackage{fullpage}
\usepackage{url}
\usepackage{varioref}
\usepackage{braket}
\usepackage{graphicx}
\usepackage{multicol}
\usepackage{amsmath,amssymb}
\usepackage{mathtools}	

\title{Spørgsmål til Georg}
\author{Kristian Knakkergaard Nielsen}
\begin{document}
\maketitle
\section{Topologi}
\begin{itemize}
\item Flere tidsomvendingssymmetrier for towiresystemet, $T^2 = \pm 1$ begge mulige..?
\item Hvorfor bliver den minimale energi ikke 0 i overgangen mellem $\Delta^{11}_k$ og $\Delta^{12}_k$ for towiresystemet? Det forudsiger topologien umiddelbart.. 
\end{itemize}

\section{Parringspotential}
\begin{itemize}
\item Hvordan kan man se, at $\Delta^{12}_k$ er lige i $k$? Numerisk popper det ud af sig selv..
\item Hvorfor stiger $\Delta^{12}_k$ lineært for små $d$? SVAR: det gør den ikke. Plot til lavere $d$ for at se dette! (deltas2.1.5)
\item Hvordan forklarer man satureringen af $d_c$ med stigende $\xi$? 
\end{itemize}


\end{document}