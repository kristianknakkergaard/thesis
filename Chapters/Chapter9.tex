\chapter{Discussion} 

\label{Chapter9} 
\lhead{Part V. \emph{Discussions \& Conclusions}}
\chead{Chapter 9. \emph{Discussion}} 

\section{Cross over in the double wire system and physical discrepancies} \label{sec.Discussion.2wires.crossover}
To have a full understanding of the cross over in the double wire system, we had to use both a bulk and boundary perspective. The bulk effect for the separated wires is, that there is only a topologically nontrivial intrawire $p$-wave pairing present. The resulting boundary effect is, that Majorana edge states can form with no energy cost. Oppositely when the wires are very close, only a topologically trivial interwire $s$-wave pairing is present. In turn there are no edge states. 

To understand the transition in-between this bulk-boundary correspondence is even more important. As the wires are brought closer together, the interwire interaction respects a time reversal symmetry with $T^2 = - \mathbb{I}$. As the analytical analysis of chapter \ref{Chapter10} shows, the edge states are time reversal (Kramers) partners. Therefore, the edge states are unaffected by the interwire interaction. This holds until a mean field is chosen, namely the interwire $s$-wave pairing. This interwire pairing can do two things through a specific choice of its phase: it can respect the $T^2 = - \mathbb{I}$ symmetry or not. If it respects the symmetry, the edge states are still uncoupled. A calculation of the bulk topological invariant shows, that this holds until the energy gap closes. Then the bulk is topologically trivial and no edge states are present. If it \textit{breaks} this symmetry, the edge states are no longer uncoupled. Actually they gain a nonzero energy for any nonzero interwire pairing. In turn the bulk energy gap does not have to close for the edge state to vanish. 

The nontrivial result of the numerical analysis is, that the interwire pairing chooses to break the $T^2 = -\mathbb{I}$ symmetry. This is simply because it is the energetically favourable choice. Further, a coexistence of the $p$-wave and $s$-wave pairing is observed in the transition. 

This mean field result does however have an important physical discrepancy. Consider the gap equations for the double wire system, equation \eqref{eq.2wiresgapequations}. As we bring the wires closer together the intrawire pairing is not affected by the interwire induced interaction until an interwire pairing forms. This is absurd. The interwire interaction will of course alter the physical properties internally in each wire, also when the wires are far apart. This also leads to the odd prediction, that the critical temperature for intrawire pairing is unaffected by the interwire interaction, as long as the interwire pairing is absent. This is \textit{not} physically reasonable. 

The reason for this discrepancy is the same as the reason why, the mean field approximation breaks down for a single wire above the critical temperature. When no mean field is present, the mean field approximation is the same as neglecting the interaction all together. There is however a way to (partially) remedy this using the socalled self-energy. The self-energy is the energy shift a fermion experiences due to its interaction with all the others. It is clear, that this self-energy will increase as the wires are brought closer together. 

\section{Kitaev chain} \label{sec.Discussion.KitaevChain}
In this thesis we have studied fermionic one-dimensional \textit{gasses}. The original work of Kitaev is based on fermions sitting in a one-dimensional \textit{lattice} with nearest neighbour hopping and pairing \cite{KitaevQuantumWires}. In recent work by Alecce and Dell'Anna an extended version of the Kitaev chain with $r$ neighbours is considered \cite{Alecce.extendKitaev}. They find, that if the hopping and pairing decays sufficiently slow, topological phases with up to $r$ Majorana edge states can be found, hence exceeding the Kitaev result of a single Majorana edge state. 

In this connection we have briefly studied, whether this is possible in our 1D-3D Fermi-Bose mixture. More specifically we think of the fermions as sitting in an optical lattice with nearest and next-nearest neighbour hopping. We must then self-consistently solve for the pairing, $\Delta_k$, and compute the $\mathbb{Z}$-topological invariant $\nu = 2\text{CS}_1$. Since we consider two neighbours, we search for $\nu = \pm 2$. Unfortunately, we have not been able to find such a solution. We speculate, that the reason for this is, that the Yukawa interaction is simply not long range enough. This is supported by the fact, that we can find $\nu = \pm 2$ solutions, when the interaction drops of as $|x|^{-\alpha}$, with $\alpha < 1$. This is also consistent with the findings of Alecce and Dell'Anna. Specifically, they show that the hopping and pairing both have to decay with $\alpha < 1$ for the number of Majorana edge states to exceed 1. This way of producing several edge states systems in the Fermi-Bose mixture framework is therefore considered inapplicable. 


