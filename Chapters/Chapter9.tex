\chapter{Conclusion} 
\label{Chapter10} 
\lhead{Part V. \emph{Conclusions}}
\chead{Chapter 10. \emph{Conclusion}} 

We have studied fermions in one dimension embedded in boson gasses. For the two parallel fermion wires in a three-dimensional Bose-Einstein condensate we have shown, that a point interaction between the fermions and bosons lead to an induced attractive interaction between the fermions of the Yukawa form in the weak-coupling limit. In a mean field approach we have found selfconsistent numerical solutions for the pairing and chemical potentials. The intrawire interaction is shown to lead to a $p$-wave pairing. In turn the separated wires are shown to have a topologically nontrivial ground state. This is verified explicitly by finding an approximate solution for the resulting edge states at the ends of the wires. The interwire interaction is shown to lead to a competing $s$-wave pairing. The resulting Hamiltonian has the structure of a spin-$1/2$ system with $p$-wave pairing between identical spins and $s$-wave pairing between opposite spins. Hence, it describes interacting Kitaev wires. The competition between the pairings is shown to be controllable through the interwire distance or analogously the Bose-Einstein coherence length. 

The edge states in each wire in the separated wire system are shown to be Kramers partners in accordance with a time reversal symmetry, that squares to minus the identity. By calculating the topological invariant we show, that there are two possible and physically distinct configurations of the system. These configurations are different only in the phase of the interwire pairing. In one, as the wires are brought closer together, the interwire pairing chooses a phase, that obeys the time reversal symmetry. In turn a topological phase transition, where the bulk energy gap closes, must occur as the wires are brought closer together. In the other, the interwire pairing chooses a phase, that breaks the time reversal symmetry. The Kramers partners of edge states are then shown to couple and gain a nonzero energy. In a numerical analysis we find the selfconsistent solutions for the pairing and chemical potentials. We hereby show, that the configuration that breaks the time reversal symmetry is the energetically favourable. Using mean field theory, we therefore predict, that the two wire system will undergo a nontopological phase transition between a topological $p$-wave pairing and trivial $s$-wave pairing phase. This is the main result of the thesis. 

Finally, we come with a proposal for a system exhibiting several Majorana edge states. This is a Kitaev chain of identical fermions with both nearest and next-nearest neighbour hopping. Further, the boson gas is made one-dimensional to make the interaction longer range. We show, that this leads to the possibility of dominant next-nearest neighbour pairing. In turn a phase with two edge states is realised. We further show, that this solution is not energetically favourable. Rather a pairing with dominant nearest neigbbour hopping is. However, the energy difference is only of a few percent. We therefore speculate, that it is possible to make the next-nearest neighbour pairing favourable by some perturbation to the system. 

We emphasize, that the mean field approach is problematic for these one-dimensional systems. However, we speculate that in the long coherence length limit, the results will be qualitatively correct. 