% Chapter 12

\chapter{Grand Hamiltonian} % Main chapter title

\label{Chapter12} % For referencing the chapter elsewhere, use \ref{Chapter12} 

\lhead{Part IV. \emph{The Kitaev chain}}
\chead{Chapter 12. \emph{Grand Hamiltonian}} % This is for the header on each page - perhaps a shortened title
%----------------------------------------------------------------------------------------
In this chapter we derive the grand Hamiltonian for the fermionic chain. In section \ref{sec.TightbindingHam.lattice} we introduce the tight-binding Hamiltonian. In section \ref{sec.interaction.lattice} we discuss the interaction mediated by the Bose-Einstein condensate and write up the effective interaction Hamiltonian for the fermions. Finally in section \ref{sec.grandhamiltonian.lattice} we outline the mean field approximation and write up the grand Hamiltonian. 

\section{Tight-binding Hamiltonian} \label{sec.TightbindingHam.lattice}
In this section we introduce the tight-binding Hamiltonian. 

The fermions are arranged in a one dimensional lattice with lattice constant $a$ and $N$ sites. The Hamiltonian for nearest and next-nearest hopping is then:
\begin{equation}
H_{0} = \frac{1}{2}\sum_{j} \left[- t_1(c^\dagger_{j+1}c_j + c^\dagger_j c_{j + 1}) - t_2(c^\dagger_{j + 2}c_j + c^\dagger_j c_{j + 2}) \right].
\label{eq.Htightbindingrealspace} 
\end{equation}
Here $c_j$ is a fermionic annihilation operator for a fermion in site $j$. $t_1$ and $t_2$ are the energies associated with nearest-neighbour and next-nearest-neighbour hopping respectively. We define the corresponding momentum space operators through: $c_j = \frac{1}{\sqrt{N}}\sum_{k} c_k \text{e}^{ikx_j}$, where $ka = \frac{2\pi n}{N}$ are the allowed momenta for $n = -\frac{N}{2}, \dots, \frac{N}{2} - 1$ for periodic boundary conditions and $x_j = ja$ the positions of the fermions. Inserting this into the above transforms the Hamiltonian to momentum space:
\begin{equation}
H_{0} = \sum_k \left[ - t_1\cos(ka) - t_2\cos(2ka)\right]c^\dagger_kc_k.
\label{eq.Htightbindingmomentumspace} 
\end{equation}
In this sense, we can already see, that going from the gas to the chain corresponds to letting $\frac{k^2}{2m_F} \to - t_1\cos(ka) - t_2\cos(2ka)$. It is hereby natural to use $t_1$ as the unit of energy and $a$ as the unit of length.

\section{Interaction Hamiltonian} \label{sec.interaction.lattice}
We return to the Bose-Fermi interaction Hamiltonian of equation \eqref{eq.HintBF}:
\begin{equation}
H_{BF}^\text{int} = g_{BF}\int d^3 r \; \hat{\psi}_F^\dagger(\mathbf{r}) \hat{\psi}_B^\dagger(\mathbf{r})\hat{\psi}_B(\mathbf{r})\hat{\psi}_F(\mathbf{r}),
\end{equation} 
This time around we trap the fermions along a wire as before \textit{and} in a lattice. Assuming, that the trapping frequency, $\omega_{\perp}$ for the perpendicular directions to the wire is large with respect to the typical energy $t_1$ means, that the fermions are trapped in the harmonic oscillator ground state, the gaussian: $\phi_0(\mathbf{r}_{\perp})$ of width $l_{\perp} = \frac{1}{\sqrt{m_F\omega_{\perp}}}$. Further, the lattice confinement means, that the states can be expanded in terms of well-localised (orthonormal) Wannier states. We approximate these states with gaussians $\phi_0(x - x_j)$ of width $l_x$. In this sense, we expand the field operators according to:
\begin{equation}
\hat{\psi}_F(x, \mathbf{r}_{\perp}) = \sum_j \phi_0(\mathbf{r}_{\perp})\phi_0(x - x_j) c_j, 
\end{equation}
where $c_j$ annihilates a fermion at site $j$. Using this we can run through the same steps as taken in the single and double wire system leading to the induced interactions. If we let $l_{\perp}$ and $l_x$ go to zero, the real space induced interaction is unchanged from the single wire case. The zero frequency component is therefore still the Yukawa potential:
\begin{equation}
\tilde{V}_{\text{ind}}(j, l) = -\frac{m_Bg_{BF}^2n_B}{\pi}\frac{\text{e}^{-\frac{\sqrt{2}|x_j - x_l|}{\xi}}}{|x_j - x_l|},
\label{eq.Vindx.lattice}
\end{equation}
which describes the interaction between sites $j$ and $l$. This is lattice periodic, as it should be. The factor in front measures the strength of the interaction, $G$. On unitless form we write this:
\begin{equation}
\frac{G}{at} = \frac{8}{\pi}\left( \frac{m_B}{m_F} + \frac{m_F}{m_B} + 2 \right) \frac{\varepsilon_0}{t} n_B^{1/3}a \left(n_B a^3_{BF}\right)^{2 / 3}, \hspace{0.5cm} \varepsilon_0 = \frac{\pi^2}{2m_Fa^2}.
\label{eq.interactionstrength.lattice}
\end{equation}
Here $\varepsilon_0$ is the energy a single particle would have in an infinite square well of width $a$. We hereby choose to express the system using the independent variables: $\frac{m_B}{m_F}, n_B^{1/3}a, \frac{\varepsilon_0}{t}, \left(n_B a^3_{BF}\right)^{1 / 3}$, and finally the Bose gas parameter: $\left(n_B a^3_{B}\right)^{1 / 3}$. In this we assume, that we can separately control the hopping $t_1$ and the lattice constant $a$. This is the reason for the additional parameter $\frac{\varepsilon_0}{t}$. The fermion interaction Hamiltonian in the zero frequency limit is then given by:
\begin{equation}
H^{\text{int}}_{FF} = \frac{1}{2}\sum_{j,l} c^\dagger_j c^\dagger_l \tilde{V}_{\text{ind}}(j, l) c_l c_j
\label{eq.Hintrealspace.lattice}
\end{equation}
Using the Fourier decomposition $c_j = \frac{1}{\sqrt{N}}\sum_k \text{e}^{ikx_j}c_k$, with $N$ the number of sites, we transform this to momentum space:
\begin{align}
H^{\text{int}}_{FF} &= \frac{1}{2N} \sum_{k, q, p} W_{\text{ind}}(k, q, p) c^\dagger_{k + p} c^\dagger_{q - p} c_q c_k, \nonumber \\  
W_{\text{ind}}(k, q, p) &= \frac{1}{2}\sum_{j\neq 0} \left[\cos(px_j) - \cos((p + k - q)x_j) \right]\tilde{V}_{\text{ind}}(0, j). 
\label{eq.Hintmomentumspace.lattice}
\end{align}
We notice, that $W_{\text{ind}}(k, q, p)$ is explicitly periodic in the Brillouin zone of width $2\pi / a$. The functional difference between the wire and the lattice is hereby, that the momentum space interaction $W_{\text{ind}}(k, q, p)$ is a finite sum of components of the real space interaction $\tilde{V}_{\text{ind}}(0, j)$, not a combination of Fourier transforms. 

\section{Grand Hamiltonian, gap equation and filling fraction} \label{sec.grandhamiltonian.lattice}
The mean field approximation is performed in exactly the same way as for the single wire. The sum in equation \eqref{eq.Hintmomentumspace.lattice} is truncated to $q = -k$, and the operator $c_kc_{-k}$ is written as it's mean plus a deviation. The deviation is then only kept up to first order. See section \ref{sec.meanfieldapproximation} for the details. This means, that the Hamiltonian has the exact same structure as for the single wire, but now we use $W_{\text{ind}}(k, k') = W_{\text{ind}}(k, q = -k, p = k' - k)$ of equation \eqref{eq.Hintmomentumspace.lattice} and $\varepsilon_k = - t_1\cos(ka) - t_2\cos(2ka) - \mu$. The system hereby realises the Kitaev chain with next-nearest hopping included: 
\begin{equation}
H_{FF} = \frac{1}{2}\sum_k \left[\varepsilon_k - \Delta_k \braket {c^\dagger_k c^\dagger_{-k}}\right] + \frac{1}{2}\sum_{k} \begin{bmatrix} c_k^\dagger & c_{-k} \end{bmatrix} \mathcal{H}_{FF,k} \begin{bmatrix} c_k \\ c^\dagger_{-k} \end{bmatrix}, \hspace{0.5cm} \mathcal{H}_{FF,k} = \begin{bmatrix} \varepsilon_k & \Delta_k \\ \Delta^*_k & -\varepsilon_k \end{bmatrix}, 
\label{eq.grandhamiltonian.lattice}
\end{equation}
where the sum over $k$ extends over the interval $[-\pi/a, \pi/a )$, and: 
\begin{equation}
\Delta_k = - \frac{1}{N}\sum_{k'} W_{\text{ind}}(k,k')\braket{c_{k'}c_{-k'}}.
\label{eq.pairingpotential.lattice}
\end{equation}
The Hamiltonian is diagonalised in the same manner as the single wire, yielding a dispersion $E_{F,k} = \sqrt{\varepsilon^2_k + |\Delta_k|^2}$. Further using the transformation to the quasiparticle operators, we obtain the gap equation for the pairing potential: 
\begin{equation}
\Delta_k = - \frac{1}{N}\sum_{k'} W_{\text{ind}}(k,k')\frac{\Delta_{k'}}{2E_{F,k'}}\tanh\left[\frac{\beta E_{F,k'}}{2}\right].
\label{eq.gapequation.lattice}
\end{equation}
So far everything has been analogous to the single wire system. In that system we externally set the number of fermions. In turn the chemical potential adjusted according to the number equation. For the lattice we will externally set the chemical potential. The number equation $N_F = \sum_k \braket{c^\dagger_k c_k}$ then determines the filling fraction $N_F / N$:
\begin{equation}
\frac{N_F}{N} = \frac{1}{N}\sum_k |v_{F,k}|^2 = \frac{1}{2}\left(1 - \frac{1}{N}\sum_k \frac{\varepsilon_k}{E_{F,k}}\tanh\left[\frac{\beta E_{F,k}}{2} \right] \right), 
\label{eq.fillingfraction.lattice}
\end{equation}
which measures how full the lattice is. Since the lattice consists of identical fermions the filling fraction is less than 1: $0 \leq \frac{N_F}{N} \leq 1$. For $\Delta_k \to 0$ the above filling fraction hits exactly 1 for $T = 0$. Hence, a nonzero pairing makes the filling fraction deviate from 1. This is also physically intuitive: the lattice can only be a superfluid, if there are a significant number of vacant sites. Else the lattice turns rigid, since the fermions can only hop to vacant sites. 



