% Chapter 3

\chapter{Induced interaction} % Main chapter title

\label{Chapter3} % For referencing the chapter elsewhere, use \ref{Chapter3} 

\lhead{Part II. \emph{One wire}}
\chead{Chapter 3. \emph{Induced interaction}} % This is for the header on each page - perhaps a shortened title

%----------------------------------------------------------------------------------------
In this chapter we derive the induced interaction between the fermions on the wire. First stop is to look at the effective interaction between the fermions on the wire and the bosons in the cloud.
\section{Effective Bose-Fermi interaction}
The effective interaction between the fermions ($F$) and the bosons ($B$) is modelled by a delta function potential with strength $g_{BF}$: $V(\mathbf{r})=g_{BF}\delta(\mathbf{r})$. This means, that the \textit{int}eraction Hamiltonian for the effective pair interaction reads:
\begin{equation}
H_{BF}^\text{int}  = \int d^3 r d^3 r' \; \hat{\psi}_F^\dagger(\mathbf{r}) \hat{\psi}_B^\dagger(\mathbf{r}')V(\mathbf{r}-\mathbf{r}')\hat{\psi}_B(\mathbf{r}')\hat{\psi}_F(\mathbf{r}) = g_{BF}\int d^3 r \; \hat{\psi}_F^\dagger(\mathbf{r}) \hat{\psi}_B^\dagger(\mathbf{r})\hat{\psi}_B(\mathbf{r})\hat{\psi}_F(\mathbf{r}),
\label{eq.HintBF}
\end{equation}
where $\hat{\psi}_i(\mathbf{r})$ is the field operator for the $i$-particles, and so $\hat{\psi}_i^\dagger(\mathbf{r})$ creates a particle at position $\mathbf{r}$.\footnote{The subscript $BF$ specifies, that it is the Hamiltonian for the interaction between bosons and fermions.} Notice, that there is no factor of $1/2$ in front of the integrals. As noted in section \ref{sec.secondquantization} the factor of $1/2$ should be omitted, when the particles are distinguishable, as fermions and bosons most certainly are. 

The fermions ($F$) are confined to a one dimensional wire. Let this be along the $x$ axis. The confinement is provided by two harmonic traps in the $y$- and $z$-directions with the same trapping frequency $\omega_t$.\footnote{This is mostly for simplicity. It can easily be generalized to different trapping frequencies $\omega_y$ and $\omega_z$ for the two directions.} The energy gap from the ground state to the first excited state is therefore $\omega_t$ in the perpendicular directions $y$ and $z$. Hence by making $\omega_t$ sufficiently large we trap the fermions in the lowest lying state: $\phi_0(\mathbf{r}_\perp) = \frac{1}{\sqrt{\pi}l_t}\exp\left(-\frac{r_\perp^2}{2l_t^2}\right)$, where $\mathbf{r}_\perp = (y,z)$ is the perpendicular direction away from the wire, and $l_t = \frac{1}{\sqrt{m_F \omega_t}}$ is roughly the length over which the fermions spread out in the two perpendicular directions $y$ and $z$. The typical energy of the fermions is the free gas Fermi energy $\epsilon_{F,0} = \frac{k_F^2}{2m_F}$. Hence, we require that $\frac{\epsilon_{F,0}}{\omega_t} \ll 1$ or equivalently $(k_Fl_t)^2\ll 1$. One might be afraid of the Pauli exclusion principle in this context. However, the states in the $x$-direction are allowed to be different, and so we can symmetrize in the $y$- and $z$-directions. This means, that we can expand the field operators in the following fashion:
\begin{equation}
\psi_F(x,\mathbf{r}_\perp) = \frac{1}{\sqrt{\mathcal{L}}}\sum_p \text{e}^{ipx} \phi_0(\mathbf{r}_\perp) c_p, \hspace{0.5cm} \psi_B(\mathbf{r}) = \frac{1}{\sqrt{\mathcal{V}}}\sum_{\mathbf{k}} \text{e}^{i\mathbf{k}\cdot \mathbf{r}} b_\mathbf{k}. 
\end{equation}   
Hence, we expand the fields in plane wave states. $b_\mathbf{k}$ and $c_p$ denote respectively bosonic and fermionic annihilation operators with momentum $\mathbf{k}$ and $p$. By inserting this into $H_{BF}^\text{int}$ and expressing $\mathbf{k} = (q,\mathbf{k}_\perp)$ we get:
\begin{equation}
H_{BF}^\text{int} = \frac{g_{BF}}{\mathcal{LV}}\int d^3 r \sum_{p,p',q,q'}\sum_{\mathbf{k}_{\perp},\mathbf{k}_{\perp}'}\text{e}^{i((p-p')+(q-q'))x} \phi^2_0(\mathbf{r}_{\perp})\text{e}^{i(\mathbf{k}_{\perp}-\mathbf{k}_{\perp}')\cdot \mathbf{r}_\perp} c^\dagger_{p'} b^\dagger_{q',\mathbf{k}_\perp'}b_{q,\mathbf{k}_\perp}c_p. \nonumber
\end{equation}
The integral over $x$ yields the factor $\mathcal{L}\delta_{p-p',q'-q}$, implying total momentum conservation along the wire. The integral over $\mathbf{r}_\perp$ gives the Fourier transform of $\phi_0^2(\mathbf{r}_\perp)$. Since $\phi_0$ is a gaussian the Fourier transform is as well. Explicitly:
\begin{equation}
\int d^2 r_\perp \; \phi^2_0(\mathbf{r}_{\perp})\text{e}^{i(\mathbf{k}_{\perp}-\mathbf{k}_{\perp}')\cdot \mathbf{r}_\perp} = \text{e}^{-\frac{l_t^2}{4}(\mathbf{k}_{\perp}-\mathbf{k}_{\perp}')^2}. \nonumber
\end{equation}
Taking the sum over $q'$ forces $q' = p-p' + q$. In total:
\begin{align}
H_{BF}^\text{int} &= \frac{g_{BF}}{\mathcal{V}}\sum_{p,p',q} \sum_{\mathbf{k}_\perp, \mathbf{k}_\perp'} \text{e}^{-\frac{l_t^2}{4}(\mathbf{k}_\perp-\mathbf{k}_\perp')^2} c^\dagger_{p'} b^\dagger_{p-p'+q, \mathbf{k}_\perp'} b_{q,\mathbf{k}_\perp}c_{p} \nonumber \\
                  &= \frac{g_{BF}}{\mathcal{V}}\sum_{p_1,p_2,q} \sum_{\mathbf{k}_\perp, \mathbf{k}_\perp'} \text{e}^{-\frac{l_t^2}{4}(\mathbf{k}_\perp-\mathbf{k}_\perp')^2} c_{p_2-q}^\dagger b_{p_1+q, \mathbf{k}_\perp'}^\dagger b_{p_1,\mathbf{k}_\perp}c_{p_2}.
\end{align}
The last line is obtained by appropriate renaming. To be very specific: $p_1,p_2$ and $q$ are in the $x$ direction, along the wire, whilst $\mathbf{k}_\perp'$ and $\mathbf{k}_\perp$ are momenta in the $(y,z)$-plane, hence perpendicular to the wire. A relatively obvious thing one might do now, is to take the limit $\omega_t \to \infty$ or equivalently $l_t \to 0$. However, as we shall see later, it will be crucial to keep the trapping frequency finite, at least at the present stage. This interaction Hamiltonian shows, that the pair interaction between the fermions and bosons are associated with a factor $g_{BF}\; \text{e}^{-\frac{l_t^2}{4}(\mathbf{k}_\perp'-\mathbf{k}_\perp)^2}$, the transverse momenta being free. 

\section{1D-3D induced interaction}
The bosons are assumed to be in a perfect Bose-Einstein Condensate (BEC). This means, that the bosons are essentially lying still.\footnote{There is a small fraction in nonzero momenta states, but these are not considered.}  We are further only interested in the weak coupling limit. In a more general setup, one would have to calculate contributions from increasing orders in the underlying bare interaction leading to a socalled $T$-matrix, but this will not be pursued here. We will write the coupling strength as $g_{BF} = \frac{2\pi a_{BF}}{m_r}$, with $m_r = \frac{m_Fm_B}{m_B + m_F}$ the reduced mass, and $a_{BF}$ the effective scattering length. In a formally precise manner one can expand the $T$-matrix in orders of $a_{BF}$. From this analysis, it is clear that the smallness of the coupling strength is equivalent to demanding $(n_Ba_{BF}^3)^{1/3}\ll 1$. We can also argue for this by a dimensional analysis. From the form of $g_{BF}$ it is clear, that we should argue from $a_{BF}$. This has dimension of length, and so we need a parameter of dimension per length to get a unitless expression. The only reasonable parameter is the one that describes how many bosons, the fermions can actually interact with, leading to a length scale $n_B^{-1/3}$. Since a higher density of bosons leads to more fermion-boson interactions, we should therefore assume $(n_Ba_{BF}^3)^{1/3} \ll 1$. It follows, that the four essential diagrams for the fermion-fermion induced interaction are the ones showed in figure \ref{fig.feynmandiagrams}. 

\begin{figure}
\begin{tikzpicture}[scale=0.25]
  \begin{feynman}[small]
    \vertex (number1) {\( (1) \)};
    \vertex [above left=of number1] (fermion1) {\( \tilde{p}_1 \)};
    \vertex [above right=of fermion1] (a);
    \vertex [below right=of a] (fermion2) {\(\tilde{p}_1+\tilde{q}\)}; 
    \vertex [above=of a] (b);
    \vertex [left=of b] (boson1) {\( \sqrt{n_B} \)}; 
    \vertex [above= of b] (c);
    \vertex [right= of c] (boson2) {\( \sqrt{n_B} \)};
    \vertex [above= of c] (d);
    \vertex [above left=of d] (f3) {\(\tilde{p}_2\)};
    \vertex [above right=of d] (f4) {\(\tilde{p}_2-\tilde{q}\)};
 
    \diagram* {
      (number1) -- [opacity=0.0] (fermion1) -- [fermion] (a) -- [fermion] (fermion2),
      (a) -- [photon, edge label'=\(g_{BF}\)] (b),
      (b) -- [dashed] (boson1),
      (b) -- [blue, fermion, edge label' = {\(-\tilde{q}, \mathbf{k}_\perp \)}] (c),
      (c) -- [dashed] (boson2),
      (c) -- [photon, edge label'=\(g_{BF}\)] (d),
      (d) -- [anti fermion] (f3),
      (d) -- [fermion] (f4)
    };
  \end{feynman}
\end{tikzpicture}
\begin{tikzpicture}
  \begin{feynman}[small]
    \vertex (number2) {\( (2) \)};
    \vertex [above left=of number2] (fermion1) {\( \tilde{p}_1 \)};
    \vertex [above right=of fermion1] (a);
    \vertex [below right=of a] (fermion2) {\(\tilde{p}_1+\tilde{q}\)}; 
    \vertex [above=of a] (b);
    \vertex [left=of b] (boson1) {\( \sqrt{n_B} \)}; 
    \vertex [above= of b] (c);
    \vertex [right= of c] (boson2) {\( \sqrt{n_B} \)};
    \vertex [above= of c] (d);
    \vertex [above left=of d] (f3) {\(\tilde{p}_2\)};
    \vertex [above right=of d] (f4) {\(\tilde{p}_2-\tilde{q}\)};
 
    \diagram* {
      (number2) -- [opacity=0.0] (fermion1) -- [fermion] (a) -- [fermion] (fermion2),
      (a) -- [photon, edge label'=\(g_{BF}\)] (b),
      (b) -- [dashed] (boson1),
      (b) -- [blue, anti fermion, edge label' = {\(\tilde{q}, \mathbf{k}_\perp \)}] (c),
      (c) -- [dashed] (boson2),
      (c) -- [photon, edge label'=\(g_{BF}\)] (d),
      (d) -- [anti fermion] (f3),
      (d) -- [fermion] (f4)
    };
  \end{feynman}
\end{tikzpicture}
\begin{tikzpicture}
  \begin{feynman}[small]
    \vertex (number3) {\( (3) \)};
    \vertex [above left=of number3] (fermion1) {\( \tilde{p}_1 \)};
    \vertex [above right=of fermion1] (a);
    \vertex [below right=of a] (fermion2) {\(\tilde{p}_1+\tilde{q}\)}; 
    \vertex [above=of a] (b);
    \vertex [left=of b] (boson1) {\( \sqrt{n_B} \)}; 
    \vertex [above= of b] (c);
    \vertex [right= of c] (boson2) {\( \sqrt{n_B} \)};
    \vertex [above= of c] (d);
    \vertex [above left=of d] (f3) {\(\tilde{p}_2\)};
    \vertex [above right=of d] (f4) {\(\tilde{p}_2-\tilde{q}\)};
 
    \diagram* {
      (number3) -- [opacity=0.0] (fermion1) -- [fermion] (a) -- [fermion] (fermion2),
      (a) -- [photon, edge label'=\(g_{BF}\)] (b),
      (b) -- [dashed] (boson1),
      (b) -- [blue, majorana, edge label' = {\(\tilde{q}, \mathbf{k}_\perp \)}] (c),
      (c) -- [dashed] (boson2),
      (c) -- [photon, edge label'=\(g_{BF}\)] (d),
      (d) -- [anti fermion] (f3),
      (d) -- [fermion] (f4)
    };
  \end{feynman}
\end{tikzpicture}
\begin{tikzpicture}
  \begin{feynman}[small]
    \vertex (number4) {\( (4) \)};
    \vertex [above left=of number4] (fermion1) {\( \tilde{p}_1 \)};
    \vertex [above right=of fermion1] (a);
    \vertex [below right=of a] (fermion2) {\(\tilde{p}_1+\tilde{q}\)}; 
    \vertex [above=of a] (b);
    \vertex [left=of b] (boson1) {\( \sqrt{n_B} \)}; 
    \vertex [above= of b] (c);
    \vertex [right= of c] (boson2) {\( \sqrt{n_B} \)};
    \vertex [above= of c] (d);
    \vertex [above left=of d] (f3) {\(\tilde{p}_2\)};
    \vertex [above right=of d] (f4) {\(\tilde{p}_2-\tilde{q}\)};
 
    \diagram* {
      (number3) -- [opacity=0.0] (fermion1) -- [fermion] (a) -- [fermion] (fermion2),
      (a) -- [photon, edge label'=\(g_{BF}\)] (b),
      (b) -- [dashed] (boson1),
      (b) -- [blue, anti majorana, edge label' = {\(\tilde{q}, \mathbf{k}_\perp \)}] (c),
      (c) -- [dashed] (boson2),
      (c) -- [photon, edge label'=\(g_{BF}\)] (d),
      (d) -- [anti fermion] (f3),
      (d) -- [fermion] (f4)
    };
  \end{feynman}
\end{tikzpicture}
\caption{Feynman diagrams for the induced interaction. Since the interaction is weak, we can neglect all other Feynman diagrams than (1)-(4). Diagrams (1) and (2) stems from the normal Green's function $G_{11}$. Diagrams (3) and (4) stems from the anormalous Green's functions $G_{12}$ and $G_{21}$.} 
\label{fig.feynmandiagrams}
\end{figure}

Notice that the bosons of the BEC, shown with dashed lines, carry a factor of $\sqrt{n_B}$. This is simply because it is the pour condensate wave function. Further $\tilde{p}_j = (p_j, i\omega_{m_j})$, with $\omega_{m} = (2m+1)\pi kT$ a fermionic Matsubara frequency, and $\tilde{q} = (q, i\omega_q )$, with $\omega_q = \omega_{m_1}-\omega_{m_2}$ a bosonic Matsubara frequency. The blue lines in the diagrams are the normal and anormalous Bogoliubov phonon Green's functions derived in subsection \ref{sec.BECGreens}. Functionally they are:
\begin{equation}
G_{11}(\mathbf{k},i\omega_m) = \frac{u_{B,k}^2}{i\omega_m-E_{B,k}}-\frac{v_{B,k}^2}{i\omega_m+E_{B,k}}, \hspace{0.5cm} G_{12}(\mathbf{k},i\omega_m) = G_{21}(\mathbf{k},i\omega_m) = \frac{u_{B,k}v_{B,k}}{i\omega_m+E_{B,k}}-\frac{u_{B,k}v_{B,k}}{i\omega_m-E_{B,k}}.\nonumber
\end{equation}
Here $u_{B,k}$ and $v_{B,k}$ are the BEC coherence factors found in subsection \ref{sec.BEC} with $m_B$ the mass of the bosons. Further $\xi_{B,k} = \frac{k^2}{2m_B}+n_Bg_B$ and $g_B = \frac{4\pi a_B}{m_B}$, with $a_B$ the scattering length in the BEC. Finally $E_{B,k}^2 = \xi_{B,k}^2-(n_Bg_B)^2$ is the BEC Bogoliubov spectrum.

The diagrams can intuitively be understood as follows. A fermion on the wire interact with a (real) boson in the condensate. This creates a ripple in the condensate described by one of the Green's functions. This ripple reaches a second fermion on the wire, where the momentum of the ripple is transferred to. Further, a (real) boson is 'returned' to the condensate. In a more technical way we say, that the fermions interact by exchanging a condensate Bogoliubov phonon. 

Since the perpendicular momentum $\mathbf{k}_\perp$ is completely free we have to integrate over this. Since the momentum of the incoming and outgoing boson in each diagram is 0 we simply get a factor of $g_{BF}\; \text{e}^{-\frac{l_t^2}{4}k_\perp^2}$ \textit{twice}. Collecting all terms gives us the four contributions from the four diagrams (1)-(4) to the fermion-fermion induced interaction $V_{\text{ind}}$: 
\begin{align}
V_{\text{ind},1}(q,i\omega_q) &= n_Bg_{BF}^2\int\frac{d^2k_\perp}{(2\pi)^2}G_{11}(-q,\mathbf{k}_\perp,-\omega_q)\text{e}^{-\frac{l_t^2}{2}k_\perp^2}, \nonumber \\
V_{\text{ind},2}(q,i\omega_q) &= n_Bg_{BF}^2\int\frac{d^2k_\perp}{(2\pi)^2}G_{11}(q,\mathbf{k}_\perp,\omega_q)\text{e}^{-\frac{l_t^2}{2}k_\perp^2}, \nonumber \\
V_{\text{ind},3}(q,i\omega_q) &= n_Bg_{BF}^2\int\frac{d^2k_\perp}{(2\pi)^2}G_{12}(q,\mathbf{k}_\perp,\omega_q)\text{e}^{-\frac{l_t^2}{2}k_\perp^2}, \nonumber \\
V_{\text{ind},4}(q,i\omega_q) &= n_Bg_{BF}^2\int\frac{d^2k_\perp}{(2\pi)^2}G_{12}(q,\mathbf{k}_\perp,\omega_q)\text{e}^{-\frac{l_t^2}{2}k_\perp^2}. 
\end{align}
The last two we see, give the same contribution in this weak interacting limit.\footnote{This would \textit{not} be the case, when including higher order terms.} Summing up the four contributions gives us the frequency dependent induced interaction $V_{\text{ind}}$:
\begin{equation}
V_{\text{ind}}(q,i\omega_q) = g_{BF}^2\int\frac{d^2k_\perp}{(2\pi)^2}\; \chi_\text{BEC}(q,\mathbf{k}_\perp,i\omega_q)\text{e}^{-\frac{l_t^2}{2}k_\perp^2}, 
\label{eq.VFFindXBEC}
\end{equation}
with $\chi_\text{BEC}(\mathbf{k},i\omega_q) = \frac{k^2}{m_B}\frac{n_B}{(i\omega_q)^2-E_{B,k}^2}$ the socalled density-density correlation function of the BEC. It now becomes clear, why we had to retain a nonzero value of $l_t$ at this stage. For $l_t\to 0$ the above has an integrand of the form $k^2/(a+bk^4+ck^2)$, with $a,b,c$ positive constants. As a result the integral is logarithmically divergent. The situation in a 2D-3D system is different. Here there is a single integral in the above, and the result converges for $l_t\to 0$. 

\section{Retardation effects} \label{sec.RetardationEffects}
In this section we will discuss the $\omega_q = 0$ limit. 

From equation \eqref{eq.VFFindXBEC} it is evident that the induced interaction has a frequency dependency. This dependency reflects, that the fermions do not interact instanteneously. This is also called retardation effects. In turn this embodies, that the bosons in the condensate, the mediators of the fermion-fermion interaction, moves at a finite speed $c_0 = \frac{\sqrt{4\pi n_B a_B}}{m_B}$.\footnote{It is analogous to the retarded fields in electrodynamics. There it reflects the finiteness of the speed of light.} To neglect these effects we therefore need to assume, that the typical speed of the fermions is much smaller than the speed of the bosons: $v_F \ll c_0$, $v_F = k_F/m_F$ the Fermi speed for free fermions. This leads to the relation:
\begin{equation}
1 \gg \frac{v_F}{c_0} = \frac{\sqrt{\pi}}{2} \frac{m_B}{m_F}\frac{1}{ \sqrt{ (n_Ba_B^3)^{1/3} } }\frac{n_F}{ n_B^{1/3} }, 
\label{eq.RetardationEffectsneglectionassumption}
\end{equation}
whereby we have expressed the ratio of velocities in terms of unitless quantities. With this assumption at hand we can focus on the zero frequency induced interation $V_{\text{ind}}(q,0)$. 

\section{Zero frequency induced interaction}
Let us now look at the zero frequency induced interaction $V_{\text{ind}}(q,0)$. We can write this interaction in two equivalent ways:
\begin{equation}
V_{\text{ind}}(q,0) = -\frac{m_Bg_{BF}^2n_B}{\pi}\int_{0}^\infty dv \frac{\text{e}^{-v}}{v+F(q)} = -\frac{m_Bg_{BF}^2n_B}{\pi} \text{e}^{F(q)} E_1(F(q)),
\label{eq.VFF(q,0)}
\end{equation}
where $F(q) = \frac{l_t^2}{2}\left(q^2+\frac{2}{\xi^2} \right)$, $\frac{1}{\xi^2} = 2m_Bn_Bg_B$ is the BEC coherence length and $E_1(x)$ is the exponential integral: $E_1(x) = \int_1^\infty du \frac{\text{e}^{-xu}}{u}$. I now wish to calculate the position space potential $\tilde{V}_{\text{ind}}(x,0)$. This is of course done by Fourier transforming the above. Using equation \eqref{eq.VFF(q,0)} we obtain:
\begin{equation}
\tilde{V}_{\text{ind}}(x,0) = -\frac{m_Bg_{BF}^2n_B}{2\pi^2}\int_0^\infty dv\;  \text{e}^{-v}\int_{-\infty}^\infty dq \frac{\text{e}^{iqx}}{v+F(q)}.
\end{equation}
Firstly we notice, that since $F(q)$ is even in $q$ the $\text{e}^{iqx}$ can be replaced by $\cos(qx)$. This shows, that the induced interaction is even in $x$ and we can restrict ourselves to $x > 0$. Secondly when we think of $q$ as a complex variable we see, that we can make a half-circle contour $\mathcal{C}$ in the upper complex half plane. Since $x>0$ the integrand goes exponentially fast to 0 at the circle boundary, and we therefore get:
\begin{equation}
\int_{-\infty}^\infty dq \frac{\text{e}^{iqx}}{v+F(q)} = \int_\mathcal{C} dq  \frac{\text{e}^{iqx}}{v+F(q)}. \nonumber
\end{equation}
This integral is solvable using Cauchy's residue theorem. Calculating the residue, plugging this into the $v$-integral, using that $\tilde{V}_{\text{ind}}(x,0)$ is even in $x$ and doing some algebra leads us to the following form:
\begin{equation}
\tilde{V}_{\text{ind}}(x,0) = -\frac{\sqrt{2}m_Bg_{BF}^2n_B}{\pi l_t}\text{e}^{\frac{l_t^2}{\xi^2}+\frac{x^2}{2l_t^2}}\int_{\frac{l_t}{\xi}+\frac{|x|}{\sqrt{2}l_t}}^\infty du \; \text{e}^{-u^2}. \nonumber
\end{equation}
Finally recognizing the last part as proportional to the complementary error function, we arrive at the result:
\begin{equation}
\tilde{V}_{\text{ind}}(x,0) = -\frac{m_Bg_{BF}^2n_B}{\sqrt{2\pi} l_t} \text{e}^{-\frac{\sqrt{2}|x|}{\xi}}\text{e}^{\left(\frac{l_t}{\xi}+\frac{|x|}{\sqrt{2}l_t}\right)^2}\text{erfc}\left(\frac{l_t}{\xi}+\frac{|x|}{\sqrt{2}l_t}\right).
\label{eq.VFFx_exact}
\end{equation}
Notice, that there are \textit{two} relevant length scales: the coherence length $\xi$ and the trapping width $l_t$. For small values of $l_t/|x|$ we can use the asymptotic form: $\text{e}^{y^2}\text{erfc}(y) \to \frac{1}{\sqrt{\pi}y}$ and we get the Yukawa potential in one dimension with range $\xi/\sqrt{2}$: 
\begin{equation}
\lim_{l_t\to 0} \left[ \tilde{V}_{\text{ind}}(x,0) \right] = -\frac{m_Bg_{BF}^2n_B}{\pi}\frac{\text{e}^{-\frac{\sqrt{2}|x|}{\xi}}}{|x|}.
\label{eq.Vx_lt=0}
\end{equation}
For any nonzero value of $l_t$ this will fundamentally fail at $x=0$, since $\tilde{V}_{\text{ind}}(x,0)$ really is finite here. The result for $V_{\text{ind}}(q,0)$ is shown in figure \ref{fig.Vq} and the result for $\tilde{V}_{\text{ind}}(x,0)$ along with the asymptotic form is shown in \ref{fig.Vx}. Both have been done for several values of $l_t$. As expected we see, that the momentum space interaction diverges as $l_t \to 0$, and that the real space interaction asymptotically goes to the Yukawa interaction. The plots are performed by going to a dimensionless form of the interactions; utilizing that the relevant energy scale is the Fermi energy $\epsilon_{F,0} = \frac{k_F^2}{2m_F}$. In detail the asymptotic Yukawa potential in dimensionless form is:
\begin{equation}
\frac{\tilde{V}_{\text{ind}}(x,0)}{\epsilon_{F,0}} = - 8\left( \frac{m_F}{m_B} + \frac{m_B}{m_F} + 2 \right) \frac{n_B^{1/3}}{n_F}(n_Ba_{BF}^3)^{2/3} \cdot \frac{\text{e}^{ -\frac{ \sqrt{2}|x|}{\xi} } }{k_F|x|}.
\label{eq.Vxdimensionless}
\end{equation}
Since the momentum space induced interaction is given by the Fourier transform of induced interaction in real space one unit of $k_F$ is lost to the integration. The dimensionless form is hereby for any $l_t > 0$: 
\begin{equation}
\frac{2m_F}{k_F}V_{\text{ind}}(q,0) = - 8\left( \frac{m_F}{m_B} + \frac{m_B}{m_F} + 2 \right) \frac{n_B^{1/3}}{n_F}(n_Ba_{BF}^3)^{2/3} \cdot \text{e}^{F(q)} E_1(F(q)).
\label{eq.Vqdimensionless}
\end{equation}
The mass ratio $m_F/m_B$, the ratio of interparticle distances $n_B^{1/3}/n_F$ and the Bose-Fermi gas parameter $(n_Ba_{BF}^3)^{1/3}$ hereby comes naturally about by going to dimensionless quantities. 

\begin{figure} 
\begin{center}  
\input{Figures/Vq/plot.tex}  
\caption{The zero frequency potential $V_{\text{ind}}(q,0)$ plotted as a function of $q/k_F$ for several values of $l_t$. Parameters: $(n_Ba_B^3)^{1/3} = 0.01, (n_Ba_{BF}^3)^{1/3} = 0.1$, $\frac{m_B}{m_F} = 7/40$, $\frac{n_F}{n_B^{1/3}} = 0.215$, $v_F/c_0 = 0.33$.}  
\label{fig.Vq}  
\end{center}    
\end{figure}

\begin{figure} 
\begin{center}  
\input{Figures/Vx/plot.tex}  
\caption{The zero frequency potential $\tilde{V}_{\text{ind}}(x,0)$ plotted as a function of $k_Fx$ for several values of $l_t$. The black curve is the asymptotic Yukawa interaction for $l_t \to 0$. Parameters: $(n_Ba_B^3)^{1/3} = 0.01$, $(n_Ba_{BF}^3)^{1/3} = 0.1$, $\frac{m_B}{m_F} = 7/40$, $\frac{n_F}{n_B^{1/3}} = 0.215$, $v_F/c_0 = 0.33$.}  
\label{fig.Vx}  
\end{center}    
\end{figure}


