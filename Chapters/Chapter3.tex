% Chapter 3

\chapter{Induced interaction} % Main chapter title

\label{Chapter3} % For referencing the chapter elsewhere, use \ref{Chapter3} 

\lhead{Part II. \emph{Kitaev wires}}
\chead{Chapter 3. \emph{Induced interaction}} % This is for the header on each page - perhaps a shortened title

%----------------------------------------------------------------------------------------
In this chapter we derive the induced interactions between the fermions on the wires. First stop is to look at the effective interaction between the fermions on the wires and the bosons in the condensate.
\section{Effective Bose-Fermi interaction}
The effective interaction between the fermions ($F$) and the bosons ($B$) is modelled by a delta function potential with strength $g_{BF}$: $V(\mathbf{r})=g_{BF}\delta(\mathbf{r})$. This means, that the \textit{int}eraction Hamiltonian for the effective pair interaction reads:
\begin{equation}
H_{BF}^\text{int}  = \int d^3 r d^3 r' \; \psi_F^\dagger(\mathbf{r}) \psi_B^\dagger(\mathbf{r}')V(\mathbf{r}-\mathbf{r}')\psi_B(\mathbf{r}')\psi_F(\mathbf{r}) = g_{BF}\int d^3 r \; \psi_F^\dagger(\mathbf{r}) \psi_B^\dagger(\mathbf{r})\psi_B(\mathbf{r})\psi_F(\mathbf{r}),
\label{eq.HintBF}
\end{equation}
where $\psi_i(\mathbf{r})$ is the field operator for the $i$-particles, and so $\psi_i^\dagger(\mathbf{r})$ creates a particle at position $\mathbf{r}$.\footnote{The subscript $BF$ specifies, that it is the Hamiltonian for the interaction between bosons and fermions.} Notice, that there is no factor of $1/2$ in front of the integrals. As noted in section \ref{sec.secondquantization} the factor of $1/2$ should be omitted, when the particles are distinguishable, as fermions and bosons most certainly are. 

The fermions ($F$) are confined to two one-dimensional wires. Let the first one be along the $x$-axis, the second at $z = d, y = 0$. The confinement is provided by two harmonic traps in the $y$- and $z$-directions with the same trapping frequency $\omega_t$.\footnote{This is mostly for simplicity. It can easily be generalized to different trapping frequencies $\omega_y$ and $\omega_z$ for the two directions.} In this connection we have two central assumptions. Firstly, we require that the fermions are trapped in the ground state with respect to the perpendicular directions. The energy gap from the ground state to the first excited state is $\omega_t$. Hence by making $\omega_t$ sufficiently large we trap the fermions in the lowest lying state, $\phi_0$, along the two wires. Specifically, the typical energy of the fermions is the free gas Fermi energy $\epsilon_{F,0} = \frac{k_F^2}{2m_F}$. Hence, we require $\frac{\epsilon_{F,0}}{\omega_t} \ll 1$. Using the trapping width $l_t = \frac{1}{\sqrt{m_F\omega_t}}$, we can also express this as $(k_Fl_t)^2 \ll 1$. One might be afraid of violating the Pauli exclusion principle in this context. However, the states in the $x$-direction are allowed to be different, and so we can symmetrize in the $y$- and $z$-directions.

The second assumption is, that the distance between the wires is much larger than the trapping width of the wires, $l_t$. Hence, $l_t/d \ll 1$. This is done, so that we can actually talk about distinguishable wires of fermions. This leads to the following expansion in momentum eigenstates:
\begin{equation}
\psi_F(x,\mathbf{r}_\perp) = \frac{1}{\sqrt{\mathcal{L}}}\sum_p \text{e}^{ipx} \left[\phi_0(\mathbf{r}_\perp) c_{1,p} + \phi_0(\mathbf{r}_\perp - \mathbf{d}) c_{2,p}\right], \hspace{0.5cm} \psi_B(\mathbf{r}) = \frac{1}{\sqrt{\mathcal{V}}}\sum_{\mathbf{k}} \text{e}^{i\mathbf{k}\cdot \mathbf{r}} b_\mathbf{k}, 
\end{equation}  
with $\phi_0(\mathbf{r}_\perp) = \frac{1}{\sqrt{\pi}l_t}\exp\left(-\frac{r_\perp^2}{2l_t^2}\right)$ the harmonic ground state with respect to the perpendicular directions and $\mathbf{d} = d\cdot\hat{z}$ the position of wire 2. $c^\dagger_{j,p}$ creates a fermion in wire $j$ with momentum $p$. $b^\dagger_\mathbf{k}$ creates a boson with momentum $\mathbf{k}$.  We assume, that the wires are truly distinguishable. This means that all anticommutators like $\{c_{1,p}, c^\dagger_{2,p'}\}$ vanish. Inserting these expressions into $H_{BF}^\text{int}$ yields in total four terms. However, the cross terms where a fermion is annihilated in one wire and created in the other are proportional to the integral: $\int d^2 r_\perp \phi_0(\mathbf{r}_\perp)\phi_0(\mathbf{r}_\perp-\mathbf{d}) = 0$. The integral is negligible, because by assumption $l_t/d \ll 1$. The terms arising from interactions with fermions in wire 1 are:
\begin{equation}
H_{BF, 1}^{\text{int}} = \frac{g_{BF}}{\mathcal{LV}}\int d^3 r \sum_{p,p',q,q'}\sum_{\mathbf{k}_{\perp},\mathbf{k}_{\perp}'}\text{e}^{i((p-p')+(q-q'))x} \phi^2_0(\mathbf{r}_{\perp})\text{e}^{i(\mathbf{k}_{\perp} - \mathbf{k}_{\perp}')\cdot \mathbf{r}_\perp} c^\dagger_{1,p'} b^\dagger_{q',\mathbf{k}_\perp'}b_{q,\mathbf{k}_\perp}c_{1,p}. \nonumber
\end{equation}
The integral over $x$ yields the factor $\mathcal{L}\delta_{p-p',q'-q}$, implying total momentum conservation along the wire. The integral over $\mathbf{r}_\perp$ gives the Fourier transform of $\phi_0^2(\mathbf{r}_\perp)$. Since $\phi_0$ is a gaussian, the Fourier transform is as well. Explicitly: $\int d^2 r_\perp \; \phi^2_0(\mathbf{r}_{\perp})\text{e}^{i(\mathbf{k}_{\perp}-\mathbf{k}_{\perp}')\cdot \mathbf{r}_\perp} = \text{e}^{-\frac{l_t^2}{4}(\mathbf{k}_{\perp}-\mathbf{k}_{\perp}')^2}$. Taking the sum over $q'$ forces $q' = p - p' + q$. In total:
\begin{align}
H_{BF, 1}^{\text{int}} &= \frac{g_{BF}}{\mathcal{V}}\sum_{p, p', q} \sum_{\mathbf{k}_\perp, \mathbf{k}_\perp'} \text{e}^{-\frac{l_t^2}{4}(\mathbf{k}_\perp-\mathbf{k}_\perp')^2} c^\dagger_{1, p'} b^\dagger_{p - p' + q, \mathbf{k}_\perp'} b_{q, \mathbf{k}_\perp}c_{1, p} \nonumber \\
                  &= \frac{g_{BF}}{\mathcal{V}}\sum_{p_1, p_2, q} \sum_{\mathbf{k}_\perp, \mathbf{k}_\perp'} \text{e}^{-\frac{l_t^2}{4}(\mathbf{k}_\perp-\mathbf{k}_\perp')^2} c_{1, p_2 - q}^\dagger b_{p_1 + q, \mathbf{k}_\perp'}^\dagger b_{p_1, \mathbf{k}_\perp}c_{1, p_2}.
\end{align}
The last line is obtained by appropriate renaming. To be very specific: $p_1, p_2$ and $q$ are in the $x$ direction, along wire 1, whilst $\mathbf{k}_\perp'$ and $\mathbf{k}_\perp$ are momenta in the $(y,z)$-plane, hence perpendicular to wire 1. The interactions with fermions in wire 2 yields a very similar result. The only alteration is, that $\phi_0(\mathbf{r}_\perp) \to \phi_0(\mathbf{r}_\perp - \mathbf{d})$. The above Fourier transformation therefore yields an additional phase of $\text{e}^{i(\mathbf{k}_\perp - \mathbf{k}_\perp')\cdot \mathbf{d}}$. In total the interaction Hamiltonian hereby becomes:
\begin{align}
H_{BF}^\text{int} = \frac{g_{BF}}{\mathcal{V}}\sum_{p_1,p_2,q} \sum_{\mathbf{k}_\perp, \mathbf{k}_\perp'} & \text{e}^{-\frac{l_t^2}{4}(\mathbf{k}_\perp - \mathbf{k}_\perp')^2}\left[ c^\dagger_{1,p_2-q} b^\dagger_{p_1+q, \mathbf{k}_\perp'} b_{p_1,\mathbf{k}_\perp}c_{1,p_2} + \right. \nonumber \\
& \left. \text{e}^{i(\mathbf{k}_\perp - \mathbf{k}_\perp')\cdot \mathbf{d}}c_{2,p_2-q}^\dagger b_{p_1+q, \mathbf{k}_\perp'}^\dagger b_{p_1,\mathbf{k}_\perp}c_{2,p_2} \right].
\end{align}
This shows, that a scattering event in wire 1 is associated with the factor $g_{BF} \text{e}^{-\frac{l_t^2}{4}(\mathbf{k}_\perp - \mathbf{k}_\perp')^2}$, and that the corresponding scattering in wire 2 has an extra factor of $\text{e}^{i(\mathbf{k}_\perp - \mathbf{k}_\perp')\cdot \mathbf{d}}$. In both cases the transverse momenta are free. An obvious thing to do now would be to take the limit $\omega_t \to \infty$ or equivalently $l_t \to 0$. However, as we shall see later, it will be crucial to keep the trapping frequency finite, at least at the present stage.  

\section{1D-3D induced interactions} \label{sec.1D3Dinducedinteraction}
\subsection{Feynman diagrams} \label{subsec.Feynmandiagrams}
The bosons are assumed to be in a perfect Bose-Einstein condensate (BEC). This means, that they are essentially lying still. We are further only interested in the weak coupling limit.\footnote{There is a small fraction in nonzero momenta states, which is neglected. It would be necessary to include this for stronger interactions.} In a more general setup, one would have to calculate contributions from increasing orders in the underlying bare interaction leading to a socalled $T$-matrix, but this will not be pursued here. We will write the coupling strength as $g_{BF} = \frac{2\pi a_{BF}}{m_r}$, with $m_r = \frac{m_Fm_B}{m_B + m_F}$ the reduced mass, and $a_{BF}$ the effective scattering length. In a formally precise manner one can expand the $T$-matrix in orders of $a_{BF}$. From this analysis, it is clear that the smallness of the coupling strength is equivalent to demanding $(n_Ba_{BF}^3)^{1/3}\ll 1$. We can also argue for this by a dimensional analysis. From the form of $g_{BF}$ it is clear, that we should argue from $a_{BF}$. This has dimension of length, and so we need a parameter of dimension per length to get a unitless expression. The only reasonable parameter is the one that describes how many bosons, the fermions can actually interact with, leading to a length scale $n_B^{-1/3}$. Since a higher density of bosons leads to more fermion-boson interactions, we should therefore assume $(n_Ba_{BF}^3)^{1/3} \ll 1$. In this weak coupling limit we can safely take interactions only up to second order in $g_{BF}$. This means, that the four essential diagrams for the fermion-fermion induced interaction are the ones showed in figure \ref{fig.feynmandiagrams}. 

\begin{figure}
\begin{tikzpicture}[scale=0.25]
  \begin{feynman}[small]
    \vertex (number1) {\( (1) \)};
    \vertex [above left=of number1] (fermion1) {\( \tilde{p}_1 \)};
    \vertex [above right=of fermion1] (a);
    \vertex [below right=of a] (fermion2) {\(\tilde{p}_1 + \tilde{q}\)}; 
    \vertex [above=of a] (b);
    \vertex [left=of b] (boson1) {\( \sqrt{n_B} \)}; 
    \vertex [above= of b] (c);
    \vertex [right= of c] (boson2) {\( \sqrt{n_B} \)};
    \vertex [above= of c] (d);
    \vertex [above left=of d] (f3) {\(\tilde{p}_2\)};
    \vertex [above right=of d] (f4) {\(\tilde{p}_2 - \tilde{q}\)};
 
    \diagram* {
      (number1) -- [opacity=0.0] (fermion1) -- [fermion] (a) -- [fermion] (fermion2),
      (a) -- [photon, edge label'=\(g_{BF}\)] (b),
      (b) -- [dashed] (boson1),
      (b) -- [blue, fermion, edge label' = {\(-\tilde{q}, \mathbf{k}_\perp \)}] (c),
      (c) -- [dashed] (boson2),
      (c) -- [photon, edge label'=\(g_{BF}\)] (d),
      (d) -- [anti fermion] (f3),
      (d) -- [fermion] (f4)
    };
  \end{feynman}
\end{tikzpicture}
\begin{tikzpicture}
  \begin{feynman}[small]
    \vertex (number2) {\( (2) \)};
    \vertex [above left=of number2] (fermion1) {\( \tilde{p}_1 \)};
    \vertex [above right=of fermion1] (a);
    \vertex [below right=of a] (fermion2) {\(\tilde{p}_1 + \tilde{q}\)}; 
    \vertex [above=of a] (b);
    \vertex [left=of b] (boson1) {\( \sqrt{n_B} \)}; 
    \vertex [above= of b] (c);
    \vertex [right= of c] (boson2) {\( \sqrt{n_B} \)};
    \vertex [above= of c] (d);
    \vertex [above left=of d] (f3) {\(\tilde{p}_2\)};
    \vertex [above right=of d] (f4) {\(\tilde{p}_2 - \tilde{q}\)};
 
    \diagram* {
      (number2) -- [opacity=0.0] (fermion1) -- [fermion] (a) -- [fermion] (fermion2),
      (a) -- [photon, edge label'=\(g_{BF}\)] (b),
      (b) -- [dashed] (boson1),
      (b) -- [blue, anti fermion, edge label' = {\(\tilde{q}, \mathbf{k}_\perp \)}] (c),
      (c) -- [dashed] (boson2),
      (c) -- [photon, edge label'=\(g_{BF}\)] (d),
      (d) -- [anti fermion] (f3),
      (d) -- [fermion] (f4)
    };
  \end{feynman}
\end{tikzpicture}
\begin{tikzpicture}
  \begin{feynman}[small]
    \vertex (number3) {\( (3) \)};
    \vertex [above left=of number3] (fermion1) {\( \tilde{p}_1 \)};
    \vertex [above right=of fermion1] (a);
    \vertex [below right=of a] (fermion2) {\(\tilde{p}_1+\tilde{q}\)}; 
    \vertex [above=of a] (b);
    \vertex [left=of b] (boson1) {\( \sqrt{n_B} \)}; 
    \vertex [above= of b] (c);
    \vertex [right= of c] (boson2) {\( \sqrt{n_B} \)};
    \vertex [above= of c] (d);
    \vertex [above left=of d] (f3) {\(\tilde{p}_2\)};
    \vertex [above right=of d] (f4) {\(\tilde{p}_2-\tilde{q}\)};
 
    \diagram* {
      (number3) -- [opacity=0.0] (fermion1) -- [fermion] (a) -- [fermion] (fermion2),
      (a) -- [photon, edge label'=\(g_{BF}\)] (b),
      (b) -- [dashed] (boson1),
      (b) -- [blue, majorana, edge label' = {\(\tilde{q}, \mathbf{k}_\perp \)}] (c),
      (c) -- [dashed] (boson2),
      (c) -- [photon, edge label'=\(g_{BF}\)] (d),
      (d) -- [anti fermion] (f3),
      (d) -- [fermion] (f4)
    };
  \end{feynman}
\end{tikzpicture}
\begin{tikzpicture}
  \begin{feynman}[small]
    \vertex (number4) {\( (4) \)};
    \vertex [above left=of number4] (fermion1) {\( \tilde{p}_1 \)};
    \vertex [above right=of fermion1] (a);
    \vertex [below right=of a] (fermion2) {\(\tilde{p}_1+\tilde{q}\)}; 
    \vertex [above=of a] (b);
    \vertex [left=of b] (boson1) {\( \sqrt{n_B} \)}; 
    \vertex [above= of b] (c);
    \vertex [right= of c] (boson2) {\( \sqrt{n_B} \)};
    \vertex [above= of c] (d);
    \vertex [above left=of d] (f3) {\(\tilde{p}_2\)};
    \vertex [above right=of d] (f4) {\(\tilde{p}_2-\tilde{q}\)};
 
    \diagram* {
      (number3) -- [opacity=0.0] (fermion1) -- [fermion] (a) -- [fermion] (fermion2),
      (a) -- [photon, edge label'=\(g_{BF}\)] (b),
      (b) -- [dashed] (boson1),
      (b) -- [blue, anti majorana, edge label' = {\(\tilde{q}, \mathbf{k}_\perp \)}] (c),
      (c) -- [dashed] (boson2),
      (c) -- [photon, edge label'=\(g_{BF}\)] (d),
      (d) -- [anti fermion] (f3),
      (d) -- [fermion] (f4)
    };
  \end{feynman}
\end{tikzpicture}
\caption{Feynman diagrams for the induced interaction. Since the interaction is weak, we can neglect all other Feynman diagrams than (1)-(4). Diagrams (1) and (2) stems from the normal Green's function $G_{11}$. Diagrams (3) and (4) stems from the anormalous Green's functions $G_{12}$ and $G_{21}$. The diagrams have the exact same form for interactions between fermions in the same and opposite wires. The only difference is, that $g_{BF}$ carries an extra phase factor for interactions in wire 2. } 
\label{fig.feynmandiagrams}
\end{figure}

Notice that the bosons of the BEC, shown with dashed lines, carry a factor of $\sqrt{n_B}$. This is simply because it is the pour condensate wave function. Further $\tilde{p}_j = (p_j, i\omega_{m_j})$, with $\omega_{m} = (2m + 1)\pi kT$ a fermionic Matsubara frequency, and $\tilde{q} = (q, i\omega_q )$, with $\omega_q = \omega_{m_1} - \omega_{m_2}$ a bosonic Matsubara frequency. The blue lines in the diagrams are the normal and anormalous Bogoliubov phonon Green's functions derived in subsection \ref{sec.BECGreens}. Functionally they are:
\begin{equation}
G_{11}(\mathbf{k},i\omega_m) = \frac{u_{B,k}^2}{i\omega_m-E_{B,k}}-\frac{v_{B,k}^2}{i\omega_m+E_{B,k}}, \hspace{0.5cm} G_{12}(\mathbf{k},i\omega_m) = G_{21}(\mathbf{k},i\omega_m) = \frac{u_{B,k}v_{B,k}}{i\omega_m+E_{B,k}}-\frac{u_{B,k}v_{B,k}}{i\omega_m-E_{B,k}}.\nonumber
\end{equation}
Here $u_{B,k}$ and $v_{B,k}$ are the BEC coherence factors found in subsection \ref{sec.BEC} with $m_B$ the mass of the bosons. Further $E^2_{B,k} = \frac{k^2}{2m_B}\left(\frac{k^2}{2m_B} + 2g_Bn_B \right)$ is the BEC Bogoliubov spectrum and $g_B = \frac{4\pi a_B}{m_B}$, with $a_B$ the scattering length in the BEC. The diagrams can intuitively be understood as follows. A fermion in one wire interact with a (real) boson in the condensate. This creates a ripple in the condensate described by one of the Green's functions. This ripple reaches a second fermion in one of the two wires, where the momentum of the ripple is transferred to.  

Let us first focus on the induced interaction of two fermions in wire 1. Since the perpendicular momentum $\mathbf{k}_\perp$ is completely free we have to integrate over this. Since the momentum of the incoming and outgoing boson in each diagram is 0 we simply get a factor of $g_{BF}\; \text{e}^{-\frac{l_t^2}{4}k_\perp^2}$ \textit{twice}. Collecting all factors gives us the four contributions from the four diagrams (1)-(4) to the fermion-fermion induced interaction $V_{\text{ind}}^{11}$: 
\begin{align}
V^{11}_{\text{ind}, 1}(q,i\omega_q) &= n_Bg_{BF}^2\int\frac{d^2k_\perp}{(2\pi)^2}G_{11}(-q,\mathbf{k}_\perp,-\omega_q)\text{e}^{-\frac{l_t^2}{2}k_\perp^2}, \nonumber \\
V^{11}_{\text{ind}, 2}(q,i\omega_q) &= n_Bg_{BF}^2\int\frac{d^2k_\perp}{(2\pi)^2}G_{11}(q,\mathbf{k}_\perp,\omega_q)\text{e}^{-\frac{l_t^2}{2}k_\perp^2}, \nonumber \\
V^{11}_{\text{ind}, 3}(q,i\omega_q) &= n_Bg_{BF}^2\int\frac{d^2k_\perp}{(2\pi)^2}G_{12}(q,\mathbf{k}_\perp,\omega_q)\text{e}^{-\frac{l_t^2}{2}k_\perp^2}, \nonumber \\
V^{11}_{\text{ind}, 4}(q,i\omega_q) &= n_Bg_{BF}^2\int\frac{d^2k_\perp}{(2\pi)^2}G_{12}(q,\mathbf{k}_\perp,\omega_q)\text{e}^{-\frac{l_t^2}{2}k_\perp^2}. 
\end{align}
We notice, that the last two give the same contribution in this weak interacting limit.\footnote{This would \textit{not} be the case, when including higher order terms.} The 11 superscript means, that it is for two fermions in wire 1. Summing up the four contributions gives us the frequency dependent induced interaction $V^{11}_{\text{ind}}$ for fermions in wire 1:
\begin{equation}
V^{11}_{\text{ind}}(q,i\omega_q) = g_{BF}^2\int\frac{d^2k_\perp}{(2\pi)^2}\; \chi_\text{BEC}(q,\mathbf{k}_\perp,i\omega_q)\text{e}^{-\frac{l_t^2}{2}k_\perp^2}, 
\label{eq.V11indXBEC}
\end{equation}
with $\chi_\text{BEC}(\mathbf{k},i\omega_q) = \frac{k^2}{m_B}\frac{n_B}{(i\omega_q)^2 - E_{B,k}^2}$ the socalled density-density correlation function of the BEC. It now becomes clear, why we had to retain a nonzero value of $l_t$ at this stage. For $l_t\to 0$ the above has an integrand of the form $k^2/(ak^4 + bk^2 + c)$, with $a,b,c$ positive constants. As a result the integral is logarithmically divergent. The situation in a 2D-3D system is different. Here there is a single integral in the above, and the result converges for $l_t\to 0$. We return to this later on. For two fermions in wire 2, there is two additional phase factors. The incoming boson is associated with $\text{e}^{i\mathbf{k}_\perp\cdot \mathbf{d}}$, the outgoing with $\text{e}^{-i\mathbf{k}_\perp\cdot \mathbf{d}}$. These phase factors cancel and $V^{22}_{\text{ind}}(q,i\omega_q) = V^{11}_{\text{ind}}(q,i\omega_q)$ as one would expect. We denote these \textit{intra}wire interactions. Additionally, there is an \textit{inter}wire induced interaction between the wires, which we will denote $V_{\text{ind}}^{12}(q,i\omega_q)$. The preceding section shows, that the calculation of this induced interaction is analogous to the above, but with the additional factor of $\text{e}^{i\mathbf{k}_\perp\cdot \mathbf{d}}$ from scattering in the second wire. Hence:
\begin{equation}
V_{\text{ind}}^{12}(q,i\omega_q) = g_{BF}^2\int\frac{d^2k_\perp}{(2\pi)^2}\; \chi_\text{BEC}(q,\mathbf{k}_\perp, i\omega_q)\text{e}^{-\frac{l_t^2}{2}k_\perp^2}\text{e}^{i\mathbf{k}_\perp\cdot \mathbf{d}}. 
\label{eq.V12indXBEC} 
\end{equation}
We notice, that the \textit{inter}wire interaction goes to the \textit{intra}wire interaction for $d \to 0$. The presence of the condensate density-density correlation expresses, that it is density fluctuations in the condensate that mediate the induced interactions. It turns out, that we can find quite simple expressions for these induced interactions in real space in the zero frequency limit: $\omega_q = 0$. Since we in general are going to restrict ourselves to this limit, we briefly discuss what it physically means. 

\subsection{Retardation effects} \label{sec.RetardationEffects}
From equations \eqref{eq.V11indXBEC} and \eqref{eq.V12indXBEC} it is evident that the induced interactions have a frequency dependency. This dependency reflects, that the fermions do not interact instanteneously, socalled retardation effects. In turn this embodies, that the phonons in the condensate, the mediators of the fermion-fermion interaction, moves at a finite speed $c_0 = \sqrt{\frac{n_Bg_B}{m_B}} = \frac{\sqrt{4\pi n_B a_B}}{m_B}$.\footnote{It is analogous to the retarded fields in electrodynamics. There it reflects the finiteness of the speed of light.} To neglect these effects we therefore need to assume, that the typical speed of the fermions is much smaller than the speed of the bosons: $v_F \ll c_0$, $v_F = k_F/m_F$ the Fermi speed for free fermions. This leads to the relation:
\begin{equation}
1 \gg \frac{v_F}{c_0} = \frac{\sqrt{\pi}}{2} \frac{m_B}{m_F}\frac{1}{ \sqrt{ (n_Ba_B^3)^{1/3} } }\frac{n_F}{ n_B^{1/3} } = \frac{m_B}{m_F}\frac{k_F\xi}{\sqrt{2}}, 
\label{eq.RetardationEffectsneglectionassumption}
\end{equation}
whereby we have expressed the ratio of velocities in terms of unitless quantities and in terms of the coherence length, $\xi$, defined by $\frac{1}{\xi^2} = 2m_Bn_Bg_B$. With this assumption at hand we can focus on the zero frequency induced interaction $V^{ij}_{\text{ind}}(q,0)$. For fermions and bosons of similar mass we notice, that the neglect of retardation effects is equivalent to only studying short range interactions: $k_F\xi \lesssim 1$. 

\subsection{Real space}
\label{subsec.inducedinteraction.realspace}
In this subsection we calculate the real space interactions, $\tilde{V}^{ij}_{\text{ind}}(x, 0)$. 

We start with the \textit{inter}wire interaction. We get:
\begin{equation}
\tilde{V}^{12}_{\text{ind}}(x, 0) = \int \frac{dq}{2\pi} \; \text{e}^{iqx} V^{12}_{\text{ind}}(q, 0) = g^2_{BF}\int \frac{d^3 k}{(2\pi)^3}\;\chi_\text{BEC}(\mathbf{k}, 0)\text{e}^{-\frac{l_t^2}{2}k_\perp^2}\text{e}^{i\mathbf{k}\cdot \mathbf{r}}, \nonumber
\end{equation}
where we write $\mathbf{k} = (q, \mathbf{k}_{\perp})$ and $\mathbf{r} = (x, \mathbf{r}_{\perp})$. The above expression converges for $l_t \to 0$, because the integrand is monotonically increasing for decreasing $l_t$. The gaussian, $\text{e}^{-\frac{l_t^2}{2}k_\perp^2}$, hereby drops out and we are left with the Fourier transformation of $\chi_\text{BEC}(\mathbf{k}, 0) = -4m_Bn_B\frac{1}{k^2 + 1/\xi^2}$. We recognise this as the Yukawa potential in momentum space. For completeness we here show how to perform the Fourier transformation. The integral of interest is:
\begin{equation}
I(\mathbf{r}) = \int \frac{d^3 k}{(2\pi)^3} \frac{1}{k^2 + 1/\xi^2}\text{e}^{i\mathbf{k}\cdot\mathbf{r}} = \int_{0}^{\pi}d\theta \sin(\theta)\int_{0}^{\infty}\frac{dk}{(2\pi)^2} \frac{k^2}{k^2 + 1/\xi^2}\text{e}^{ikr\cos(\theta)},
\label{eq.def.Fourierintegral}
\end{equation}
where we in the second expression set the angle between $\mathbf{k}$ and $\mathbf{r}$ to be $\theta$ and transform the integral to polar coordinates. We have further used, that the integrand does not depend on the azimuthal angle, $\phi$, so that this integral simple yields a factor of $2\pi$. The $\theta$ dependent part then simply gives:
\begin{equation}
\int_{0}^{\pi}d\theta \sin(\theta)\text{e}^{ikr\cos(\theta)} = -\left.\frac{1}{ikr}\text{e}^{ikr\cos(\theta)}\right|_{0}^{\pi} = \frac{1}{ikr}\left(\text{e}^{ikr} - \text{e}^{-ikr}\right). \nonumber
\end{equation}
Inserting this into equation \eqref{eq.def.Fourierintegral} yields: $I(\mathbf{r}) = \frac{1}{2\pi r}\int \frac{dk}{2\pi i} \frac{k}{k^2 + 1/\xi^2}\text{e}^{ikr}$, where the integration limits are now implicitly $\pm \infty$. This integral is solvable using Cauchy's residue theorem. Explicitly, we now think of $k$ as a complex variable. We define the half-circle contour $\mathcal{C}$ in the upper half-plane with a radius $R$ tending to infinity. Since $r > 0$ the integrand goes exponentially to zero at the circle boundary, so that this part of the contour integral does not contribute. We have to find the poles of the integrand in the upper complex plane of $k$ and calculate the residues. There is a single pole given by $k = \frac{i\sqrt{2}}{\xi}$. Cauchy's residue theorem then states, that:
\begin{equation}
I(\mathbf{r}) = \frac{1}{2\pi r}\int \frac{dk}{2\pi i} \frac{k}{k^2 + 1/\xi^2}\text{e}^{ikr} = \frac{1}{2\pi r}\text{Res}\left(\frac{k}{k^2 + 1/\xi^2}\text{e}^{ikr}, \frac{i\sqrt{2}}{\xi}\right) = \frac{1}{4\pi r} \text{e}^{-\sqrt{2}r/\xi}, \nonumber
\end{equation}
where $\text{Res}(f(x), x_0)$ denotes the residue of $f(x)$ in $x = x_0$. Setting $\mathbf{r} = (x, \mathbf{d})$ we get for the induced interaction between the wires:
\begin{equation}
\tilde{V}^{12}_{\text{ind}}(x, 0) = -4m_Bg^2_{BF}n_B I(x, \mathbf{d}) = -\frac{m_Bg_{BF}^2n_B}{\pi}\frac{\text{e}^{ -\sqrt{2}\sqrt{x^2 + d^2}/\xi }}{\sqrt{x^2 + d^2}}.
\label{eq.V12indx}
\end{equation}
As noted in section \ref{sec.1D3Dinducedinteraction} we can obtain the intrawire interaction by simply setting $d = 0$. We get:
\begin{equation}
\tilde{V}^{11}_{\text{ind}}(x, 0) = -\frac{m_Bg_{BF}^2n_B}{\pi}\frac{\text{e}^{ -\sqrt{2}|x|/\xi }}{|x|}.
\label{eq.V11indx}
\end{equation}
The induced interaction in real space is seen to be the Yukawa interaction with a range of interaction given by the BEC coherence length, $\xi$. We can also understand this physically. The coherence length is in general the length scale over which, the condensate adjusts to an influence on the boson density. Hence, it is physically intuitive that this is exactly the range of the interaction. 

We could have calculated the intrawire induced interaction for a general $l_t > 0$ and then taken the $l_t \to 0$ limit. The reader is referred to appendix \ref{Appendix.intrawireinteraction.ltnonzero} to see this explicitly. In the end it gives the same result. 

We get a unitless form of the real space interaction by dividing with the typical energy of the fermions, the Fermi energy $\epsilon_{F,0}$. In this way the resulting front factor of $\tilde{V}^{ij}_{\text{ind}} / \epsilon_{F,0}$ is a measure of the strength of interaction given by:
\begin{equation}
G = - 8\left( \frac{m_F}{m_B} + \frac{m_B}{m_F} + 2 \right) \frac{n_B^{1/3}}{n_F}(n_Ba_{BF}^3)^{2/3}.
\label{eq.interactionstrength.wires}
\end{equation}
The mass ratio $m_B / m_F$, The relative interparticle distance $n_B^{1/3} / n_F$ and the Bose-Fermi gas parameter $(n_Ba_{BF}^3)^{1/3}$ hereby comes naturally about by going to unitless quantities. 

In figure \ref{fig.V12indx} we show the interwire interaction for several interwire distances, $d$. The black curve is the $d = 0$ limit where the inter- and intrawire interactions coincide. 

\begin{figure} 
\begin{center}  
% GNUPLOT: LaTeX picture with Postscript
\begingroup
  \makeatletter
  \providecommand\color[2][]{%
    \GenericError{(gnuplot) \space\space\space\@spaces}{%
      Package color not loaded in conjunction with
      terminal option `colourtext'%
    }{See the gnuplot documentation for explanation.%
    }{Either use 'blacktext' in gnuplot or load the package
      color.sty in LaTeX.}%
    \renewcommand\color[2][]{}%
  }%
  \providecommand\includegraphics[2][]{%
    \GenericError{(gnuplot) \space\space\space\@spaces}{%
      Package graphicx or graphics not loaded%
    }{See the gnuplot documentation for explanation.%
    }{The gnuplot epslatex terminal needs graphicx.sty or graphics.sty.}%
    \renewcommand\includegraphics[2][]{}%
  }%
  \providecommand\rotatebox[2]{#2}%
  \@ifundefined{ifGPcolor}{%
    \newif\ifGPcolor
    \GPcolortrue
  }{}%
  \@ifundefined{ifGPblacktext}{%
    \newif\ifGPblacktext
    \GPblacktexttrue
  }{}%
  % define a \g@addto@macro without @ in the name:
  \let\gplgaddtomacro\g@addto@macro
  % define empty templates for all commands taking text:
  \gdef\gplbacktext{}%
  \gdef\gplfronttext{}%
  \makeatother
  \ifGPblacktext
    % no textcolor at all
    \def\colorrgb#1{}%
    \def\colorgray#1{}%
  \else
    % gray or color?
    \ifGPcolor
      \def\colorrgb#1{\color[rgb]{#1}}%
      \def\colorgray#1{\color[gray]{#1}}%
      \expandafter\def\csname LTw\endcsname{\color{white}}%
      \expandafter\def\csname LTb\endcsname{\color{black}}%
      \expandafter\def\csname LTa\endcsname{\color{black}}%
      \expandafter\def\csname LT0\endcsname{\color[rgb]{1,0,0}}%
      \expandafter\def\csname LT1\endcsname{\color[rgb]{0,1,0}}%
      \expandafter\def\csname LT2\endcsname{\color[rgb]{0,0,1}}%
      \expandafter\def\csname LT3\endcsname{\color[rgb]{1,0,1}}%
      \expandafter\def\csname LT4\endcsname{\color[rgb]{0,1,1}}%
      \expandafter\def\csname LT5\endcsname{\color[rgb]{1,1,0}}%
      \expandafter\def\csname LT6\endcsname{\color[rgb]{0,0,0}}%
      \expandafter\def\csname LT7\endcsname{\color[rgb]{1,0.3,0}}%
      \expandafter\def\csname LT8\endcsname{\color[rgb]{0.5,0.5,0.5}}%
    \else
      % gray
      \def\colorrgb#1{\color{black}}%
      \def\colorgray#1{\color[gray]{#1}}%
      \expandafter\def\csname LTw\endcsname{\color{white}}%
      \expandafter\def\csname LTb\endcsname{\color{black}}%
      \expandafter\def\csname LTa\endcsname{\color{black}}%
      \expandafter\def\csname LT0\endcsname{\color{black}}%
      \expandafter\def\csname LT1\endcsname{\color{black}}%
      \expandafter\def\csname LT2\endcsname{\color{black}}%
      \expandafter\def\csname LT3\endcsname{\color{black}}%
      \expandafter\def\csname LT4\endcsname{\color{black}}%
      \expandafter\def\csname LT5\endcsname{\color{black}}%
      \expandafter\def\csname LT6\endcsname{\color{black}}%
      \expandafter\def\csname LT7\endcsname{\color{black}}%
      \expandafter\def\csname LT8\endcsname{\color{black}}%
    \fi
  \fi
    \setlength{\unitlength}{0.0500bp}%
    \ifx\gptboxheight\undefined%
      \newlength{\gptboxheight}%
      \newlength{\gptboxwidth}%
      \newsavebox{\gptboxtext}%
    \fi%
    \setlength{\fboxrule}{0.5pt}%
    \setlength{\fboxsep}{1pt}%
\begin{picture}(7200.00,5040.00)%
    \gplgaddtomacro\gplbacktext{%
      \csname LTb\endcsname%
      \put(682,1188){\makebox(0,0)[r]{\strut{}$-4$}}%
      \csname LTb\endcsname%
      \put(682,2030){\makebox(0,0)[r]{\strut{}$-2$}}%
      \csname LTb\endcsname%
      \put(682,2872){\makebox(0,0)[r]{\strut{}$0$}}%
      \csname LTb\endcsname%
      \put(682,3713){\makebox(0,0)[r]{\strut{}$2$}}%
      \csname LTb\endcsname%
      \put(682,4555){\makebox(0,0)[r]{\strut{}$4$}}%
      \csname LTb\endcsname%
      \put(877,484){\makebox(0,0){\strut{}$-10$}}%
      \csname LTb\endcsname%
      \put(2343,484){\makebox(0,0){\strut{}$-5$}}%
      \csname LTb\endcsname%
      \put(3809,484){\makebox(0,0){\strut{}$0$}}%
      \csname LTb\endcsname%
      \put(5274,484){\makebox(0,0){\strut{}$5$}}%
      \csname LTb\endcsname%
      \put(6740,484){\makebox(0,0){\strut{}$10$}}%
    }%
    \gplgaddtomacro\gplfronttext{%
      \csname LTb\endcsname%
      \put(176,2871){\rotatebox{-270}{\makebox(0,0){\strut{}$2m_F/k_F W_{	ext{ind}}(k, k_F)$}}}%
      \put(3808,154){\makebox(0,0){\strut{}$k / k_F$}}%
      \csname LTb\endcsname%
      \put(4565,4803){\makebox(0,0)[l]{\strut{}$k_Fd = 0.720$}}%
      \csname LTb\endcsname%
      \put(4565,4583){\makebox(0,0)[l]{\strut{}$k_Fd = 0.735$}}%
      \csname LTb\endcsname%
      \put(4565,4363){\makebox(0,0)[l]{\strut{}$k_Fd = 0.750$}}%
      \csname LTb\endcsname%
      \put(4565,4143){\makebox(0,0)[l]{\strut{}$k_Fd = 0.765$}}%
      \csname LTb\endcsname%
      \put(4565,3923){\makebox(0,0)[l]{\strut{}$k_Fd = 0.775$}}%
    }%
    \gplbacktext
    \put(0,0){\includegraphics{InducedInteraction}}%
    \gplfronttext
  \end{picture}%
\endgroup
  
\caption{Coloured lines: The \textit{inter}wire induced interaction, $\tilde{V}^{12}_{\text{ind}}(x, 0)$, plotted as a function of $x$ for several values of $d$. Black line: the \textit{intra}wire induced interaction, $\tilde{V}^{11}_{\text{ind}}(x, 0)$. Parameters: $(n_Ba_B^3)^{1/3} = 0.01$, $(n_Ba_{BF}^3)^{1/3} = 0.1$, $l_t = 0$, $\frac{m_B}{m_F} = 7/40$, $\frac{n_F}{n_B^{1/3}} = 0.215$, $v_F/c_0 = 0.33$.}  
\label{fig.V12indx}  
\end{center}    
\end{figure}

In appendix \ref{Appendix.inducedinteraction.realspace} we calculate the real space induced interaction for general Matsubara frequencies $\omega_m = 2\pi m k_BT$ in the $l_t = 0$ limit. This general interaction is qualitatively the same as the zero frequency component.  

\subsection{Momentum space}
\label{subsec.inducedinteraction.momentumspace}
In this section we calculate the momentum space induced interactions.

We start with the \textit{inter}wire interaction. In chapter \ref{Chapter4} we will see, that the relevant expression is simply the Fourier transform of the real space interaction. Therefore, we could go back to equation \eqref{eq.V12indXBEC} for $\omega_q = 0$. However, it is simpler to use the real space expression found in subsection \ref{subsec.inducedinteraction.realspace} and perform the Fourier transform explicitly. We get:
\begin{equation}
V_{\text{ind}}^{12}(q,0) = \int dx \; \text{e}^{-iqx}\tilde{V}_{\text{ind}}^{12}(x,0) = 2\int_0^\infty \cos(qx)\tilde{V}_{\text{ind}}^{12}(x,0), \nonumber
\end{equation}
since the induced interaction is even in real space. Writing the variables in units of the range $\xi/\sqrt{2}$:
\begin{equation}
V_{\text{ind}}^{12}(\tilde{q},0) = -\frac{2n_Bg^2_{BF}m_B}{\pi}\int_0^\infty d\tilde{x} \cos(\tilde{q}\tilde{x})\frac{ \text{e}^{ -\sqrt{\tilde{x}^2+\tilde{d}^2} } }{\sqrt{\tilde{x}^2+\tilde{d}^2}} = -\frac{2n_Bg^2_{BF}m_B}{\pi}K_0\left[\tilde{d}\sqrt{\tilde{q}^2+1}\right], \nonumber
\end{equation}
where $K_0(x)$ is the modified Bessel function of the second kind of order zero. This calculation is done with the use of the Fourier cosine transform routine in Maple 16. Since $\tilde{q} = \frac{q\xi}{\sqrt{2}}, \tilde{d} = \frac{\sqrt{2}d}{\xi}$, we finally get:
 \begin{equation}
V_{\text{ind}}^{12}(q,0) = -\frac{2n_Bg^2_{BF}m_B}{\pi}K_0\left[\sqrt{(qd)^2+\frac{2d^2}{\xi^2}}\right]. 
\label{eq.V12indq.zerofrequency}
\end{equation}
We hereby have a closed form expression for the interwire induced interaction in momentum space. Next we turn to the \textit{intra}wire interaction. The situation here is more tricky. As already noted, the induced interaction in equation \eqref{eq.V11indXBEC} diverges for $l_t \to 0$. However, in chapter \ref{Chapter4} we will see, that the anticommutator relations for the fermionic creation and annihilation operators leads to the following expression for the scattering amplitude in momentum space:
\begin{equation}
W^{11}_{\text{ind}}(k, q, p) = \frac{1}{2}\left(V^{11}_{\text{ind}}\left( p, 0 \right) - V^{11}_\text{ind}\left( p + k - q, 0 \right) \right). 
\label{eq.Wkqp.scattering.amplitude}
\end{equation}
The scattering amplitude is hereby a combination of two Fourier transforms. We will comment further on the meaning of this expression in chapter \ref{Chapter4}. For now we simply wish to calculate the $l_t \to 0$ limit. Going back to equation \eqref{eq.V11indXBEC} for $\omega_q = 0$, it turns out, that we can express $V^{11}_\text{ind}\left( q, 0 \right)$ as:
\begin{equation}
V^{11}_{\text{ind}}(q, 0) = -\frac{m_Bg_{BF}^2n_B}{\pi} \text{e}^{F(q)} E_1(F(q)),
\label{eq.V11indq.zerofrequency.ltnonzero}
\end{equation}
where $F(q) = \frac{l_t^2}{2}\left(q^2 + \frac{2}{\xi^2} \right)$ and $E_1(x)$ is the exponential integral: $E_1(x) = \int_1^\infty du \frac{\text{e}^{-xu}}{u}$. Inserting this into the above expression for $W^{11}_{\text{ind}}$ yields:
\begin{align}
W^{11}_{\text{ind}}(k, q, p) &= \frac{1}{2}\left[V^{11}_\text{ind}(p, 0) - V^{11}_\text{ind}(p + k - q, 0)\right] \nonumber \\
&= -\frac{m_Bg_{BF}^2n_B}{2\pi}\left[ \text{e}^{F(p)} E_1(F(p)) - \text{e}^{F(p + k - q)} E_1(F(p + k - q)) \right], \nonumber
\end{align}
Now we take the $l_t \to 0$ limit. We have $\partial_x E_1(x) = -\int_1^{\infty}du\; \text{e}^{-xu} = -\frac{1}{x}\text{e}^{-x}$. For $0 < x \ll 1$, this gives $\partial_xE_1(x) = -\frac{1}{x}$. Hence, in this regime we have $E_1(x) = C -\ln(x)$, where $C$ is a constant.\footnote{The constant is minus the Euler-Mascheroni constant $\gamma$. To ten digits precision $\gamma = 0.5772156649$. This is found in Maple 16.} The exponentials $\text{e}^{F(p)}$ and $\text{e}^{F(k + p - q)}$ just give $1$ in the $l_t \to 0$ limit, and so we are left with the expression:
\begin{equation}
\lim_{l_t \to 0} \; W^{11}_{\text{ind}}(k, q, p) = -\frac{m_Bg_{BF}^2n_B}{2\pi} \ln\left[\frac{(k - q + p)^2 + 2/\xi^2}{p^2 + 2/\xi^2}\right].
\label{eq.Wkqp.scattering.amplitude.lt=0} 
\end{equation}
Hereby we also have a closed form expression for the intrawire interaction in momentum space. The fact that we have these closed form expressions is crucial for the feasibility of the later numerical analyses.




