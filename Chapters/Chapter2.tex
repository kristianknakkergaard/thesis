% Chapter 1

\chapter{Prerequisites} % Main chapter title

\label{Chapter2} % For referencing the chapter elsewhere, use \ref{Chapter1} 

\lhead{Chapter 2. \emph{Prerequisites}} % This is for the header on each page - perhaps a shortened title

%----------------------------------------------------------------------------------------
In this chapter we come with a summary of the necessary prerequisites for the understanding of the 1D-3D system at hand. This includes the free fermion gas, the Kitaev model and the weakly interacting Bose-Einstein condensate. Further the notation of the thesis is in large introduced here, and I refer to the relevant section in this chapter, if any confusion should arise. If one is familiar with the concepts introduced here, they can be skipped. 

\section{The free fermion gas in one dimension} \label{sec.chemicalpotential.freegas}
We consider a free gas of $N_F$ \textit{identical} fermions on a string of length $\mathcal{L}$ with an associated chemical potential $\mu_0$. The statistical mechanics are governed by the Fermi-Dirac distribution:
\begin{equation}
f(k) = \frac{1}{\text{e}^{\beta(E_{F0,k}-\mu_0)} + 1},
\end{equation}
with $\beta = 1/k_BT$ and $E_{F0,k} = \frac{k^2}{2m_F}$ is the energy of the free fermions, with $k$ the momentum, and $m_F$ the mass. As usual we let the Fermi energy be the chemical potential at zero temperature: $\epsilon_{F,0} = \mu_0(T=0)$.\footnote{The 0 index is to indicate, that the parameters are for the \textit{free} gas.} We further impose periodic boundary conditions on the string. Since the wave functions are proportional to $\text{e}^{ikx}$ for the free gas, this gives us the condition $\text{e}^{ik\mathcal{L}} = \text{e}^{ik\cdot 0} = 1$, and so $k = n\frac{2\pi}{\mathcal{L}}$ for integers $n$. Hence, we have a specific quantization of the $k$'s. We define the Fermi momentum $k_F$ through: $\epsilon_{F,0} = \frac{k_F^2}{2m_F}$. Because of the Pauli exclusion principle and from the fact, that the fermions are identical, two fermions cannot be in the same energetic state. This means, that the number of fermions $N_F$ can be expressed as:
\begin{equation}
N_F = \sum_{|k|< k_F} = \frac{\mathcal{L}}{2\pi} \int_{-k_F}^{k_F} dk = \frac{\mathcal{L}}{\pi} k_F \Rightarrow k_F = \pi n_F, 
\label{eq.relationkfnf}
\end{equation}
where $n_F = \frac{N_F}{\mathcal{L}}$ is the density of the fermions. I do not put a 0 subscript on $n_F$, since I consider the length and the number of fermions as fixed. Hence, $n_F$ is an externally fixed variable. The chemical potential for temperatures $T>0$ is determined from the number equation: $N_F = \sum_k f(k)$, since $f(k)$ is the mean occupancy of the $k$'th state. Transforming this into an energy integral yields:
\begin{equation}
N_F = \int_0^\infty dE \; \frac{\mathcal{L}}{\pi}\sqrt{\frac{m}{2E}} f(E). 
\label{eq.numberequationfreegas}
\end{equation}
From here we can also see, that the density of states is:
\begin{equation}
D_0(E) = \left\{\begin{matrix}
 \frac{\mathcal{L}}{\pi}\sqrt{\frac{m}{2E}}, & E > 0,  \\ 
 0, & \text{otherwise}. 
\end{matrix}\right. 
\label{eq.densityofstatesfreegas}
\end{equation}
The above number equation gives the chemical potential for all temperatures. It is worthwhile to study the behaviour for low temperatures analytically though. This is quite simple using the Sommerfeld expansion. We let the Fermi temperature be defined by $\epsilon_{F,0} = k_B T_{F,0}$. The Sommerfeld expansion for the chemical potential leads to the differential equation for $T/T_{F,0} \ll 1$ \cite{GiuseppeGiuseppe}:
\begin{equation}
\frac{d\mu_0}{dT} = -\frac{\pi^2}{3}k_B^2 T \frac{D_0'(\mu_0)}{D_0(\mu_0)}. \nonumber
\end{equation}
From above the density of states is proportional to $E^{-1/2}$, and so the equation for $\mu_0$ can be solved by separation of variables. The result for $T/T_{F,0}\ll 1$ is:
\begin{equation}
\frac{\mu_0(T)}{\epsilon_{F,0}} = 1 + \frac{\pi^2}{12}\left(\frac{T}{T_{F,0}}\right)^2.
\end{equation}
Hence, we see that the chemical potential of the free gas is \textit{increasing} quadratically for $T/T_{F,0} \ll 1$. This is in contrast to the case in both two and three dimension, where the chemical potential \textit{decreases} monotonically. For $T/T_{F,0} \gg 1$ the chemical potential must asymptotically go to the chemical potential for the classical gas. This means, that it must go to minus infinity for $T\to \infty$.\cite{SchroederThermal} This also means, that the chemical potential has a maximal value for some temperature. A numerical calculation for equation \eqref{eq.numberequationfreegas} shows, that this maximum is around the Fermi temperature $T_{F,0}$. 

\section{Units}
In this section I will specify what units energies, momenta, distances, densities etc. will be in. This is especially important for the numerical analyses we will perform later on. 

Any physical process of the fermions considered is expected only to affect the fermions near the Fermi energy $\epsilon_{F,0}$. This is essentially a consequence of the assumption, that the other relevant energies we consider are small with respect to the Fermi energy. For this reason we express energies in terms of the Fermi energy for the free gas $\epsilon_{F,0}$. Similarly we express temperatures in units of $T_{F,0}$, momenta in units of $k_F$ and distances in units of $1/k_F$. The density of the three dimensional Bose-Einstein condensate will be in units of $n_F^3$ and all masses in units of $m_F$. In some circumstances variables will be have a "$\tilde{}$" to express that it is unitless. For example: $\tilde{k} = k/k_F$.  

\section{The Kitaev Model}
The goal of this thesis is fundamentally to show, that the 1D-3D mixture of Fermi and Bose gases realises the socalled Kitaev model, so let us briefly review the fundamental properties of this model. The system consists of identical fermions on a string of length $\mathcal{L}$. The Hamiltonian in momentum space is: 
\begin{equation}
H_{FF} = \frac{1}{2}\sum_{k} F_k^\dagger \mathcal{H}_{FF,k} F_k, \hspace{0.5cm} \mathcal{H}_{FF,k} = \begin{bmatrix} \varepsilon_k & \Delta_k \\ \Delta^*_k & -\varepsilon_k \end{bmatrix}, \hspace{0.5cm} F^\dagger_k = \begin{bmatrix} f_k^\dagger & f_{-k} \end{bmatrix}. 
\label{eq.HKitaevpre}
\end{equation}

Here $f_k$ ($f_k^\dagger$) creates (annihilates) a fermion ($F$) on the string with momentum $k$.\footnote{The subscript $F$ is to distinguish the parameters here from the analogous ones for the boson gas, which will be introduced in the following section. The double subscript $F$ on $H$ and $\mathcal{H}$ is to specify, that this is a Hamiltonian describing a fermion-fermion interaction.} To investigate the bulk properties of the string, we will use periodic boundary conditions, and so the momenta are quantized in the same way as for the free gas: $k = n\frac{2\pi}{\mathcal{L}}$. The Hamiltonian kernel $\mathcal{H}_{FF,k}$ contains two independent quantities: $\varepsilon_k$ and $\Delta_k$. $\varepsilon_k$ we will see is the kinetic energy of the fermions measured relative to the chemical potential: $k^2/2m-\mu$, and so it is real. $\Delta_k$ is the socalled $p$-wave pairing amplitude in $k$-space, and is generally complex. The Hamiltonian is diagonalized by diagonalizing the $2\times 2$ kernel $\mathcal{H}_{FF,k}$ for each $k$. This is done by a standard Bogoliubov transform: 
\begin{equation}
 \begin{bmatrix} f_k \\ f_{-k}^\dagger \end{bmatrix} = U_{F,k} \begin{bmatrix} \zeta_k \\ \zeta^\dagger_{-k} \end{bmatrix}, \hspace{0.5cm} U_{F,k} = \begin{bmatrix} u^*_{F,k} & -v_{F,k} \\ v^*_{F,k} & u_{F,k} \end{bmatrix}. 
 \label{eq.fermionquasiparticledef}
\end{equation}

The operators on the right are new fermionic annihilation and creation operators. This means, that they have to obey the standard anticommutation relations for fermionic operators: $\{\zeta_k,\zeta_{k'}^\dagger \} = \delta_{k,k'}$ with all other anticommutators 0. By checking these explicitly it is easy to see, that then $\det(U_{F,k}) = |u_{F,k}|^2+|v_{F,k}|^2 = 1$ must hold, leaving $U_{F,k}$ as a $SU(2)$ transformation of the original operators. The coefficients $u_{F,k}, v_{F,k}$ are determined from the demand that for each $k$, $U^\dagger_{F,k} \mathcal{H}_{FF,k}U_{F,k}$ must be diagonal. This eventually leads to two solutions for both $v_{F,k}$ and $u_{F,k}$. Finally using the minimal energy principle, we must choose the solution, that yields the minimal energy for the ground state: $\zeta_k\ket{g.s} = 0$. This leaves us with the following solution. The Hamiltonian is now diagonal in the $\zeta$-operators: 
\begin{equation}
H_{FF} = \frac{1}{2}\sum_k (\varepsilon_k-2E_{F,k}) + \sum_k E_{F,k} \zeta^\dagger_k \zeta_k, \hspace{0.5cm} E_{F,k} = \sqrt{\varepsilon_k^2 + |\Delta_k|^2}
\label{eq.Kitaev.H_diagonalpre}
\end{equation}

And the solutions for $u_{F,k}$ and $v_{F,k}$ are: 
\begin{equation}
|u_{F,k}|^2 = \frac{1}{2}\left(1 + \frac{\varepsilon_k}{E_{F,k}}\right), \hspace{0.5cm} |v_{F,k}|^2 = \frac{1}{2}\left(1-\frac{\varepsilon_k}{E_{F,k}}\right), \hspace{0.5cm} \frac{v_{F,k}\Delta^*_k}{u_{F,k}}=E_{F,k}-\varepsilon_k.
\label{eq.Kitaev.uk_vk}
\end{equation}
The latter expression is given, because it is an often used relation between $u_{F,k}$ and $v_{F,k}$. 

%%%%%%%%%%%%%%%%%%%%%%%%%%%%%%%%

\section{The Bose-Einstein condensate}
In this section we introduce a microscopic theory for the three dimensional Bose-Einstein condensate (BEC). It is largely based on chapter 8 in \cite{Pethick}. Further we calculate the condensate Green's functions. 

\subsection{Microscopic theory of the uniform BEC}
\label{sec.BEC}
The second quantized Hamiltonian describing the uniform BEC (no trapping potential is inferred) is given by the following expansion in real space: 
\begin{equation}
H_{BB} = \int d^3 r \left(\hat{\psi}_B^\dagger(\mathbf{r})\left[-\frac{\nabla^2}{2m_B}\right]\hat{\psi}_B(\mathbf{r}) + \frac{g_B}{2}\hat{\psi}_B^\dagger(\mathbf{r})\hat{\psi}_B^\dagger(\mathbf{r})\hat{\psi}_B(\mathbf{r})\hat{\psi}_B(\mathbf{r})  \right), 
\end{equation}

The second part stems from a pair interaction between the bosons (B) given by $g_B\delta(\mathbf{r})$.\footnote{$BB$ as subscript on $H$: boson-boson interaction.} $m_B$ is the mass of the bosons. The wave function satisfies the normalization $\int d^3 r |\psi(\mathbf{r})|^2 = N$, $N$ being the number of bosons in the gas. We expand this in free particle states: $\hat{\psi}(\mathbf{r}) = \frac{1}{\sqrt{\mathcal{V}}}\sum_\mathbf{k} \text{e}^{i\mathbf{k}\cdot\mathbf{r}}b_\mathbf{k}$, where $b_\mathbf{k}, b^\dagger_\mathbf{k}$ are the annihilation and creation operators for bosons in the cloud with momentum $\mathbf{k}$. $\mathcal{V}$ is the volume of the gas. Inserting into the Hamiltonian $H_{BB}$ we obtain: 
\begin{equation}
H_{BB} = \sum_\mathbf{k} \frac{k^2}{2m_B}b_\mathbf{k}^\dagger b_\mathbf{k} + \frac{g_B}{2\mathcal{V}}\sum_{\mathbf{k}_1,\mathbf{k}_2,\mathbf{q}} b^\dagger_{\mathbf{k}_1+\mathbf{q}}b^\dagger_{\mathbf{k}_2-\mathbf{q}}b_{\mathbf{k}_1}b_{\mathbf{k}_2}.  
\end{equation}

Now writing $\hat{\psi} = \sqrt{n_B} + \delta \psi(\mathbf{r})$ we have perturbatively separated the field operator in a perfect condensate term $\sqrt{n_B}$, $n_B$ being the uniform density of the condensed bosons, and a fluctuating excited term $\delta \hat{\psi}(\mathbf{r}) =  \frac{1}{\sqrt{\mathcal{V}}}\sum_{\mathbf{k}\neq \mathbf{0}} \text{e}^{i\mathbf{k}\cdot\mathbf{r}}b_\mathbf{k}$. Assuming a small number of excited bosons, we neglect terms with more than two $b$-operators with $\mathbf{k}$ nonzero. This along with other simple algebraic manipulations lead us to the form:\footnote{More details can be found in chapter 8 in \cite{Pethick}.} 

\begin{equation}
H_{BB} = \frac{N^2g_B}{2\mathcal{V}} + \sum_{\mathbf{k}\neq \mathbf{0}}\left[\left(\frac{k^2}{2m_B}+n_Bg_B\right)b_\mathbf{k}^\dagger b_\mathbf{k} + \frac{n_Bg_B}{2}\left( b_\mathbf{k}^\dagger b_{-\mathbf{k}}^\dagger + b_{\mathbf{k}} b_{-\mathbf{k}} \right) \right].
\label{eq.bosonHamiltonian}
\end{equation}
This Hamiltonian can be diagonalized by a canonical Bogoliubov transformation:

\begin{equation}
\begin{bmatrix} b_\mathbf{k} \\ b^\dagger_{-\mathbf{k}} \end{bmatrix} = U_{B,k} \begin{bmatrix} \beta_\mathbf{k} \\ \beta^\dagger_{-\mathbf{k}}\end{bmatrix}, \hspace{0.5cm} U_{B,k} = \begin{bmatrix} u_{B,k} & -v_{B,k} \\ -v_{B,k} & u_{B,k} \end{bmatrix}. 
\end{equation}

Demanding that the $\beta_\mathbf{k}$-operators are bosonic as well leads to $\det(U_{B,k}) = u_{B,k}^2-v_{B,k}^2=1$. The analogy with the fermion case in the preceding section should now be clear. Mind the sign differences in $U_{B,k}$ and $U_{F,k}$ stemming fundamentally from requiring specific \textit{commutator} relations here, not \textit{anti}commutator relations as before. A diagonalization procedure analogous to the fermion case leads to: 
\begin{equation}
H_{BB} = \sum_\mathbf{k} E_{B,\mathbf{k}}\beta_\mathbf{k}^\dagger \beta_\mathbf{k}. 
\end{equation}

Here the ground state energy is neglected. The Bogoliubov spectrum is $E_{B,k}^2 = \xi_{B,k}^2-(n_Bg_B)^2$ with $\xi_{B,k} = \frac{k^2}{2m_B}+n_Bg_B$ the kinetic energy shifted by the mean field interaction $n_Bg_B$. Finally the condensate coherence factors $u_{B,k}$ and $v_{B,k}$ are found to be: 
\begin{equation}
u_{B,k}^2 = \frac{1}{2}\left(\frac{\xi_{B,\mathbf{k}}}{ E_{B,\mathbf{k}}}+1 \right), \hspace{0.5cm} v_{B,k}^2 = \frac{1}{2}\left(\frac{\xi_{B,\mathbf{k}}}{ E_{B,\mathbf{k}}}-1 \right)
\end{equation}

This ends our microscopic theory prerequisite. 

%%%%%%%%%%%%%%%%%%%%%%%%%%%%%%%
\subsection{Imaginary time Green's functions for the BEC}
\label{sec.BECGreens}
Later we will see, that the investigated interactions are between physical ($f$) fermions on the wire and physical ($b$) bosons in the condensate. This means, that it is the terms in the Hamiltonian with respect to the $b$-operators, that determine which Green's functions we need. These terms are $b_\mathbf{k}^\dagger b_\mathbf{k}, b_\mathbf{k}^\dagger b_{-\mathbf{k}}^\dagger$ and $b_{\mathbf{k}} b_{-\mathbf{k}}$. Further we take the average with respect to the assumed state of the system $\ket{\text{BEC}}$: the thermalized state of the condensate at nonzero temperatures. Hence, we get the following Green's functions:
\begin{align}
\feynmandiagram [inline=(a.base), horizontal=a to b] 
{
a --  [blue, fermion, edge label' = {\(\mathbf{k} \)}] b,
}; &= G_{11}(\mathbf{k},\tau) = -\bra{\text{BEC}} b_\mathbf{k}(\tau)b_\mathbf{k}^\dagger(0)\ket{\text{BEC}},  \nonumber \\
\feynmandiagram [inline=(a.base), horizontal=a to b] 
{
a --  [blue, anti majorana, edge label' = {\(\mathbf{k} \)}] b,
}; &= G_{12}(\mathbf{k},\tau) = -\bra{\text{BEC}} b_\mathbf{k}(\tau)b_{-\mathbf{k}}(0)\ket{\text{BEC}},  \nonumber \\
\feynmandiagram [inline=(a.base), horizontal=a to b] 
{
a --  [blue, majorana, edge label' = {\(\mathbf{k} \)}] b,
}; &=G_{21}(\mathbf{k},\tau) = -\bra{\text{BEC}} b^\dagger_\mathbf{k}(\tau)b_{-\mathbf{k}}^\dagger(0)\ket{\text{BEC}},  
\label{eq.defBECGreens}
\end{align}

all for $\tau > 0$. Normally one includes a imaginary time time ordering operator $T_\tau$, however here we only need the Green's functions for $\tau > 0$. Please notice, that the ground state at $T=0$ is defined by the equation $\beta_\mathbf{k}\ket{\text{BEC}}_0 = 0$. The thermalized state $\ket{\text{BEC}}$ is a superposition of all $T=0$ eigenstates, since there are thermal excitations in the ground state. The probabilities of being in one of these excited states is given by the Bose-Einstein distribution $n_B(E_\mathbf{k})$. $G_{11}$ is called the normal Green's function, whilst $G_{12}$ and $G_{21}$ are the anomalous Green's functions. This is probably because the $G_{11}$ Green's function is analogous to the one encountered for a simple non-interacting system \cite{BruusFlensberg}. $G_{12}$ and $G_{21}$ on the other hand are new "anormal" functions. Notice the simple relation: $G_{12}^* = G_{21}$. The form of the Green's functions means, that we use the diagrammatic rules shown to the left in the above.

Let us do the calculation for $G_{11}$ explicitly to see how the imaginary time formalism works. In this formalism the imaginary time Heisenberg operators are $A(\tau) = \text{e}^{\tau H_{BB}}A(0)\text{e}^{-\tau H_{BB}}$ \cite{BruusFlensberg}. It is then easy to show, that $\beta_\mathbf{k}(\tau) = \beta_\mathbf{k}\text{e}^{-\tau E_\mathbf{k}}$ and $\beta^\dagger_\mathbf{k}(\tau) = \beta^\dagger_\mathbf{k}\text{e}^{\tau E_\mathbf{k}}$. Rewriting the $b$-operators in terms of Bogoliubov $\beta_\mathbf{k}$-operators we then get\footnote{Here I am afraid the notation is a little hazardous. $\beta$ is the inverse temperature $1/(kT)$. $\beta_\mathbf{k}$ is the annihilation operator for the Bogoliubov excitations of the condensate}:
\begin{align}
G_{11}(\mathbf{k},\tau) &= -\bra{\text{BEC}} b_\mathbf{k}(\tau)b_\mathbf{k}^\dagger(0)\ket{\text{BEC}} \nonumber \\
&= -\bra{\text{BEC}} \left(u_k\beta_\mathbf{k}\text{e}^{-\tau E_\mathbf{k}} - v_k \beta^\dagger_{-\mathbf{k}}\text{e}^{\tau E_k} \right)\left( -v_k\beta_{-\mathbf{k}} + u_k \beta^\dagger_{\mathbf{k}} \right)\ket{\text{BEC}} \nonumber \\
&\overset{(1)}{=} -\bra{\text{BEC}} u_k^2\beta_\mathbf{k}\beta^\dagger_\mathbf{k}\text{e}^{-\tau E_\mathbf{k}} - v_k^2 \beta^\dagger_{-\mathbf{k}}\beta_{-\mathbf{k}}\text{e}^{\tau E_k} \ket{\text{BEC}} \nonumber \\
&\overset{(2)}{=} -\left(u_k^2 \text{e}^{-\tau E_k}(n_B(E_k)+1)+ v_k^2\text{e}^{\tau E_k}n_B(E_k)\right). \nonumber
\end{align}
In (1) we use, that $\bra{\text{BEC}}\beta_\mathbf{k}\beta_{-\mathbf{k}}\ket{\text{BEC}} = 0$, since we are removing two particles from the state to the right.\footnote{Remember that $\beta_\mathbf{k}\ket{\text{BEC}} \neq 0$ for nonzero temperatures.} The condensate is in thermal  (nondiffusive) equilibrium at some nonzero temperature $T$, and the $\beta_\mathbf{k}$-quasiparticles are bosonic excitation of the condensate. This means, that $\bra{\text{BEC}}\beta_\mathbf{k}^\dagger\beta_{\mathbf{k}}\ket{\text{BEC}} = n_B(E_k) = \frac{1}{\text{e}^{\beta E_k}-1}$, the Bose-Einstein distribution, with $\beta = \frac{1}{kT}$. This is used in (2) along with the commutator for the $\beta_\mathbf{k}$-operators. This allows us to calculate the desired Matsubara Green's function $G_{11}(\mathbf{k},i\omega_m)$, where $\omega_m = 2\pi m/\beta$ is a bosonic Matsubara frequency. The function is calculated by Fourier transforming $G_{11}(\mathbf{k},\tau)$ \cite{BruusFlensberg}: 
\begin{equation}
G_{11}(\mathbf{k},i\omega_m) = \int_0^\beta d\tau \; G_{11}(\mathbf{k},\tau) \text{e}^{i\omega_m\tau} = \frac{u_k^2}{i\omega_m-E_k}-\frac{v_k^2}{i\omega_m+E_k}. 
\end{equation}
Similarly $G_{12}(\mathbf{k},\tau)$ can be calculated. It is found to be real, and hence we also get the other anormalous Green's function: $G_{21}(\mathbf{k},\tau)=G_{12}^*(\mathbf{k},\tau)=G_{12}(\mathbf{k},\tau)$. For this reason they are also equal in frequency space. The result for them is:
\begin{equation}
G_{12}(\mathbf{k},i\omega_m)= \frac{u_kv_k}{i\omega_m+E_k}-\frac{u_kv_k}{i\omega_m-E_k}. 
\end{equation}

This ends our Green's function prerequisite.



