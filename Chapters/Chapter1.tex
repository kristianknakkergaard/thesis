% Chapter 1

\chapter{Introduction} % Main chapter title

\label{Chapter1} % For referencing the chapter elsewhere, use \ref{Chapter1} 

\lhead{Part I. \emph{Introductions}}
\chead{Chapter 1. \emph{Introduction}} % This is for the header on each page - perhaps a shortened title

%----------------------------------------------------------------------------------------
In the early days of quantum mechanics the theory was used to understand the structure of atoms. Bohr and Sommerfeld developed an early theory, that focused on explaining atomic spectral lines. Later Schr{\"o}dinger, Heisenberg and Dirac developed quantum mechanics as we know it today. The motivation was epistemological. They wanted to understand nature as it is, as is evident from the important discussions of Einstein and Bohr \cite{EinsteinEPRparadox, BohrEPRparadox}. However, the experiments that tested and pushed the theory forward was more and more manipulative. A good example is the coming of particle physics in the 1950s. Uptil then unseen states of matter was produced from collisions in increasingly high-energy beams \cite{Martin.NuclearAndParticlePhysics}. The connection to the natural world grew increasingly abstract. However, the motivation in particle physics still was, and arguably is to this day, to understand the fundamental properties of matter as they occur in nature. 

In the late 20th century a new motivation took shape. Physicist now not only wished to \textit{understand} phases of matter. Especially in quantum optics and condensed matter physics they now wanted to \textit{manipulate} and \textit{control} them. This became a motivation in its own right. Under one this is called quantum engineering. This development is driven by the construction of transistors, diods and superconductors, where precise control is key. 

In condensed matter physics it is further possible to construct and control completely \textit{new} phases of matter. Systems consisting of cold atoms constitute extremely controllable and tunable environments, where a wide range of effective interactions can be constructed, see e.g. \cite{Pohl.supersolidity, BruunZhigangTopSuperfluid}. Also the study of \textit{topological} phases is in a thriving development, being one of the biggest research areas of modern condensed matter physics and receiving the 2016 Nobel Prize in Physics \cite{NobelPrize2016}. Precise control and tunability of the system in question is here important to be able to experimentally realise the topological properties. 

\newpage
This thesis is a study of how one can realise topological superfluids in mixtures of fermions and bosons. First, we theoretically investigate two parallel fermion wires immersed in a sea of bosonic cold atoms, a 1D-3D Fermi-Bose mixture. We investigate how a point interaction between the fermions and bosons induce an attractive interaction between the fermions themselves. Using a mean field approach in a weak coupling limit, this leads to the realisation of interacting Kitaev wires with intrawire $p$-wave and interwire $s$-wave pairing. The separated wires with only $p$-wave pairing are characterized by having a topological ground state. In turn a single edge state in each wire forms. The closely spaced wires only have $s$-wave pairing, which is topologically trivial and therefore exhibit no edge states. We can hereby control a $p$- to $s$-wave phase transition. The main question then becomes, whether this transition is topological or not. 

Second, we analyse a 1D-1D mixture with the fermions trapped in a single lattice. This realises an extended Kitaev chain, where we include next-nearest neighbour hopping. The Bose gas is one-dimensional. This is done to make the interaction longer range. We investigate, how this enables us to observe pairing to several neighbours. In turn we study the possibility of realising a system with winding number larger than 1, supporting several edge states. Such a system has never been experimentally realised using superfluids. 
