% Chapter 1

\chapter{Introduction} % Main chapter title

\label{Chapter1} % For referencing the chapter elsewhere, use \ref{Chapter1} 

\lhead{Part I. \emph{Introductions}}
\chead{Chapter 1. \emph{Introduction}} % This is for the header on each page - perhaps a shortened title

%----------------------------------------------------------------------------------------
In the early days of quantum mechanics the theory was mainly used to understand naturally occuring phenomena. Bohr and Sommerfeld developed an early theory, that primarily focused on explaining the structure of atoms and in turn atomic spectral lines. Later Schr{\"o}dinger, Heisenberg and Dirac developed the theory as we know it today. The motivation was epistemological, they wanted to understand nature as it is. This characterizes the development of quantum theory through the 20th century. However, the methods of achieving this knowledge was more and more manipulative. A good example is the coming of particle physics in the 1950s. Uptil then unseen states of matter was produced from collisions in increasingly high-energy beams. The connection to the natural world grew increasingly abstract. However, the motivation in particle physics still was, and is to this day, to understand the fundamental properties of matter as they occur in nature. 

In the late 20th century a new motivation took shape. One now not only wished to \textit{understand} phases of matter. Especially in quantum optics and condensed matter physics physicists now wanted to \textit{manipulate} and \textit{control} them. This became a motivation in its own right. Under one this is called quantum engineering. This development is in part due quantum computation, where deep principles of quantum mechanics, like superposition and entanglement, is used to try to develope computers consisting of few level quantum systems as logical circuits. Precise control is therefore key. This is a driving force in much quantum optics research. 

In condensed matter physics one also meets this line of thinking. The prevailing motivation here is however to construct and control completely \textit{new} phases of matter. Systems consisting of cold atoms constitute extremely controllable and tunable environments, where effective interactions of almost arbitrary shape can be constructed. Also the study of \textit{topological} phases is in a thriving development, being one of the biggest research areas of modern condensed matter physics and receiving the 2016 Nobel Prize in Physics \cite{NobelPrize2016}. Precise control and tunability of the system in question is here important to be able to experimentally realise the topological properties. 

\newpage
This thesis is a study of how one can realise a topological superfluid in mixtures of fermions and bosons. Specifically, we theoretically investigate fermion wires immersed in a sea of bosonic cold atoms, a 1D-3D Fermi-Bose mixture. We investigate how a point interaction between the fermions and bosons induce an attractive interaction between the fermions themselves. For a single wire, using a mean field approach in a weak coupling limit, this leads to the realisation of the Kitaev model for spinless $p$-wave superfluids. The applicability of the mean field approach is also discussed. The system is characterized by having a topological phase as its ground state. Next we study a system with two parallel fermion wires. Here the interwire distance becomes a parameter with which we can control whether the interaction internally in each wire or between the wires is dominant. This makes it possible to study a $p$- to $s$-wave phase transition. The main question then becomes, whether this transition is topological or not.  

