% Chapter 1

\chapter{Introduction} % Main chapter title

\label{Chapter1} % For referencing the chapter elsewhere, use \ref{Chapter1} 

\lhead{Part I. \emph{Introductions}}
\chead{Chapter 1. \emph{Introduction}} % This is for the header on each page - perhaps a shortened title

%----------------------------------------------------------------------------------------

This thesis is a study of how one can theoretically realise a topological superfluid in mixtures of fermions and bosons. Specifically, we look at fermion wires immersed in a sea of bosons, a 1D-3D Fermi-Bose mixture. We investigate how a point interaction between the fermions and bosons induce an attractive interaction between the fermions themselves. Using a mean field approach in a weak coupling limit this leads to the realisation of the socalled Kitaev model for spinless $p$-wave superfluids. The system is characterized by having a topological phase as its ground state. 

The study of topological phases of matter is in a thriving development, being one of the biggest research areas of modern condensed matter physics and receiving the 2016 Nobel Prize in Physics \cite{NobelPrize2016}. The reason for this development is in part, that topological phases of matter gives a whole new perspective on condensed matter physics, revealing up to now unknown states of matter. Therefore, it is of high interest to realise simple yet tunable topological phases of matter. The specific 1D-3D Fermi-Bose mixture is in this sense a well suited candidate for investigation. Firstly, an analogous 2D-3D mixture has been experimentally realised, and so it is reasonable to believe that this system can as well. (INSERT REFERENCE) Secondly, just like the 2D-3D mixture there is a high degree of tunability through several system parameters. 

The thesis is organised in the following way. In chapter \ref{Chapter2} we come with a short summary of the most important ingredients of condensed matter physics for understanding the system at hand. In part II, chapters \ref{Chapter3} through \ref{Chapter7}, we investigate the \textit{single} fermion wire system. The focus is on the bulk properties of the system, specifically how the superfluidity of the fermion wire responds to changes in the system variables. This is primarily done through numerical analyses. In part III, chapters \ref{Chapter8} through \ref{Chapter10}, we investigate a two wire system of two parallel fermion wires. Here we will find, that there are two types of superfluidity appearing, both $s$- and $p$-wave. We will focus on how we can control the appearance of these by adjusting the distance between the wires. This neatly links to the underlying topological theory. In particular we will investigate how the topology of the system influences the transition between $s$- and $p$-wave superfluidity.   