\chapter{Conclusion} 
\label{Chapter10} 
\lhead{Part V. \emph{Discussions \& Conclusions}}
\chead{Chapter 10. \emph{Conclusion}} 

We have studied the many-body effects of wires of identical fermions embedded in a three-dimensional Bose-Einstein condensate. We have shown, that a point interaction between the fermions and bosons lead to an induced attractive interaction between the fermions of the Yukawa form in the weak-coupling limit. In a mean field approach we have shown, that the single wire system realises the Kitaev model of superfluidity. In turn we have found selfconsistent numerical solutions for the pairing and chemical potentials, that demonstrates a superfluid phase with $p$-wave pairing. Using the linearized gap equation we have numerically demonstrated, that any nonzero point interaction between the fermions and bosons lead to the formation of the superfluid. We have calculated the bulk topological invariant of the system and demonstrated, that the ground state is topologically nontrivial. This is verified explicitly by finding an approximate solution for the resulting edge states at the boundary of the wire. 

For the double wire system the above steps have been repeated. Here the interwire interaction is shown to lead to a competing interwire $s$-wave pairing. The resulting Hamiltonian has the structure of a spin-$1/2$ system with $p$-wave pairing between identical spins and $s$-wave pairing between opposite spins. The competition between the pairings is shown to be controllable through the interwire distance or analogously the Bose-Einstein coherence length. The edge states in each wire in the separated wire system are shown to be Kramers partners in accordance with a time reversal symmetry, that squares to minus the identity. By calculating the topological invariant we show, that there are two possible and physically distinct configurations of the system. These configurations are different only in the phase of the interwire pairing. In one, as the wires are brought closer together, the interwire pairing chooses a phase, that obeys the time reversal symmetry. In turn a topological phase transition, where the bulk energy gap closes, must occur as the wires are brought closer together. In the other, the interwire pairing chooses a phase, that breaks this time reversal symmetry. The Kramers partners of edge states are then shown to couple and gain a nonzero energy. Then in a numerical analysis we find the selfconsistent solutions for the pairing and chemical potentials. We hereby show, that the configuration that breaks the time reversal symmetry is the energetically favourable. Using mean field theory, we therefore predict, that the two wire system will undergo a normal, nontopological phase transition between a topologically nontrivial and trivial phase. This is the main result of the thesis.

We emphasize, that we are aware, that the mean field approach is problematic for these one-dimensional systems. However, we speculate that in the long coherence length limit, the results will be qualitatively correct. 