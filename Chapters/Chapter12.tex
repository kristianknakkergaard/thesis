\chapter{Discussion} % Main chapter title

\label{Chapter12} % Change X to a consecutive number; for referencing this chapter elsewhere, use \ref{ChapterX}
\lhead{Part IV. \emph{Discussions \& Conclusions}}
\chead{Chapter 12. \emph{Discussion}} % Change X to a consecutive number; this is for the header on each page - perhaps a shortened title

\section{Mean field approach} \label{sec.Discussion.meanfieldapproach}
In section \ref{sec.meanfieldvalidity} it was pointed out, that using the mean field approach in a one dimensional system is possibly quite problematic. It was also mentioned, that we would need to take steps beyond mean field theory to actually estimate, how this influences the analysis and in turn the results. It is clear, that if the mean field approach is problematic in the single wire system, the situation is only worsened in the double wire system. 

In the spirit of this we do not expect the results of the thesis to be quantitatively in agreement with the findings of an experimental realisation of the system. A simple way to see this is the following. Consider the gap equations for the double wire system, equation \eqref{eq.2wiresgapequations}. As we bring the wires closer together the intrawire pairing is not affected by the interwire induced interaction until an interwire pairing potential forms. This is absurd. The interwire interaction will of course alter the physical properties internally in each wire, also when the wires are far apart. This also leads to the odd prediction, that the critical temperature for intrawire pairing is unaffected by the interwire interaction, as long as the interwire pairing is absent. This is \textit{not} physically reasonable. 

The relevance of the present thesis is therefore in some sense in doubt. However, also as mentioned in section \ref{sec.meanfieldvalidity} there has been an impressive amount of work in one dimensional superfluid system, most in its fundamentals based on mean field theory. This includes systems where different types of pairings is in competition, see e.g. \cite{LiYangChen}. This suggests, that although the analysis might be quantitavely wrong, as mean field theory always is to some extent, the qualitative behaviour is correct. Hence, we would e.g. suspect that although the specific form of the cross over from intra- to interwire pairing is wrong, it remains valid that the intra- and interwire pairings coexist in a cross over region, when the interwire pairing is imaginary. It is more uncertain in the case of a real interwire pairing, whether the abrubt shift from intra- to interwire pairing remains valid. However, an analogous calculation in a 2D-3D system with fermions in two parallel layers give the same qualitative result. \footnote{Private communication with Jonathan Melk{\ae}r Midtgaard.} Since, the mean field theory is suspected to be more accurate in two dimensions, this does support the findings of the present thesis. 

The qualitative behaviour is also supported by the topological invariants calculated in section \ref{sec.2wires_CSinv}. However, from this it is still unclear whether the transition in the case of a real interwire pairing is abrubt, as we also mentioned in the end of subsection \ref{subsec.2wires_CSinv_Delta12real}. 

Further, the one dimensional system also serves as an interest starting point for more advanced studies in higher dimensions. This is especially because the concepts and more crucially the numerics are simpler in one dimension. Hence, we can gain insight and intuition for Fermi-Bose mixtures by addressing a simplified one dimensional case. 

It is however by no means a simple task to estimate the validity of the mean field approach. Such an analysis is therefore an obvious candidate for future work.

\section{Several wires}
In this thesis we have looked at one and two fermionic wires embedded in a Bose-Einstein condensate. An obvious thing one might then do is to study systems consisting of severeal such wires placed in e.g. regular polygons. This is also a unique configuration for the 1D-3D systems. Let us take the case of three wires as an example. 

Given that the mean field approach produces qualitatively correct results we can discuss the expected physical properties of such a system. If we fix all intrawire pairings to be identical, the three phases of the interwire pairings are independent. This are between the wires according to: $1 \leftrightarrow 2, 2 \leftrightarrow 3$ and $3 \leftrightarrow 1$. Now the calculation in subsection \ref{subsec.2wires_CSinv_Delta12imag} suggests that even in the absence of Kramers degeneracy\footnote{Which is to say, that there is \textit{no} $T^2 = -\mathbb{I}$ symmetry.}, the separated wires constitute a topologically nontrivial system. For the separated system we namely expect the invariant to be the sum of the invariants for three single wire systems: $\nu = \nu_1 + \nu_2 + \nu_3$. The sign of the three invariants depends on the chosen sign of the corresponding intrawire pairing. Hence the sum gives four equivalent results $-3, -1, 1$ and $3$. All of these are different from the topologically trivial value of $\nu = 0$. Further, when the wires are closely spaced the interwire $s$-wave pairings will dominate, which is topologically trivial. This means, that there is a topological phase transition in the system. Hence, the bulk energy gap is forced to close in a three wire system. This makes such a system a good candidate for investigation. The downside is, that we now have to keep the three phases of the interwire pairings free and make a diagonalisation of a $3 \times 3$ matrix.  


