% Chapter 4

\chapter{The one dimensional Fermi Hamiltonian} % Main chapter title

\label{Chapter4} % For referencing the chapter elsewhere, use \ref{Chapter3} 

\lhead{Chapter 4. \emph{1D Fermi Hamiltonian}} % This is for the header on each page - perhaps a shortened title

%----------------------------------------------------------------------------------------
In this chapter we will finally arrive at the promised Kitaev Hamiltonian promised in chapter \ref{Chapter2}. Firstly, we calculate the pair interaction Hamiltonian for the fermions in section \ref{sec.HFFint}. This is done under the assumption, that we can restrict the induced interaction to have a vanishing Matsubara frequency: $\omega_q = 0$. I will make an argument for the validity of this approach in section \ref{sec.RetardationEffects}. The result for the interaction Hamiltonian is then inserted into the full Hamiltonian in section \ref{sec.HFFfull}. 

\section{The fermion interacting Hamiltonian} \label{sec.HFFint}

We start out in real space. The interaction Hamiltonian for pair interactions is given by:
\begin{equation}
H^\text{int}_{FF} = \frac{1}{2}\int dxdx' \hat{\psi}^\dagger_F(x)\hat{\psi}^\dagger_F(x')\tilde{V}^\text{ind}_{FF}(x'-x,0) \hat{\psi}_F(x') \hat{\psi}_F(x).
\label{eq.HFFintdef}
\end{equation}
With $\tilde{V}^\text{ind}_{FF}(x'-x,0)$ being the induced interaction at zero frequency in real space (see equation \eqref{eq.VFFx_exact}). The factor of $1/2$ is present, since the particles are identical. We make a mean field approximation to get the Hamiltonian on a solvable quadratic form. We write:
\begin{equation}
\hat{\psi}_F(x') \hat{\psi}_F(x) = \braket{\hat{\psi}_F(x') \hat{\psi}_F(x)} + \left(\hat{\psi}_F(x') \hat{\psi}_F(x) - \braket{\hat{\psi}_F(x') \hat{\psi}_F(x)} \right) = \braket{\hat{\psi}_F(x') \hat{\psi}_F(x)} + A_F(x',x)
\end{equation}
and treat the last part $A_F$ as a small quantity. The meaning of this will become clearer later on. The mean in the end has to be taken with respect to the specific state of the system, we will also see how this plays out later on as well. We now only keep terms of $A_F$ and $A^\dagger_F$ up to first order, so that we get the following three terms for the interaction Hamiltonian:\footnote{Formally we are decoupling the Hamiltonian, so that it so no longer really describing any interaction.}
\begin{align}
H^\text{int}_{FF} = & \frac{1}{2}\int dxdx' \braket {\hat{\psi}^\dagger_F(x) \hat{\psi}^\dagger_F(x')}  \tilde{V}^\text{ind}_{FF}(x'-x,0) \braket{\hat{\psi}_F(x') \hat{\psi}_F(x)} + \nonumber \\
& \frac{1}{2}\int dxdx' A^\dagger_F(x,x') \tilde{V}^\text{ind}_{FF}(x'-x,0) \braket{\hat{\psi}_F(x') \hat{\psi}_F(x)} + \nonumber \\
& \frac{1}{2}\int dxdx' \braket{\hat{\psi}^\dagger_F(x) \hat{\psi}^\dagger_F(x')} \tilde{V}^\text{ind}_{FF}(x'-x,0) A_F(x',x). \nonumber
\end{align} 
To get back to momentum space we expand the field operators in terms of plane waves: $\hat{\psi}_F(x) = \frac{1}{\sqrt{\mathcal{L}}}\sum_k \text{e}^{ikx} f_k$, where as usual $f_k$ is the fermion annihilation operator. Further we assume, that only states belonging to opposite momenta couples.\footnote{At least they are assumed to be absolutely dominant.} This means, that $f_kf_{k'} = f_kf_{k'}\delta_{k,-k'}$, which is the usual BCS assumption. With this, we get the following expression:
\begin{align}
\hat{\psi}_F(x') \hat{\psi}_F(x) &= \frac{1}{\mathcal{L}}\sum_{k} \text{e}^{ik(x-x')}f_kf_{-k} = \frac{1}{2\mathcal{L}}\sum_{k} \left(\text{e}^{ik(x-x')}f_{k}f_{-k}+\text{e}^{-ik(x-x')}f_{-k}f_{k}\right) \nonumber \\
&= \frac{i}{\mathcal{L}}\sum_{k} \sin(k(x-x'))f_{k}f_{-k}, 
\end{align}
where we in the last equality used, that the operators anticommute. Let us plug this into the first term of the interaction Hamiltonian to see how it plays out: 
\begin{align}
H^\text{int}_{FF,1} &= \frac{1}{2}\int dxdx' \braket {\hat{\psi}^\dagger_F(x) \hat{\psi}^\dagger_F(x')} \tilde{V}^\text{ind}_{FF}(x'-x,0) \braket {\hat{\psi}_F(x') \hat{\psi}_F(x)} \nonumber \\
&= - \frac{1}{2\mathcal{L}^2}\sum_{k,k'} \braket {f^\dagger_{k}f^\dagger_{-k} } \braket {f_{k'}f_{-k'}} \int dx dx' \sin(k(x'-x))\sin(k'(x'-x))\tilde{V}^\text{ind}_{FF}(x'-x,0) \nonumber \\
&= - \frac{1}{2\mathcal{L}}\sum_{k,k'} \braket {f^\dagger_{k}f^\dagger_{-k}} \braket{f_{k'}f_{-k'}} \int du \sin(ku)\sin(k'u)\tilde{V}^\text{ind}_{FF}(u,0) \nonumber \\
&= - \frac{1}{2\mathcal{L}}\sum_{k,k'} \braket {f^\dagger_{k}f^\dagger_{-k}} \braket {f_{k'}f_{-k'}} W^\text{ind}_{FF}(k,k')
\end{align}
where we use the substitution $u = x'-x$ and use that $\int dx = \mathcal{L}$. Further we have defined the coupling potential $W^\text{ind}_{FF}(k,k')$. The two last terms are analogous and yield:
\begin{align}
H^\text{int}_{FF,2} &= - \frac{1}{2\mathcal{L}}\sum_{k,k'}\left(f^\dagger_k f^\dagger_{-k} - \braket {f^\dagger_k f^\dagger_{-k}}\right) \braket {f_{k'}f_{-k'}} W^\text{ind}_{FF}(k,k'), \nonumber \\
H^\text{int}_{FF,3} &= - \frac{1}{2\mathcal{L}}\sum_{k,k'}\left(f_k f_{-k} - \braket{f_k f_{-k}} \right)\braket {f^\dagger_{k'}f^\dagger_{-k'} } W^\text{ind}_{FF}(k,k'), \nonumber
\end{align}
Now seeing, that the coupling potential enters in all of the terms, we better calculate, what it actually is. We get, that it can be expressed simply in terms of the induced interaction in momentum space calculated in chapter \ref{Chapter3}:
\begin{equation}
W^\text{ind}_{FF}(k,k') = \int du \sin(ku)\sin(k'u)\tilde{V}^\text{ind}_{FF}(u,0) = \frac{1}{2}\left[V^\text{ind}_{FF}(k-k',0) - V^\text{ind}_{FF}(k+k',0)\right].
\label{eq.CouplingPotential}
\end{equation}
Since $V^\text{ind}_{FF}(q,0)$ only depends on $q^2$, we get the following important properties of the coupling potential: 
\begin{align}
W^\text{ind}_{FF}(k,k') &= W^\text{ind}_{FF}(k',k), \hspace{0.5cm} \text{Symmetry in arguments}, \nonumber \\
W^\text{ind}_{FF}(-k,k') &= -W^\text{ind}_{FF}(k,k'), \hspace{0.5cm} \text{Odd in single argument}, \nonumber \\
W^\text{ind}_{FF}(-k,-k') &= W^\text{ind}_{FF}(k,k'), \hspace{0.5cm} \text{Even in double argument}.
\label{eq.CouplingPotentialSymmetries}
\end{align}
These symmmetries will be important later on. From here on and out the analysis is very similar to the historical Bardeen-Cooper-Schrieffer treatment. However, we will not be assuming any constancy of the coupling potential over a range of momentum, as is the case in traditional BCS-theory\cite{Tinkham,LandauStatPhys2,PlischkeStatPhys}. Adding the three terms yields:
\begin{equation}
H^\text{int}_{FF} = -\frac{1}{2\mathcal{L}}\sum_{k,k'} W^\text{ind}_{FF}(k,k')\left[ \braket{ f_{k'}f_{-k'}} f^\dagger_k f^\dagger_{-k} + \braket {f^\dagger_{k'}f^\dagger_{-k'}} f_k f_{-k} - \braket {f_{k}f_{-k}} \braket {f^\dagger_{k'}f^\dagger_{-k'}} \right] \nonumber
\end{equation}

We simplify the expression by defining the pairing potential:
\begin{equation}
\Delta_k = -\frac{1}{\mathcal{L}}\sum_{k'}W^\text{ind}_{FF}(k,k')\braket {f_{k'}f_{-k'}}.
\label{eq.pairingpotentialdef}
\end{equation}

Then the interaction Hamiltonian can finally be written as:
\begin{equation}
H^\text{int}_{FF} = \frac{1}{2}\sum_{k} \left[ \Delta_k f^\dagger_k f^\dagger_{-k} - \Delta^*_k f_k f_{-k} - \Delta_k \braket {f^\dagger_{k'}f^\dagger_{-k'}} \right]
\label{eq.HFFintfinal}
\end{equation}
The minus sign in front of $\Delta^*_k$ comes from anticommuting the operators. Since the coupling potential is odd in a single argument $\Delta_{-k} = -\Delta_k$: the pairing potential is odd as well. We are now ready to study the full Hamiltonian.

\section{The full fermion Hamiltonian} \label{sec.HFFfull}
The free particle Hamiltonian for the fermions is of course $H_0 = \sum_k \frac{k^2}{2m_F} f^\dagger_k f_k$. Further we inforce, that the number of fermionic particles is conserved. This is achieved by subtracting $\mu N_F = \mu \sum_k f^\dagger_k f_k$, where $\mu$ is the chemical potential. We hereby obtain the full fermion (grand) Hamiltonian:
\begin{equation}
H_{FF} = H_0-\mu N_F + H^\text{int}_{FF} = \sum_k \varepsilon_k f^\dagger_k f_k + \frac{1}{2}\sum_{k} \left[ \Delta_k f^\dagger_k f^\dagger_{-k} - \Delta^*_k f_k f_{-k} - \Delta_k \braket {f^\dagger_{k'}f^\dagger_{-k'}} \right], 
\label{eq.HFFdef}
\end{equation} 
where $\varepsilon_k = \frac{k^2}{2m_F}-\mu$ is the kinetic energy relative to the chemical potential. We can bring the above into a matrix form:
\begin{equation}
H_{FF} = -\sum_k \Delta_k \braket {f^\dagger_k f^\dagger_{-k}} + \frac{1}{2}\sum_{k} F_k^\dagger \mathcal{H}_{FF,k} F_k, \hspace{0.5cm} \mathcal{H}_{FF,k} = \begin{bmatrix} \varepsilon_k & \Delta_k \\ \Delta^*_k & -\varepsilon_k \end{bmatrix}, \hspace{0.5cm} F^\dagger_k = \begin{bmatrix} f_k^\dagger & f_{-k} \end{bmatrix}, 
\end{equation}

and we see that up to a constant it is on the same form as the Kitaev Hamiltonian defined in equation \eqref{eq.HKitaevpre}. Hence, we can use the diagonalization made there modified by the constant $-\sum_k \Delta_k\braket {f^\dagger_k f^\dagger_{-k}} $ and get a result completely analogous to equation \eqref{eq.Kitaev.H_diagonalpre}: 
\begin{equation}
H_{FF} = \frac{1}{2}\sum_k (\varepsilon_k-2E_{F,k}-2\Delta_k\braket {f^\dagger_k f^\dagger_{-k}}) + \sum_k E_{F,k} \zeta^\dagger_k \zeta_k, \hspace{0.5cm} E_{F,k} = \sqrt{\varepsilon_k^2 + |\Delta_k|^2}.
\label{eq.Kitaev.HFF_diagonal}
\end{equation}
The new quasiparticle fermionic operators are defined in equation \eqref{eq.fermionquasiparticledef} with $u_{F,k},v_{F,k}$ defined in equation \eqref{eq.Kitaev.uk_vk}. From this we see, that the ground state of the system at $T=0$ is defined by having no quasiparticles $\zeta$ present: $\zeta_k \ket{\text{S}}_0 = 0$ for all $k$.\footnote{S is short for \text{S}uperconducting ground state.} Further we now know have to take the averages met in this section. They are to be taken with respect to the thermalized state of the system, which in turn can be calculated using Fermi statistics, as we shall see in the following section. 

\section{The distribution of the quasiparticles}
In the previous section we inforced, that the number of physical ($f$) fermions in the system is fixed. However, we do not fix the number of quasiparticles ($\zeta$). This means, that the partition function takes the form of the \textit{canonical} partition function: $Z = \tr\left[\text{e}^{-\beta H_{FF}}\right]$. In terms of thermodynamics this means, that every single quasiparticle is in thermal (and not diffusive) equilibrium with all the others, hence working as a heat reservoir. Since the fermion Hamiltonian is diagonal in the quasiparticles $\zeta_k$, we can calculate the partition function for each $k$ by replacing $H_{FF}$ with the $k$'th (diagonal) term. Dropping the unimportant ground state energy $E_0 = \frac{1}{2}\sum_k (\varepsilon_k-2E_{F,k}-2\Delta_k\braket {f^\dagger_k f^\dagger_{-k}})$, we get:
\begin{equation}
Z_k = \tr\left[\text{e}^{-\beta H_{FF,k}}\right] = \tr\left[\text{e}^{-\beta E_{F,k}\zeta^\dagger_k\zeta_k }\right] = 1 + \text{e}^{-\beta E_{F,k}}. 
\end{equation}     
For the calculation of the trace, one can for example use the single particle complete basis $\{\ket{\text{S}}_0, \zeta^\dagger_k\ket{\text{S}}_0\}$. This is all a rather involved way of saying, that the quasiparticle can either be absent $\ket{\text{S}}_0$ and have zero energy or present $\zeta^\dagger_k\ket{\text{S}}_0$ and have energy $E_{F,k}$. Finally we get the Fermi-Dirac distribution of the quasiparticles:
\begin{equation}
f(E_{F,k}) = \braket {\zeta^\dagger_k\zeta_k} = \sum_N N P_k(N) = \frac{0\cdot 1 + 1 \cdot \text{e}^{-\beta E_{F,k}} }{Z_k} = \frac{1}{\text{e}^{\beta E_{F,k}} + 1}, 
\end{equation}
where $N=0,1$ is the possible number of fermions in the state, and $P_k(N) = \text{e}^{-\beta N E_{F,k}}/Z_k$ is the probability of the state being occupied\cite{PlischkeStatPhys,SchroederThermal}. This is the average number of $\zeta_k$ particles in the thermalized state of the system. We now know how to take all the averages met so far, and so the program is (in principle) clear.  

\section{Retardation effects} \label{sec.RetardationEffects}
We are now in a position, where we can understand the $\omega_q = 0$ limit. Let us first remind ourselves of the frequency dependent induced interaction in equation \eqref{eq.VFFindXBEC}:
\begin{equation}
V_{FF}^\text{ind}(q,i\omega_n) = g_{BF}^2\int\frac{d^2k_\perp}{(2\pi)^2}\frac{k^2}{m_B}\frac{n_B}{(i\omega_n)^2-E_{B,k}^2}\text{e}^{-\frac{l_t^2}{2}k_\perp^2}. \nonumber
\end{equation}
The frequency dependency of the induced interaction reflects, that the fermions do not interact instanteneously leading to socalled retardation effects.\footnote{This is analogous to the retarded fields in electrodynamics. There it reflects the finiteness of the speed of light.} In turn this embodies, that the bosons in the condensate, the mediators of the fermion-fermion interaction, moves at a finite speed $c_0 = \frac{\sqrt{4\pi n_B a_B}}{m_B}$. To neglect these effects we therefore need to assume, that the typical speed of the fermions is much smaller than the speed of the bosons: $v_F \ll c_0$, $v_F = k_F/m_F$ the Fermi speed for free fermions. This leads to the relation:
\begin{equation}
\left(\frac{m_F}{m_B}\right)^2\frac{4}{\pi^2}\frac{n_B}{n_F^3}k_Fa_B \gg 1.
\label{eq.RetardationEffectsneglectionassumption}
\end{equation}
This justifies the neglection of the retardation effects in the preceding sections. 

\section{The $l_t \to 0$ limit}
In this section we will finally see, how we now can obtain a $l_t\to 0$ limit. From equation \eqref{eq.CouplingPotential} and the alternative form of the induced interaction in momentum space written after equation \eqref{eq.VFF(q,0)}, we get that:
\begin{align}
W^\text{ind}_{FF}(k,k') &= \frac{1}{2}\left[V^\text{ind}_{FF}(k-k',0) - V^\text{ind}_{FF}(k+k',0)\right] \nonumber \\
&= -\frac{m_Bg_{BF}^2n_B}{2\pi}\left[ \text{e}^{F(k-k')} E_1(F(k-k')) - \text{e}^{F(k+k')} E_1(F(k+k')) \right], \nonumber
\end{align}
where $E_1(x) = \int_1^\infty du \frac{\text{e}^{-xu}}{u}$ is the exponential integral. Separately $V^\text{ind}_{FF}(k-k',0)$ and $V^\text{ind}_{FF}(k+k',0)$ do not have a well-defined $l_t \to 0$ limit. However, it turns out that the coupling potential $W^\text{ind}_{FF}(k,k')$ does. To see this, we use the asymptotic behaviour of $E_1(x)$ for $0 < x \ll 1$: $E_1(x) \approx C -\ln(x)$, where $C$ is a constant \footnote{The constant is minus the Euler-Mascheroni constant $\gamma$. To ten digits precision $\gamma = 0.5772156649$. This expansion is found by using Maple 16.}. The exponentials $\text{e}^{F(k-k')}$ and $\text{e}^{F(k+k')}$ just give $1$ in the $l_t \to 0$ limit, and so we are left with the expression:
\begin{equation}
\lim_{l_t \to 0} \; W^\text{ind}_{FF}(k,k') = -\frac{m_Bg_{BF}^2n_B}{2\pi} \ln\left[\frac{(k+k')^2+2/\xi^2}{(k-k')^2+2/\xi^2}\right].
\label{eq.CouplingPotentiallt=0} 
\end{equation}
We finally have a closed form expression for the coupling potential. This will be used extensively in the following. We notice that along the lines $k = k'$ and $k = -k'$ the coupling potential diverges logarithmically for $k\to \infty$. Away from these lines the function quickly goes to zero.

\section{The ground state and Cooper pairs}
The ground state $\ket{\text{S}}_0$ can be expressed simply in terms of the original $f$-operators as:
\begin{equation}
\ket{\text{S}}_0 = \prod_{k > 0} \left(u_{F,k} + v_{F,k}f^\dagger_{-k}f^\dagger_k\right)\ket{0},
\end{equation} 
where $\ket{0}$ is the vacuum: $f_k\ket{0} = 0$. This can be verified explicitly by checking, that $\zeta_k\ket{\text{S}}_0 = 0$ for all $k$, and that 
$\braket{\text{S}|\text{S}} = 1$. In this it is important to notice, that the terms $(u_{F,k} + v_{F,k}f^\dagger_{-k}f^\dagger_k)$ commute with each other, since all daggered operators anticommute. The appearance of $f^\dagger_{-k}f^\dagger_k$ for each $k$ is a manifestation of the fact, that the fermions pair up in socalled Cooper pairs with opposite momenta. In this line of thinking it is also interesting to see, how the singly excited states look. A simple calculation shows:
\begin{equation}
\zeta^\dagger_k\ket{\text{S}}_0 = \prod_{k' > 0, k' \neq k }\left(u_{F,k'} + v_{F,k'}f^\dagger_{-k'}f^\dagger_{k'}\right)f^\dagger_k\ket{0}.
\end{equation}
The action of $\zeta^\dagger_k$ is thus to break up the $k$'th Cooper pair. This also shows, that the ground state for an even number of fermions is lower, than that for an odd number. 

As is the case for other mean field theories the ground state has a broken symmetry.\footnote{This e.g. includes ferromagnets and Bose Einstein condensates.} By inspecting 
