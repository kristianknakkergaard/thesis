% Chapter 4

\chapter{Grand Fermi Hamiltonian} % Main chapter title

\label{Chapter4} % For referencing the chapter elsewhere, use \ref{Chapter3} 

\lhead{Part II. \emph{One wire}}
\chead{Chapter 4. \emph{Grand Fermi Hamiltonian}} % This is for the header on each page - perhaps a shortened title

%----------------------------------------------------------------------------------------
In this chapter we will study the induced interactions more extensively. Firstly, we calculate the pair interaction Hamiltonian for the fermions in section \ref{sec.HFFint}. This is done under the assumption, that we can restrict the induced interactions to have a vanishing Matsubara frequency: $\omega_q = 0$. See subsection \ref{sec.RetardationEffects} for details. The result for the interaction Hamiltonian is then inserted into the full Hamiltonian in section \ref{sec.HFFfull}. The approach is an example of a Bardeen-Cooper-Schrieffer (BCS) theory, see e.g. \cite[chapter 3]{Tinkham}, \cite[pp. 153-163]{LandauStatPhys2} and \cite[pp. 359-369]{PlischkeStatPhys}. 

\section{Interaction Hamiltonian} \label{sec.HFFint}
In this section we transform the interactions from real to momentum space. The \textit{intrawire} interaction Hamiltonian for pair interactions between fermions in wire 1 is given by:
\begin{equation}
H^\text{int,11}_{FF} = \frac{1}{2}\int dx_1dx_2\; \psi^\dagger_{1,F}(x_1)\psi^\dagger_{1,F}(x_2)\tilde{V}^{11}_{\text{ind}}(x_1 - x_2, 0) \psi_{1,F}(x_2) \psi_{1,F}(x_1).
\label{eq.HFF11intdef}
\end{equation}
With $\tilde{V}^{11}_\text{ind}(x,0)$ the induced interaction at zero frequency in real space, equation \eqref{eq.V11indx}. The factor of $1/2$ is present, since the particles are identical. Transforming the above expression to momentum space will make it clearer of how to perform the BCS assumption and in turn the mean field approximation. 

The transformation to momentum space is carried out by expanding the field operators in plane waves: $\psi_{1,F}(x) = \frac{1}{\sqrt{\mathcal{L}}}\sum_k \text{e}^{ikx} c_{1,k}$. The transformation is a little tricky, because the momentum space induced interaction $V^{11}_{\text{ind}}(k, 0)$ is not well-defined in the zero trapping width limit: $l_t = 0$.\footnote{The Yukawa potential simply has no Fourier transform in one dimension.} However, it is possible to get an expression in momentum space that is well-defined for any value of $l_t$. We do this in the following manner. Firstly:
\begin{align}
\psi_{1,F}(x_2) \psi_{1,F}(x_1) &= \frac{1}{\mathcal{L}}\sum_{k_1,k_2} \text{e}^{i(k_1x_1 + k_2x_2)} c_{1, k_2}c_{1, k_1} = \frac{1}{2\mathcal{L}}\sum_{k_1,k_2} \text{e}^{i(k_1x_1 + k_2x_2)} \left[c_{1, k_2}c_{1, k_1} - c_{1, k_1}c_{1, k_2}\right] \nonumber \\
&= \frac{1}{2\mathcal{L}}\sum_{k_1,k_2} \left[\text{e}^{i(k_1x_1 + k_2x_2)} - \text{e}^{i(k_2x_1 + k_1x_2)}\right]c_{1, k_2}c_{1, k_1}, \nonumber
\end{align}
where we in the last equality switch the momenta in the last term: $k_1 \leftrightarrow k_2$. In this way we take care of the antisymmetry of the fermionic operators. A similar expression for $\psi^\dagger_{1, F}(x_1)\psi^\dagger_{1, F}(x_2)$ is obtained by letting $k_1 \to q_1$ and $k_2 \to q_2$ and taking the hermitian conjugate. All of this is then inserted into the interaction Hamiltonian of equation \eqref{eq.HFF11intdef}. After a little computation, we get:
\begin{align}
H^\text{int,11}_{FF} = \frac{1}{8\mathcal{L}^2} \sum_{k_1,k_2,q_1,q_2} c^\dagger_{1, q_1}c^\dagger_{1, q_2}c_{1, k_2}c_{1, k_1} & \int dx_2 \; \text{e}^{i((k_1 + k_2) -(q_1 + q_2))x_2}\cdot \nonumber \\ 
&\int du\left( \text{e}^{-iq_1u} - \text{e}^{-iq_2u} \right)\left( \text{e}^{ik_1u} - \text{e}^{ik_2u} \right)\tilde{V}^{11}_\text{ind}(u, 0), \nonumber
\end{align}
where we have made the substitution of variables $u = x_1 - x_2$. The integral over $x_2$ is now independent of $\tilde{V}^{11}_{\text{ind}}$. Since, we are working with periodic boundary conditions, we get $\int dx_2 \text{e}^{i((k_1 + k_2) - (q_1 + q_2))u} = \mathcal{L}\delta_{k_1 + k_2, q_1 + q_2}$, which ensures total conservation of momentum. This means, that we can get the above expression on the following form: 
\begin{equation}
H^\text{int,11}_{FF} = \frac{1}{4\mathcal{L}} \sum_{k, q, p} c^\dagger_{1, k + p}c^\dagger_{1, q - p}c_{1, q}c_{1, k} \int du\left[\cos\left(pu\right) - \cos\left(\left( p + k - q \right)u\right)\right]\tilde{V}^{11}_{\text{ind}}(u,0), \nonumber
\end{equation}
where we have made the substitutions: $k_1 = k, k_2 = q, q_1 = k + p$ and $q_2 = k - p$. Finally, since the induced interaction is even in $u$, we have that $V^{11}_{\text{ind}}(p, 0) = \int du \cos(pu) \tilde{V}^{11}_{\text{ind}}(u, 0)$. Hence, we see that the scattering amplitude $W^{11}_{\text{ind}}(k, q, p) = \frac{1}{2}\left(V^{11}_\text{ind}\left( p, 0 \right) - V^{11}_\text{ind}\left( p + k - q, 0 \right) \right)$ arises as promised. We hereby obtain the final Hamiltonian expanded in momentum space operators:
\begin{equation}
H^\text{int,11}_{FF} = \frac{1}{2\mathcal{L}} \sum_{k,q,p} W_{\text{ind}}(k, q, p) c^\dagger_{1, k + p} c^\dagger_{1, q - p} c_{1, q} c_{1, k}. 
\label{eq.H11intMomentumSpace}
\end{equation}
The summand describes a scattering of two fermions. They exchange momentum $p$ with an amplitude $W_{\text{ind}}(k, q, p)$. This depends both on the exchange, $p$, and the difference of incoming momenta, $k - q$. In subsection \ref{subsec.inducedinteraction.momentumspace} we showed, that this scattering amplitude has a well-defined $l_t \to 0$ limit. The total interaction Hamiltonian has hereby been transformed to momentum space. The interaction of fermions in wire 2 is identically to the above by letting $c_1 \to c_2$.  

Now we turn to the \textit{interwire} interaction Hamiltonian:
\begin{equation}
H^\text{int,12}_{FF} = \int dx_1 dx_2 \psi^\dagger_{1,F}(x_1)\psi^\dagger_{2,F}(x_2) \tilde{V}_{\text{ind}}^{12}(x_1 - x_2,0) \psi_{2,F}(x_2)\psi_{1,F}(x_1).
\label{eq.Hint12realspace}
\end{equation}
The factor in front of $1/2$ is absent, since the fermions in wire 1 and 2 are distinguishable. By going to momentum space a little calculation shows that it can be written as:
\begin{equation}
H^\text{int,12}_{FF} = \frac{1}{\mathcal{L}}\sum_{k,q,p} V_{\text{ind}}^{12}(p, 0) c^\dagger_{1,k + p} c^\dagger_{2, q - p} c_{2, q} c_{1, k}. 
\label{eq.Hint12momentumspace}
\end{equation}
The summand describes a scattering with a momentum exchange $p$ and amplitude $V_{\text{ind}}^{12}(p,0)$. 

\section{Mean field approximation} \label{sec.meanfieldapproximation}
In this section we will make the BCS mean field approximation. 

We again start with the intrawire interaction. First, we assume that only states belonging to opposite momenta couples. This is the usual BCS assumption. This means, that we can truncate the sum in equation \eqref{eq.H11intMomentumSpace} to only have $q = -k$. Second, we make a mean field approximation to get the Hamiltonian on a solvable quadratic form. For wire 1, we write:
\begin{equation}
c_{1, -k}c_{1, k} = \braket{ c_{1, -k}c_{1, k} } + \left(c_{1, -k} c_{1, k} - \braket{ c_{1, -k}c_{1, k} } \right) = \braket{ c_{1, -k}c_{1, k} } + A_{1,k}, \nonumber 
\end{equation}
and treat the last part $A_{1,k}$ as a small quantity. In this sense, $\braket{ c_{1, -k}c_{1, k} }$ is an order parameter of the phase transition. We return to this later on. The validity of this approach is discussed in section \ref{sec.meanfieldvalidity}. The mean is the thermal average: $\braket{ c_{1, -k}c_{1, k} } = \tr\left[\text{e}^{-\beta H_{FF}}c_{1, -k}c_{1, k} \right]/Z$ with $Z = \tr\left[\text{e}^{-\beta H_{FF}}\right]$ the partition function. We now only keep terms of $A_{1,k}$ and $A^\dagger_{1,k}$ up to first order. Hence:
\begin{equation}
H^\text{int,11}_{FF} = \frac{1}{2\mathcal{L}} \sum_{k, p} W^{11}_{\text{ind}}(k, -k, p)\left[A^\dagger_{1, k + p}\braket{c_{1, -k}c_{1,k}} + A_{1, k}\braket{c^\dagger_{1, k + p}c^\dagger_{1, -(k+p)}} + \braket{c_{1, -k}c_{1,k}}\braket{c^\dagger_{1, k + p}c^\dagger_{1, -(k + p)}}\right]. 
\label{eq.H11int.meanfield.firstexpression}
\end{equation}
We now write $p = k' - k$, and define the effective intrawire induced interaction as:
\begin{equation}
W^{11}_{\text{ind}}(k, k') = W^{11}_{\text{ind}}(k, -k, p = k' - k) = \frac{1}{2}\left(V^{11}_{\text{ind}}\left( k - k', 0 \right) - V^{11}_{\text{ind}}\left( k + k', 0 \right) \right), 
\label{eq.EffectiveInteraction.intrawire}
\end{equation}
with the use of equation \eqref{eq.Wkqp.scattering.amplitude}. The analysis is a BCS treatment as mentioned. However, we will not be assuming any constancy of the effective interaction over a range of momentum, as is the case in traditional BCS-theory \cite[chapter 3]{Tinkham}. Since $V^{11}_{\text{ind}}(q,0)$ only depends on $q^2$, we notice, that $W^{11}_{\text{ind}}(k, k')$ is odd in both arguments. The above sum can hereby be brought on the form:
\begin{equation}
H^\text{int,11}_{FF} = -\frac{1}{2\mathcal{L}}\sum_{k,k'} W^{11}_{\text{ind}}(k, k')\left[ \braket{ c_{1, k'}c_{1, -k'}} c^\dagger_{1,k} c^\dagger_{1, -k} + \braket {c^\dagger_{1, k'}c^\dagger_{1, -k'}} c_{1, k} c_{1, -k} - \braket {c_{1, k'}c_{1, -k'}} \braket {c^\dagger_{1, k}c^\dagger_{1, -k}} \right]. \nonumber
\end{equation}
The same procedure is performed for the fermions in wire 2. We simplify the expression by defining the intrawire pairing potentials:
\begin{equation}
\Delta^{jj}_k = -\frac{1}{\mathcal{L}}\sum_{k'}W^{jj}_{\text{ind}}(k,k')\braket {c_{j, k'}c_{j, -k'}}.
\label{eq.intrawirepairingpotentialdef}
\end{equation}
The intrawire interaction Hamiltonian for fermions in wire 1 and 2 can finally be written as:\footnote{Since the Hamiltonian is now quadratic, it is technically no longer an interaction Hamiltonian.}
\begin{equation}
H^\text{int}_{FF,jj} = \frac{1}{2}\sum_k \left[\Delta^{jj}_k c^\dagger_{j,k}c^\dagger_{j,-k} + \Delta^{jj *}_k c_{j,-k}c_{j,k} - \Delta^{jj}_k\braket{c^\dagger_{j,k}c^\dagger_{j,-k}} \right].
\label{eq.Hintintrawire.meanfield}
\end{equation}
Since the effective interaction is odd in its arguments, the intrawire pairing potentials is odd as well: $\Delta^{jj}_{-k} = -\Delta^{jj}_k$. Here it is in order to discuss the naming of $s$- and $p$-wave pairing. In the present thesis we will only use the naming to distinguish between pairings respectively even and odd in $k$. In a more general setup the naming stems from the fact, that the pairings come from a partial wave expansion, so that there are also a $d$-wave pairing, $f$-wave pairing etc. We will not go into any detail with this.

Now we turn to the \textit{inter}wire interaction. As for the intrawire case we assume, that only states with opposite momentum couples. This means, that we can truncate the sum in equation \eqref{eq.Hint12momentumspace} to $q = -k$ only. Further we make a mean field approximation with the mean field $\braket{c_{2,k}c_{1,-k}}$. Writing $p = k' - k$ we hereby get:
\begin{align}
H^\text{int}_{FF,12} = \frac{1}{\mathcal{L}} \sum_{k,k'} V_{\text{ind}}^{12}(k - k',0) & \left[\braket{c_{2,k'}c_{1,-k'}}c^\dagger_{1,-k}c^\dagger_{2,k} + \braket{c^\dagger_{1,-k'}c^\dagger_{2,k'}}c_{2,k}c_{1,-k} - \braket{c_{2,k'}c_{1,-k'}}\braket{c^\dagger_{1,-k}c^\dagger_{2,k}} \right]. \nonumber
\end{align}
We therefore define the interwire pairing potential as:
\begin{equation}
\Delta^{12}_k = -\frac{1}{\mathcal{L}} \sum_{k'} V_{\text{ind}}^{12}(k - k',0)\braket{c_{2,k'}c_{1,-k'}}.
\label{eq.interwirepairingpotentialdef}
\end{equation}
This brings the interaction part of the interwire Hamiltonian on the following form:\footnote{As for the single wire, this is technically no longer an interaction Hamiltonian, since it is only quadratic in the operators.}
\begin{equation}
H^\text{int}_{FF,12} = \sum_{k} \left[\Delta^{12}_k c^\dagger_{2,k}c^\dagger_{1,-k} + \Delta^{12 *}_k c_{1,-k}c_{2,k} - \Delta^{12}_k\braket{c^\dagger_{2,k}c^\dagger_{1,-k}} \right].
\label{eq.Hintinterwire.meanfield}
\end{equation}
Equations \ref{eq.Hintintrawire.meanfield} and \ref{eq.Hintinterwire.meanfield} are the essential equations for the diagonalisation of the Hamiltonian in section \ref{sec.HFFfull}. 

\section{Validity of the mean field approximation} \label{sec.meanfieldvalidity}
The Mermin-Wagner theorem in the context of superfluidity was first proved by P. C. Hohenberg in 1967 \cite{Hohenberg.MerminWagnertheorem}. It states, that there is no true long range order in one and two dimensions. We come with a simplified classical argument, that also illuminates how one might circumvent this discouraging fact. 

Let $\phi$ denote the phase of one of the pairing potentials, $\Delta^{ij}(x)$, in real space. For a truly ordered phase we expect $\phi$ to be constant. If we set this constant to 0, it therefore makes sense to infer a Hamiltonian in powers of $\phi$. The Hamiltonian is assumed not to depend on the phase itself. Therefore, to lowest order in $\phi$, the Hamiltonian associated with a spatially varying phase is: 
\begin{equation}
H[\phi] = \frac{K}{2}\int d^{D}x \; \left(\nabla \phi(x)\right)^2, 
\label{eq.Hphi.maintext}
\end{equation}
in $D$ spatial dimensions. There is no linear term $\nabla \phi$ present, because it vanishes under the integration. In the analysis we will as a start keep $x$ a dimensionless variable, so that there is an implicit length scale $\xi_s$. We return to this later on. $K$ is a parameter of unit energy. In appendix \ref{Appendix.longrangeorder.pairingphase} we show, that the phase-phase correlation function is then $\braket{ \text{e}^{i(\phi(x)- \phi(x'))} } = \text{e}^{G(x,x') - G(0)}$, where $G(x,x') = \braket{\phi(x)\phi(x')}$ is the two-point correlation function. The means are taken with respect to the underlying Gaussian distribution. Further, we show that:
\begin{equation}
G(x,x') - G(0) = -\frac{1}{\beta K}\int \frac{d^{D}k}{(2\pi)^D}\; \frac{1 - \text{e}^{ikx}}{k^2} = -\frac{|x - x'|}{2\beta K}, \nonumber 
\end{equation}
where the last equality is only valid in one dimension. We introduce the length scale $\xi_s$ by letting $x \to x / \xi_s$. Then the phase-phase correlation function is exponentially damped: 
\begin{equation}
\braket{ \text{e}^{i(\phi(x)- \phi(x'))} } = \text{e}^{G(x,x') - G(0)} = \text{e}^{-\frac{|x - x'|}{2\beta K \xi_s}}. 
\label{eq.phasecorrelationfunction.onedimension.maintext} 
\end{equation}
This expresses, that there is no true long range order in $D = 1$ dimensions. More specifically we can say, that the phase fluctuations associated with the Hamiltonian \eqref{eq.Hphi.maintext} destroys the long range order.\footnote{In one dimension it does so in an exponential way. In two dimension it decays with a power law. Only in three dimensions (and higher) does true long range order prevail.} We will later see, that the mean field theory predicts an energy gap in the spectrum. This gap and in turn the resulting superfluidity is in essense due to the spontaneous breaking of a continuous symmetry, namely the choosing of a specific phase of $\Delta$ \cite[pp. 341-346]{BruusFlensberg}. However, the above analysis shows, that if the exponential decay is on a microscopic scale no specific phase of $\Delta$ is actually chosen and hence no superfluidity. 

This sounds quite discouraging. However, the above analysis also proposes a loophole: what if $\xi_s$ is macroscopically large? Then the decay of the correlation function is, albeit exponential, on a macroscopic scale. One then talks of quasi long range order. There is one parameter we may suspect is directly linked to the length scale $\xi_s$. This is the range of the induced interaction; the BEC coherence length $\xi$. Hence, we speculate that $\xi_s \approx \xi$. We may therefore be able to reach a macroscopic length scale by making the coherence length sufficiently large. This is in part supported by recent work by Ortiz and Cobanero. They investigate an exactly solvable model in one dimension, which has a real space interaction that drops off as $1 / x$, $x$ being the interparticle distance \cite{Ortiz.Beyondmeanfieldtheory}. Hence, it is closely related to our system in the $\xi \to \infty$ limit. Ortiz and Cobanero show, that their model supports a gap opening, and hence the system must be ordered on a macroscopic scale. In another article Ortiz investigates an analogous two-dimensional system, which is also exactly solvable \cite{Ortiz.pxpy}. Here they find, that the mean field results fit beautifully with their exact solution.

We may therefore speculate, that since the interaction in the present context can also be made long range by going to large coherence lengths, the superfluid phase is not destroyed in one dimension. Further, we may speculate that the analysis based on the mean field approximation will be \textit{qualitatively} correct in the long coherence length limit, $k_F\xi \gg 1$. We can also argue for this in an intuitive way. In the long range limit, the fermions interact with a macroscopic number of particles. Hence, going to lower dimensions does not necessarily mean, that the fermions also have fewer neighbours. We simply have to enhance the interaction range, $\xi$. Since the model in the present context is not exactly solvable\footnote{At least not to the authors knowledge.}, the only reasonable way to investigate these speculations is to investigate the fluctuations of the order parameters $\braket{c_{i,-k}c_{j,k}}$. Such a consideration leads to self-consistent result, except for a macroscopically narrow window of temperaturs, $\Delta T$, around the critical temperature, $T_c$. However, this does not reflect the full fluctuations in the order parameters considered in this subsection. It therefore remains an open question whether the approach is applicable. We consider it a possible topic of future work. 

In the long coherence length limit, $k_F\xi \gg 1$, we can no longer ignore retardation effects, because the Bogoliubov excitations in the BEC are slowed down: $\frac{v_F}{c_0} = \frac{m_B}{m_F}\frac{k_F\xi}{\sqrt{2}} > 1$. However, as mentioned in the end of chapter \ref{Chapter3}, the induced interaction for nonzero Matsubara frequencies is qualitatively the same as the zero frequency component\footnote{See appendix \ref{Appendix.inducedinteraction.realspace} for the derivation of the general real space interaction.}. Hence, including these effects does not make a qualitative difference.

It is also possible, that the work we precent is qualitatively sound, eventhough the energy gap might close. Flensberg et al. have shown, how one can address pairing of fermions in number conserving systems in one dimension using the Luttinger liquid formalism \cite{Flensberg.numberconserving1Dfermions}. In the system they consider, they find long-range order of a specific phase, eventhough the superfluid phase has no such long range order. This work has been extended by Kane et al. and Ruhman and Altman in two recent articles \cite{Kane.Pairing.Luttingerliquids, Altman.Pairing.spinlessfermions}. For the systems they look at, they find a gapless weak coupling and gapped strong coupling phase. They conclude, that these are in a one-one correspondence to the weak and strong coupling phases of systems without number conservation. Specifically, there is a topological phase transition between the two phases. It is exactly this behaviour we will find using the mean field approach.

There are mainly two reasons, why we would like to be able to use the mean field approach. Firstly, this approximation approach is by far the simpler one. Secondly, the topological classification dubbed the tenfold way is developed for band Hamiltonians \cite{Ryu.Topology}. There is no fully developed classification scheme for interaction Hamiltonians. We will continue on with the mean field approach encouraged by the result of Ortiz and Cobanero and by the large amount of work done in one dimensional superfluid systems, see e.g. articles \cite{Alicea, KitaevTopPhases, KitaevQuantumWires, LiYangChen, FuKane2006, GreiterIsingKitaevChain, DeGottardiMajoranaFermions, BudichTopInvMajoranaWires, ZhangWu}. 

\section{Grand Hamiltonian} \label{sec.HFFfull}
The free particle Hamiltonian for the fermions is: $H_{0,1} + H_{0,2} = \sum_{j,k}\frac{k^2}{2m_F}c^\dagger_{j,k}c_{j,k}$. Further, the Hamiltonian is not particle number conserving, since it contains terms like $c^\dagger c^\dagger$. In stead we impose diffusive equilibrium by subtracting $\mu_1N_{1,F}+\mu_2N_{2,F}$. $\mu_j$ is the chemical potential of wire $j$ and $N_{j,F} = \sum_k c^\dagger_{j,k}c_{j,k}$ is the number of $j$ fermions. We hereby get the following grand Hamiltonian:
\begin{align}
H_{FF} = &\sum_j \left[H_{0,j} - \mu_j N_{j,F}\right] + H^\text{int}_{FF,11} + H^\text{int}_{FF,22} + H^\text{int}_{FF,12} \nonumber \\
       = &\frac{1}{2}\sum_k C^\dagger_k \mathcal{H}_{FF,k}C_k + \frac{1}{2}\sum_k\left[\varepsilon_{1,k} + \varepsilon_{2,k} - \Delta^{11}_k\braket{c^\dagger_{1,k}c^\dagger_{1,-k}} - \Delta^{22}_k\braket{c^\dagger_{2,k}c^\dagger_{2,-k}} - 2\Delta^{12}_k\braket{c^\dagger_{2,k}c^\dagger_{1,-k}} \right], \nonumber
\end{align}

with:
\begin{equation}
\mathcal{H}_{FF,k} = \begin{bmatrix} \varepsilon_{1,k} & \Delta^{11}_k      & 0                 & -\Delta^{12}_{-k} \\ 
                                     \Delta^{11 *}_k   & -\varepsilon_{1,k} & \Delta^{12*}_k    & 0 \\ 
                                    0                  & \Delta^{12}_k      & \varepsilon_{2,k} & \Delta^{22}_k \\ 
                                     -\Delta^{12*}_{-k}& 0                  & \Delta^{22*}_k    & -\varepsilon_{2,k} \end{bmatrix}, \hspace{0.5cm}
C_k =  \begin{bmatrix} c_{1,k} \\ c^\dagger_{1,-k} \\ c_{2,k} \\ c^\dagger_{2,-k} \end{bmatrix}.                                     
\end{equation}
Here $\varepsilon_{j,k} = \frac{k^2}{2m_F}-\mu_j$ is the kinetic energy relative to the chemical potential of the $j$ fermions. This has a quite general structure. To simplify matters we firstly assume, that the wires are held at the same chemical potential $\mu$. Hence we let $\varepsilon_{j,k} \to \varepsilon_k$. Secondly, we make two separate gauge transformations to make the intrawire pairings, $\Delta^{jj}_k$, real. This works in the following fashion. We assume, that the phase of the intrawire pairings are global, i.e. not dependent on the $k$. Hence, we let $\Delta^{jj}_k \to \Delta^{jj}_k \text{e}^{i\phi^{jj}}$, with $\text{e}^{i\phi^{jj}}$ the phase of the pairing. We then make the gauge transformation $c_{j,k} \to \text{e}^{-i\phi^{jj}/2}c_{j,k}$. From equation \eqref{eq.intrawirepairingpotentialdef} the intrawire pairing is linearly dependent on $\braket{c_{j,k}c_{j,-k}}$. Hence, under this gauge transformation we get:
\begin{equation}
\Delta^{jj}_k \text{e}^{i\phi^{jj}} \to \text{e}^{-i\phi^{jj}}\Delta^{jj}_k \text{e}^{i\phi^{jj}} = \Delta^{jj}_k, \nonumber
\end{equation}
whereby the intrawire pairings are real. Since the chemical potentials of the two wires is now the same, the two wires are equivalent. Hence, the system is symmetric in the 1 and 2 fermions. This means, that the pairings must be equal up to an overall phase: $\Delta^{22}_k = \text{e}^{i\phi} \Delta^{11}_k$. Since they are now real we can obtain the relation: $\Delta^{22}_k = -\Delta^{11}_k$. The overall sign difference is chosen, since it makes the eigenvectors to $\mathcal{H}_{FF,k}$ a bit simpler. The interwire pairing $\Delta^{12}_k$ describes a pairing between distinguishable particles. If we think of the wires as indexed with a pseudospin, the pairing is expected to be a $s$-wave type pairing. We will therefore search for an even solution for $\Delta^{12}_k$. This means, that:
\begin{equation}
\mathcal{H}_{FF,k} = \begin{bmatrix} \varepsilon_{k}   & \Delta^{11}_k      & 0                 & -\Delta^{12}_{k} \\ 
                                     \Delta^{11}_k     & -\varepsilon_{k}   & \Delta^{12*}_k    & 0 \\ 
                                    0                  & \Delta^{12}_k      & \varepsilon_{k}   & -\Delta^{11}_k \\ 
                                     -\Delta^{12*}_{k} & 0                  & -\Delta^{11}_k     & -\varepsilon_{k} \end{bmatrix}.                  
\end{equation}
We have hereby specified two global phases. Since this is the total gauge freedom of the Hamiltonian, this means that the phase of $\Delta^{12}_k$ must be held completely general for the time being. We notice, that for $\Delta^{12}_{k} = 0$, the Hamiltonian consists of two independent blocks of Kitaev Hamiltonians, as is evident from comparing the above to equation \ref{eq.HKitaevpre}. This means, that the above Hamiltonian describes interacting Kitaev wires. The eigenvalues of $\mathcal{H}_{FF,k}$ above come in plus/minus pairs. The norm of the eigenvalues give the energy dispersion. The result is:
\begin{equation}
E^{\pm}_{F,k} = \sqrt{\varepsilon^2_k + \left(\Delta^{11}_k\right)^2 + \left|\Delta^{12}_k\right|^2 \pm \Delta^{11}_k(\Delta^{12}_k + \Delta^{12*}_k)}. 
\end{equation} 
This shows, that the excitation energies depends on the phase of the interwire pairing, $\Delta^{12}_k$. Since $\Delta^{11}_k$ is odd in $k$ and $\Delta^{12}_k$ is even in $k$, we get that $E^{+}_{F,k} = E^{-}_{F,-k}$. Notice, that due to the presence of $\pm \Delta^{11}_k(\Delta^{12}_k + \Delta^{12*}_k)$ in $E^{\pm}_{F,k}$, the dispersions are neither even nor odd in $k$. If one is bothered by this, it is possible to redefine the eigenvalues to $\bar{E}^{\pm}_{F,k} = \sqrt{\varepsilon^2_k + \left(\Delta^{11}_k\right)^2 + \left|\Delta^{12}_k\right|^2 \pm |\Delta^{11}_k(\Delta^{12}_k + \Delta^{12*}_k)|}$, which are manifestly even in $k$. This is a basic reshuffling of the eigenvalues. The eigenvectors are however much more involved, and in turn the derivation of the gap equations is more cumbersome. The result in the end is the same however, and therefore we will stick to the above energy eigenvalues.  

The new quasiparticle operators are found by finding the eigenvectors to $\mathcal{H}_{FF,k}$. We define the quasiparticle fermionic operators $\gamma_{j,k}$ by:
\begin{equation}
\begin{bmatrix} c_{1,k} \\ c^\dagger_{1,-k} \\ c_{2,k} \\ c^\dagger_{2,-k} \end{bmatrix} = U_{F,k}\begin{bmatrix} \gamma_{1,k} \\ \gamma^{\dagger}_{1,-k} \\ \gamma_{2,k} \\ \gamma^{\dagger}_{2,-k} \end{bmatrix}.
\label{eq.zetaoperatorstwowiresdefinition}
\end{equation} 
The $\gamma$-operators have to obey the same anticommutation relations as the $c$-operators. Specifically all anticommutators between 1 and 2 quasiparticles, like $\{\gamma_1, \gamma_2 \} $, vanishes. $\mathcal{H}_{FF,k}$ should then be diagonalised to yield:
\begin{equation}
U^\dagger_{FF,k}\mathcal{H}_{FF,k}U_{FF,k} = \begin{bmatrix} 
E^{-}_{F,k} & 0        & 0       & 0        \\ 
0       & -E^{-}_{F,-k} & 0       & 0        \\ 
0       & 0        & E^{+}_{F,k} & 0        \\ 
0       & 0        & 0       & -E^{+}_{F,-k} \\ 
\end{bmatrix} = \begin{bmatrix} 
E^{-}_{F,k} & 0        & 0       & 0        \\ 
0       & -E^{+}_{F,k} & 0       & 0        \\ 
0       & 0        & E^{+}_{F,k} & 0        \\ 
0       & 0        & 0       & -E^{-}_{F,k} \\ 
\end{bmatrix} \nonumber
\end{equation}
Further the diagonalisation must respect the symmetry in the 1 and 2 fermions of the Hamiltonian. Specifically the assumption of $\Delta^{22}_k = -\Delta^{11}_k$ must be selfconsistent. From equation \ref{eq.intrawirepairingpotentialdef} this means, that $\braket{c_{2,k}c_{2,-k}} = -\braket{c_{1,k}c_{1,-k}}$. Finally, since the elements of $C_k$ are not independent, we get some internal structure of $U_{F,k}$. Let us write the elements of $U_{F,k}$ as $u^{ij}_k$. From equation \eqref{eq.zetaoperatorstwowiresdefinition} we take the first row and the conjugate of the second row with $-k \to k$:
\begin{align}
c_{1,k} &= u^{11}_k \gamma_{1,k} + u^{12}_k \gamma^\dagger_{1,-k} + u^{13}_k \gamma_{2,k} + u^{14}_k \gamma^\dagger_{2,-k}, \nonumber \\
c_{1,k} &= u^{22*}_{-k} \gamma_{1,k} + u^{21*}_{-k} \gamma^\dagger_{1,-k} + u^{24*}_{-k} \gamma_{2,k} + u^{23*}_{-k} \gamma^\dagger_{2,-k}. \nonumber
\end{align}
Since the coefficients in front of e.g. $\gamma_{1,k}$ must be the same in both expressions, we get that $u^{11}_k = u^{22*}_{-k}$. This means, that $U_{F,k}$ has the following structure due to the built-in particle-hole symmetry:
\begin{equation}
U_{F,k} = \begin{bmatrix} 
u^{22*}_{-k} & u^{21*}_{-k} & u^{24*}_{-k} & u^{23*}_{-k}           \\  
u^{21}_k 	 & u^{22}_k 	& u^{23}_k 	   & u^{24}_k               \\ 
u^{42*}_{-k} & u^{41*}_{-k} & u^{44*}_{-k} & u^{43*}_{-k}           \\ 
u^{41}_k 	 & u^{42}_k 	& u^{43}_k 	   & u^{44}_k
\end{bmatrix}. \nonumber
\end{equation}
Now define the norms $A^{\pm}_k = 2 \sqrt{ E^{\pm}_{F,k}(\varepsilon_k + E^{\pm}_{F,k}) }$. With a bit of trial and error the above requirements combined result in:
\begin{equation}
U_{F,k} = \begin{bmatrix} 
\frac{\varepsilon_k + E^{-}_{F,k}}{A^{-}_k}    & -\frac{\Delta^{11}_k + \Delta^{12}_k}{A^{+}_k} & \frac{\varepsilon_k + E^{+}_{F,k}}{A^{+}_k}     & -\frac{\Delta^{11}_k - \Delta^{12}_k}{A^{-}_k}  \\  
\frac{\Delta^{11}_k - \Delta^{12*}_k}{A^{-}_k} & \frac{\varepsilon_k + E^{+}_{F,k}}{A^{+}_k}    & \frac{\Delta^{11}_k + \Delta^{12*}_k}{A^{+}_k}  & \frac{\varepsilon_k + E^{-}_{F,k}}{A^{-}_k}     \\ 
-\frac{\varepsilon_k + E^{-}_{F,k}}{A^{-}_k}   & -\frac{\Delta^{11}_k + \Delta^{12}_k}{A^{+}_k} & \frac{\varepsilon_k + E^{+}_{F,k}}{A^{+}_k}     & \frac{\Delta^{11}_k - \Delta^{12}_k}{A^{-}_k} \\ 
\frac{\Delta^{11}_k - \Delta^{12*}_k}{A^{-}_k} & -\frac{\varepsilon_k + E^{+}_{F,k}}{A^{+}_k}   & -\frac{\Delta^{11}_k + \Delta^{12*}_k}{A^{+}_k} & \frac{\varepsilon_k + E^{-}_{F,k}}{A^{-}_k} 
\end{bmatrix}. \nonumber
\end{equation}
The diagonalisation of the Hamiltonian is now straight forward. $U^\dagger_{F,k}\mathcal{H}_{FF,k}U_{F,k}$ is diagonal with the eigenvalues $+E^{\pm}_{F,k}$ and $-E^{\pm}_{F,k}$ alternating in the diagonal as shown above. The result of the diagonalisation is therefore:
\begin{align}
H_{FF} &= E_0 + \sum_{k} \left[ E^{-}_{F,k}\gamma^\dagger_{1, k}\gamma_{1, k} + E^{+}_{F,k}\gamma^\dagger_{2, k}\gamma_{2, k} \right], \nonumber \\ 
E_0 &= \frac{1}{2}\sum_k \left[2\varepsilon_k - \left( E^{-}_{F,k} + E^{+}_{F,k} + \Delta^{11}_k\braket{c^\dagger_{1,k}c^\dagger_{1,-k}} + \Delta^{22}_k\braket{c^\dagger_{2,k}c^\dagger_{2,-k}} + 2\Delta^{12}_k\braket{c^\dagger_{2,k}c^\dagger_{1,-k}} \right) \right]. 
\label{eq.2wiresDiagonalisedHamiltonian}
\end{align}  
From this we see, that the ground state of the system at $T = 0$ is defined by having no quasiparticles present: $\gamma_{j,k} \ket{\text{S}}_0 = 0$ for all $k$.\footnote{S is short for \text{S}uperfluid ground state.} This means, that we can write the ground state as:
\begin{equation}
\ket{\text{S}}_0 = \prod_{j,k > 0} \gamma_{j,-k}\gamma_{j,k}\ket{0},
\label{eq.superfluidgroundstate}
\end{equation}
If one inserts the transformation to the regular fermionic $c$-operators, terms like $c^\dagger_{1,-k}c^\dagger_{1,k}$ appear. This is a manifestation of the concept of Cooper pairs: in the ground state all the fermions are in pairs with opposite momenta. The excited states hereby consists of breaking up a number of Coopers pairs.   

From the diagonalisation we can also find the distribution of the quasiparticles, $\braket{\gamma^\dagger_{j,k}\gamma_{j,k}}$. Since the presence of such a particle is a fermionic excitation with energy $E^{\pm}_{F,k}$ it is intuitively clear, that they are distributed according to the Fermi-Dirac distribution given by:
\begin{equation}
\braket{\gamma^\dagger_{j,k} \gamma_{j,k}} = f(E^{\pm}_{F,k}) = \frac{1}{\text{e}^{\beta E^{\pm}_{F,k}} + 1}, 
\label{eq.Distribution.quasiparticles}
\end{equation}
where the minus is for $j = 1$, the plus for $j = 2$. Notice the absence of a quasiparticle chemical potential. This is because the number of quasiparticles is not fixed. One can show this with a standard statistical mechanics argument, see e.g. \cite[p. 225]{SchroederThermal}. For a rigorous argument based on the second quantized operators, the reader is referred to appendix \ref{Appendix.distribution.quasiparticles}. Because of the absence of a quasiparticle chemical potential the distribution function differs functionally from the distribution of the fermions in the free gas. In the free gas for low temperatures, the distribution is close to one for $|k|< \sqrt{2m_F\mu}$ and quickly decays to zero outside this region. The present quasiparticle distribution is exactly zero for $T = 0$ and exhibit well-defined peaks around $k = \pm \sqrt{2m_F\mu}$, showing that the excitations are near the Fermi points of the free gas. 

The expected temperature dependency of the system is inferred from the original BCS-theory \cite[chapter 3]{Tinkham}. The pairing potentials $\Delta^{ij}_k(T)$ is expected to have their maximum for $T = 0$ and then monotonically decrease to 0, when the temperature is increased to a critical temperature $T_c$. This, we will see, defines a phase transition from a superfluid phase below $T_c$, as defined in section \ref{sec.Superfluidity}, to the normal phase above $T_c$. 

In terms of Landau theory of phase transitions as reviewed in section \ref{sec.landauphasetransitions} the order parameters are the quantities, that are zero in the normal phase above the critical temperature, $T_c$, and nonzero below. From this it is clear, that $\braket{c_{i,-k}c_{j,k}}$, or equivalently $\Delta^{ij}_k$, are exactly the order parameters of the system.

\section{Fluctuations in the number of fermions} \label{sec.fluctuation.fermionnumber}
In this section we discuss the variance of the fermion particle number in the $T = 0$ limit. 

As noted in the derivation of the mean field Hamiltonian, the resulting Hamiltonian does not conserve the number of fermions: $[N_{j,F}, H_{FF}] \neq 0 $, where $N_{j,F} = \sum_k c^\dagger_{j,k} c_{j,k}$ is the number operator for fermions in wire $j$. As a consequence there will be a variance $\braket{(N_{j,F}-\braket{N_{j,F}})^2} \neq 0$. Calculating the variance gives a neat demonstration of how one can utilize Wick's theorem. To keep the notation simple we do the calculation for the fermions in wire $1$ and suppress the $1$ subscript in $N_{1,F}$. We get:
\begin{equation}
\braket{(N_F - {\braket{N_F}})^2} = \braket{N_F^2} - \braket{N_F}^2 = \sum_{k,q} \braket{c^\dagger_{1,k}c_{1,k}c^\dagger_{1,q}c_{1,q}} - \braket{c^\dagger_{1,k}c_{1,k}}\braket{c^\dagger_{1,q}c_{1,q}}. \nonumber
\end{equation} 
The main challenge is thus to compute $\braket{c^\dagger_{1,k}c_{1,k}c^\dagger_{1,q}c_{1,q}}$. Since the mean field Hamiltonian is quadratic, we can use Wick's theorem to reduce this four-body mean to a sum of two-body means:
\begin{equation}
\braket{c^\dagger_{1,k}c_{1,k}c^\dagger_{1,q}c_{1,q}} = \braket{c^\dagger_{1,k}c_{1,k}}\braket{c^\dagger_{1,q}c_{1,q}} - \braket{c^\dagger_{1,k}c^\dagger_{1,q}}\braket{c_{1,k}c_{1,q}} + \braket{c^\dagger_{1,k}c_{1,q}}\braket{c_{1,k}c^\dagger_{1,q}}. \nonumber
\end{equation}
The sign $(-1)^{n}$ of the terms are thus given by the number times, $n$, we have to exchange two operators next to each other to get to the term in question \cite[pp. 198-202]{BruusFlensberg}. The first term we recognise as the part coming from $\braket{N_F}^2$. Hence, we only need to calculate the two latter terms. By changing to the quasiparticle $\gamma$-operators we can evaluate these directly. The calculation is rather lengthy and not very illuminating. It is therefore skipped. In the zero temperature limit we get:
\begin{equation}
\braket{(N_F - {\braket{N_F}})^2} = \frac{1}{4}\sum_k\left[ \frac{|\Delta^{11}_k + \Delta^{12}_k|^2}{2(E^{+}_{F,k})^2} + \frac{1}{2}\left(E^{-}_{F,k}E^{+}_{F,k} - \left(\varepsilon_k^2 + (\Delta^{11}_k)^2 + |\Delta^{12}_k|^2\right) \right) \right].
\end{equation}
Firstly, when both inter- and intrawire pairings are zero both terms in the sum vanish, which means the variance is zero. This illustrates the fact, that it is the presence of the pairings, that makes the variance nonzero. Further, we notice that the last term is only present when $E^{-}_{F,k} \neq E^{+}_{F,k}$, which means when $\Delta^{12}_k$ is \textit{not} purely imaginary. For $\Delta^{11}_k = 0$ we recover the standard variance result of the original BCS theory for $s$-wave pairing \cite[pp. 50-52]{Tinkham}. Finally, converting the sum to an integral in the thermodynamic limit gives us the relative variance:
\begin{equation}
\frac{\braket{(N_F - \braket{N_F})^2}}{\braket{N_F}^2} = \frac{1}{8\braket{N_F}}\int d\tilde{k} \; \left[ \frac{|\Delta^{11}_{\tilde{k}} + \Delta^{12}_{\tilde{k}}|^2}{2(E^{+}_{F,\tilde{k}})^2} + \frac{1}{2}\left(E^{-}_{F,\tilde{k}}E^{+}_{F,\tilde{k}} - \left(\varepsilon_{\tilde{k}}^2 + (\Delta^{11}_{\tilde{k}})^2 + |\Delta^{12}_{\tilde{k}}|^2\right) \right) \right]. \nonumber
\end{equation}
Here we write $k = k_F \tilde{k}$. We use, that the spacing in momentum space is $\Delta \tilde{k} = \frac{2}{\braket{N_F}}$. This illustrates an important aspect, namely that the relative variance $\braket{(N_F - \braket{N_F})^2}/\braket{N_F}^2$ decreases as $1/\braket{N_F}$. Hence, the number of fermions becomes increasingly ill-defined, but it does so in a slow way, namely so that the relative variance approaches zero in the thermodynamic limit. This is all completely analogous with the standard BCS theory for $s$-wave pairing \cite[pp. 50-52]{Tinkham}. In this sense we consider the number of fermions not conserved but well-defined in the thermodynamic limit. We will from now on therefore use $N_F$ both for the number operator and the mean value $\braket{N_F}$. It should be clear from the context which one is meant.

\section{Gap and number equations} \label{sec.gapandnumberequations}
In this section we find self-consistency equations for the pairing potentials $\Delta^{ij}_k$, known as gap equations. Inspecting the definitions in equations \eqref{eq.intrawirepairingpotentialdef} and \eqref{eq.interwirepairingpotentialdef} we see, that we need to calculate the mean fields $\braket{ c_{i, k} c_{j,-k} }$. We will do this in detail for $\braket{c_{1,k}c_{1,-k}}$ to see how the computation plays out. 

We write $c_{1,k} = u^{11}_{k} \gamma_{1,k} + u^{12}_{k} \gamma^\dagger_{1,-k} + u^{13}_{k} \gamma_{2,k} + u^{14}_{k} \gamma^\dagger_{2,-k}$, with $u^{ij}_k$ the elements of $U_{F,k}$. Then:
\begin{equation}
\braket{c_{1,k}c_{1,-k}} = u^{11}_{k}u^{12}_{-k}\braket{\gamma_{1,k}\gamma^\dagger_{1,k}} + u^{12}_{k}u^{11}_{-k}\braket{\gamma^\dagger_{1,-k}\gamma_{1,-k}} + u^{13}_{k}u^{14}_{-k}\braket{\gamma_{2,k}\gamma^\dagger_{2,k}} + u^{14}_{k}u^{13}_{-k}\braket{\gamma^\dagger_{2,k}\gamma_{2,k}}, \nonumber
\end{equation}
where we use that the only nonzero thermal averages are on the form $\braket{\gamma_{j,k}\gamma^\dagger_{j,k}}$ or $\braket{\gamma^\dagger_{j,k}\gamma_{j,k}}$. Using that the quasiparticles are distributed according to the Fermi-Dirac distribution, equation \eqref{eq.Distribution.quasiparticles}, we get:
\begin{equation}
\braket{c_{1,k}c_{1,-k}} = \frac{\Delta^{11}_k + \Delta^{12}_k}{4E^{+}_{F,k}}\left(1 - 2f(E^{+}_{F,k}\right) + \frac{\Delta^{11}_k - \Delta^{12}_k}{4E^{-}_{F,k}}\left(1 - 2f(E^{-}_{F,k}\right). 
\label{eq.meanfield11}
\end{equation}
Calculating $\braket{c_{2,k}c_{2,-k}}$ gives exactly minus the above, so that the assumption $\braket{c_{2,k}c_{2,-k}} = -\braket{c_{1,k}c_{1,-k}}$ is self-consistent. An analogous calculation for $\braket{c_{2,k}c_{1,-k}}$ yields:
\begin{equation}
\braket{c_{2,k}c_{1,-k}} = \frac{\Delta^{12}_k + \Delta^{11}_k}{4E^{+}_{F,k}}\left(1 - 2f(E^{+}_{F,k}\right) + \frac{\Delta^{12}_k - \Delta^{11}_k}{4E^{-}_{F,k}}\left(1 - 2f(E^{-}_{F,k}\right). 
\label{eq.meanfield12}
\end{equation}
Plugging these mean fields into the definition of the pairing potentials, equations \eqref{eq.intrawirepairingpotentialdef} and \eqref{eq.interwirepairingpotentialdef}, we get the gap equations:
\begin{align}
\Delta^{11}_k &= -\frac{1}{\mathcal{L}}\sum_{k'} W_{\text{ind}}^{11}(k, k')\frac{\Delta^{11}_{k'} + \Delta^{12}_{k'}}{2E^{+}_{F,k'}}\tanh\left[\frac{\beta E^{+}_{F,k'}}{2}\right], \nonumber \\
\Delta^{12}_k &= -\frac{1}{\mathcal{L}}\sum_{k'} W_{\text{ind}}^{12}(k, k')\frac{\Delta^{12}_{k'} + \Delta^{11}_{k'}}{2E^{+}_{F,k'}}\tanh\left[\frac{\beta E^{+}_{F,k'}}{2}\right].
\label{eq.2wiresgapequations}
\end{align}
Here the oddness of $\Delta^{11}_k$ and evenness of $\Delta^{12}_k$ have been used to simplify the expressions. Further, we have used that $1 - 2f(E) = \tanh\left[\frac{\beta E}{2}\right]$ with inverse temperature $\beta = 1 / k_BT$. We have defined the \textit{inter}wire effective interaction $W_{\text{ind}}^{12}(k, k')$ to make the equations symmetric:
\begin{equation}
W_{\text{ind}}^{12}(k, k') = \frac{1}{2}\left[V_{\text{ind}}^{12}(k - k', 0) + V_{\text{ind}}^{12}(k + k', 0) \right].
\label{eq.EffectiveInteraction.interwire}
\end{equation}
This definition is completely analogous to the definition of the intrawire effective interaction, see equation \eqref{eq.EffectiveInteraction.intrawire}. 

Now we wish to calculate the number equation $N_F = \sum_k \braket{c^\dagger_{j,k}c_{j,k}}$, hence we need the thermal average of $\braket{c^\dagger_{j,k}c_{j,k}}$. The calculation is carried through in complete analogy to the above, resulting in:
\begin{equation}
\braket{c^\dagger_{j,k}c_{j,k}} = \frac{1}{4}\left( 1 - \frac{\varepsilon_k}{E^{+}_{F,k}}\tanh\left[\frac{\beta E^{+}_{F,k}}{2}\right] \right) + \frac{1}{4}\left( 1 - \frac{\varepsilon_k}{E^{-}_{F,k}}\tanh\left[\frac{\beta E^{-}_{F,k}}{2}\right] \right).
\label{eq.meanoccupancy}
\end{equation}
This is the mean occupancy of the $k$'th momentum state in wire $j$. It is seen to be independent of the wire. Plugging it into the number equation and transforming the sum to an integral gives us the number equation:
\begin{equation}
n_F = \frac{1}{2}\int \frac{dk}{2\pi} \left( 1 - \frac{\varepsilon_k}{E^{+}_{F,k}}\tanh\left[\frac{\beta E^{+}_{F,k}}{2}\right] \right). 
\label{eq.2wiresnumberequation}
\end{equation}
The real part of the first gap equation, the real and imaginary part of the second one and the number equation gives us four equations for the four quantitites $\Delta^{11}_k, \Delta^{12}_{k,r}, \Delta^{12}_{k,i} $ and $\mu$. Here we write the interwire pairing as decomposed in its real and imaginary parts: $\Delta^{12}_k = \Delta^{12}_{k,r} + i\Delta^{12}_{k,i}$. Finally, taking the imaginary part of the first gap equation gives a fifth self-consistency equation:
\begin{equation}
0 = -\frac{1}{\mathcal{L}}\sum_{k'} W_{\text{ind}}^{11}(k, k')\frac{\Delta^{12}_{k',i}}{2E^{+}_{F,k'}}\tanh\left[\frac{\beta E^{+}_{F,k'}}{2}\right].
\end{equation}
This equation has two trivial solutions. The first is, where the interwire pairing is real. Then the integrand is 0 and the equation is fulfilled. The second is, where the interwire pairing is purely imaginary. Then the two dispersion relations are identical: $E^{\pm}_{F,k} = \sqrt{\varepsilon_k^2 + (\Delta^{11}_k)^2 + |\Delta^{12}_k|^2}$. The summand is hereby an odd function in $k'$ and so the equation is again fulfilled. We will only investigate these two cases, because they exhibit interesting symmetries of the system. This is discussed chapter \ref{Chapter5}. Before turning to these symmetries we calculate the ground state energy of the system. 

\section{Ground state energy} \label{sec.groundstateenergy}
The ground state \textit{grand} energy, $E_0$, is derived similarly to the gap equations of the previous section. For $T = 0$ the expression for the ground state grand energy is particularly simple. We get:
\begin{equation}
\frac{E_0}{\epsilon_{F,0} N_F} = -\frac{1}{4} \int d\tilde{k} \frac{(\tilde{\varepsilon}_{\tilde{k}} - \tilde{E}^{+}_{F, \tilde{k}})^2}{\tilde{E}^{+}_{F, \tilde{k}}}, \hspace{0.5cm} T = 0. 
\label{eq.2wiresGrandGroundStateEnergy}
\end{equation}
We have expressed the momentum as $\tilde{k} = k/k_F$, and the energies as $\tilde{E} = E/\epsilon_{F,0}$. We notice, that the ground state grand energy is negative. This is because, we are measuring energies with respect to the chemical potential. The motivation for calculating the grand energy is the following. For any temperature the grand energy, $\Phi$, is given by $\Phi = U - TS - 2\mu N_F = F - 2\mu N_F$, where $F$ is the Helmholtz free energy \cite[pp. 161-162]{SchroederThermal}. The presence of the factor of 2 is simply because there are $N_F$ fermions in \textit{each} wire. Physically, we hold the number of particles fixed. This means, that it is the Helmholtz free energy $F = \Phi + 2\mu N_F$ we have to minimize to find the preferred state, not $\Phi$. For $T = 0$, this means, that the energetically favourable state is the one, that exhibits a minimum of $F(T = 0) = E_0 + 2\mu N_F$. The aim is therefore to calculate the Helmholtz free energy as a function of the interwire distance, $d$, for the two interesting cases: $\Delta^{12}_k$ real and imaginary. The one that exhibits the minimal energy is the energetically favourable state. 


