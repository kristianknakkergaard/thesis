% Chapter 3

\chapter{The one dimensional Fermi Hamiltonian} % Main chapter title

\label{Chapter4} % For referencing the chapter elsewhere, use \ref{Chapter3} 

\lhead{Chapter 4. \emph{1D Fermi Hamiltonian}} % This is for the header on each page - perhaps a shortened title

%----------------------------------------------------------------------------------------
In this chapter we will finally arrive at the promised Kitaev Hamiltonian promised in chapter \ref{Chapter2}. We start by calculating the pair interaction Hamiltonian for the fermions in section \ref{sec.HFFint}. This is then inserted into the full Hamiltonian in section \ref{sec.HFFfull}.

\section{The fermion interacting Hamiltonian} \label{sec.HFFint}

We start out in real space. The interaction Hamiltonian for pair interactions is given by:
\begin{equation}
H^\text{int}_{FF} = \int dxdx' \hat{\psi}^\dagger_F(x)\hat{\psi}^\dagger_F(x')V^\text{ind}_{FF}(x'-x) \hat{\psi}_F(x') \hat{\psi}_F(x).
\label{eq.HFFintdef}
\end{equation}
Here the induced interaction in real space is achieved by a Fourier transform: $V^\text{ind}_{FF}(x'-x) = \int dq \text{e}^{-iq(x'-x)} V^\text{ind}_{FF}(q)$, with $V^\text{ind}_{FF}(q)$ written in equation \eqref{eq.VFFq} above. Further we make a mean field approximation to get the Hamiltonian on a solvable quadratic form. We write
\begin{equation}
\hat{\psi}_F(x') \hat{\psi}_F(x) = \langle \hat{\psi}_F(x') \hat{\psi}_F(x) \rangle + \left(\hat{\psi}_F(x') \hat{\psi}_F(x)-\langle \hat{\psi}_F(x') \hat{\psi}_F(x) \rangle \right) = \langle \hat{\psi}_F(x') \hat{\psi}_F(x) \rangle + A_F(x',x)
\end{equation}
and treat the last part $A_F$ as a small quantity. The meaning of this will become clearer later on. The mean in the end has to be taken with respect to the specific state of the system, we will also see how this plays out later on. We now only keep terms of $A_F$ and $A^\dagger_F$ up to first order, so that we get the following three terms for the interaction Hamiltonian:\footnote{Formally we are decoupling the Hamiltonian, so that it so no longer really describing any interaction.}
\begin{align}
H^\text{int}_{FF} = &\int dxdx' \langle \hat{\psi}^\dagger_F(x) \hat{\psi}^\dagger_F(x') \rangle V^\text{ind}_{FF}(x'-x) \langle \hat{\psi}_F(x') \hat{\psi}_F(x) \rangle + \nonumber \\
&\int dxdx' A^\dagger_F(x,x') V^\text{ind}_{FF}(x'-x) \langle \hat{\psi}_F(x') \hat{\psi}_F(x) \rangle + \nonumber \\
&\int dxdx' \langle \hat{\psi}^\dagger_F(x) \hat{\psi}^\dagger_F(x') \rangle V^\text{ind}_{FF}(x'-x) A_F(x',x). \nonumber
\end{align} 
To get back to momentum space we expand the field operators in terms of plane waves: $\hat{\psi}_F(x) = \frac{1}{\sqrt{\mathcal{L}}}\sum_k \text{e}^{ikx} f_k$, where as usual $f_k$ is the fermion annihilation operator. Further we assume, that only states belonging to opposite momenta couples.\footnote{At least they are assumed to be absolutely dominant.} This means, that $f_kf_{k'} = f_kf_{k'}\delta_{k,-k'}$, which is the usual BCS assumption. With this, we get the following expression:
\begin{align}
\hat{\psi}_F(x') \hat{\psi}_F(x) &= \frac{1}{\mathcal{L}}\sum_{k} \text{e}^{ik(x-x')}f_kf_{-k} = \frac{1}{2\mathcal{L}}\sum_{k} \left(\text{e}^{ik(x-x')}f_{k}f_{-k}+\text{e}^{-ik(x-x')}f_{-k}f_{k}\right) \nonumber \\
&= \frac{i}{\mathcal{L}}\sum_{k} \sin(k(x-x'))f_{k}f_{-k}, 
\end{align}
where we in the last equality used, that the operators anticommute. Let us plug this into the first term of the interaction Hamiltonian to see how it plays out: 
\begin{align}
H^\text{int}_{FF,1} &= \int dxdx' \langle \hat{\psi}^\dagger_F(x) \hat{\psi}^\dagger_F(x') \rangle V^\text{ind}_{FF}(x'-x) \langle \hat{\psi}_F(x') \hat{\psi}_F(x) \rangle \nonumber \\
&= - \frac{1}{\mathcal{L}^2}\sum_{k,k'}\langle f^\dagger_{k}f^\dagger_{-k} \rangle \langle f_{k'}f_{-k'} \rangle \int dx dx' \sin(k(x'-x))\sin(k'(x'-x))V^\text{ind}_{FF}(x'-x) \nonumber \\
&= - \sum_{k,k'}\langle f^\dagger_{k}f^\dagger_{-k} \rangle \langle f_{k'}f_{-k'} \rangle \int \frac{du}{\mathcal{L}} \sin(ku)\sin(k'u)V^\text{ind}_{FF}(u) \nonumber \\
&= - \sum_{k,k'}\langle f^\dagger_{k}f^\dagger_{-k} \rangle \langle f_{k'}f_{-k'} \rangle \tilde{V}^\text{ind}_{FF}(k,k')
\end{align}
where we use the substitution $u = x'-x$ and use that $\int dx = \mathcal{L}$ and where we have defined the coupling potential $\tilde{V}^\text{ind}_{FF}(k,k')$. The two last terms are analogous and yield:
\begin{align}
H^\text{int}_{FF,2} &= - \sum_{k,k'}\left(f^\dagger_k f^\dagger_{-k} - \langle f^\dagger_k f^\dagger_{-k}\rangle\right)\langle f_{k'}f_{-k'} \rangle \tilde{V}^\text{ind}_{FF}(k,k'), \nonumber \\
H^\text{int}_{FF,3} &= - \sum_{k,k'}\left(f_k f_{-k} - \langle f_k f_{-k}\rangle\right)\langle f^\dagger_{k'}f^\dagger_{-k'} \rangle \tilde{V}^\text{ind}_{FF}(k,k'), \nonumber
\end{align}
Now seeing, that the coupling potential enters in all of the terms, we better calculate, what it actually is. We get, that it can actually be expressed very simply in terms of the induced interactions calculated above:
\begin{equation}
\tilde{V}^\text{ind}_{FF}(k,k') = \frac{1}{2\mathcal{L}}\left[ V^\text{ind}_{FF}(k-k') - V^\text{ind}_{FF}(k+k')  \right].
\end{equation}
Since $V^\text{ind}_FF(q)$ only depends on $q^2$, we get the following important properties of our coupling potential: 
\begin{align}
\tilde{V}^\text{ind}_{FF}(k,k') &= \tilde{V}^\text{ind}_{FF}(k',k), \hspace{0.5cm} \text{Symmetry in arguments}, \nonumber \\
\tilde{V}^\text{ind}_{FF}(-k,k') &= -\tilde{V}^\text{ind}_{FF}(k,k'), \hspace{0.5cm} \text{Uneven in single argument}, \nonumber \\
\tilde{V}^\text{ind}_{FF}(-k,-k') &= \tilde{V}^\text{ind}_{FF}(k,k'), \hspace{0.5cm} \text{Even in double argument}.
\end{align}
These symmmetries will be important later on. Finally adding them all up yields: 
\begin{equation}
H^\text{int}_{FF} = -\sum_{k,k'} \tilde{V}^\text{ind}_{FF}(k,k')\left[ \langle f_{k'}f_{-k'} \rangle f^\dagger_k f^\dagger_{-k} + \langle f^\dagger_{k'}f^\dagger_{-k'} \rangle f_k f_{-k} - \langle f_{k}f_{-k} \rangle  \langle f^\dagger_{k'}f^\dagger_{-k'} \rangle \right] \nonumber
\end{equation}

We simplify the expression by defining the pairing potential:
\begin{equation}
\Delta_k = -2\sum_{k'}\tilde{V}^\text{ind}_{FF}(k,k')\langle f_{k'}f_{-k'} \rangle.
\label{eq.pairingpotentialdef}
\end{equation}

Then the interaction Hamiltonian can finally be written as:
\begin{equation}
H^\text{int}_{FF} = \frac{1}{2}\sum_{k} \left[ \Delta_k f^\dagger_k f^\dagger_{-k} - \Delta^*_k f_k f_{-k} - \Delta_k \langle f^\dagger_{k'}f^\dagger_{-k'} \rangle \right]
\label{eq.HFFintfinal}
\end{equation}
The minus sign in front of $\Delta^*_k$ comes from anticommuting the operators. Since the coupling potential is uneven in a single argument $\Delta_{-k} = -\Delta_k$: the pairing potential is uneven as well. We are now ready to study the full Hamiltonian.

\section{The full fermion Hamiltonian} \label{sec.HFFfull}
The free particle Hamiltonian for the fermions is of course $H_0 = \sum_k \frac{k^2}{2m_F} f^\dagger_k f_k$. Further we inforce, that the number of fermionic particles is conserved. This is achieved by subtracting $\mu N = \mu \sum_k f^\dagger_k f_k$, where $\mu$ is the chemical potential. This means that our full fermion-fermion Hamiltonian is:
\begin{equation}
H_{FF} = H_0-\mu N + H^\text{int}_{FF} = \sum_k \varepsilon_k f^\dagger_k f_k + \frac{1}{2}\sum_{k} \left[ \Delta_k f^\dagger_k f^\dagger_{-k} - \Delta^*_k f_k f_{-k} - \Delta_k \langle f^\dagger_{k'}f^\dagger_{-k'} \rangle \right], 
\label{eq.HFFdef}
\end{equation} 
where $\varepsilon_k = \frac{k^2}{2m_F}-\mu$ is the kinetic energy relative to the chemical potential. We can bring the above into a matrix form:
\begin{equation}
H_{FF} = -\sum_k \Delta_k\langle f^\dagger_k f^\dagger_{-k}\rangle + \frac{1}{2}\sum_{k} F_k^\dagger \mathcal{H}_{FF,k} F_k, \hspace{0.5cm} \mathcal{H}_{FF,k} = \begin{bmatrix} \varepsilon_k & \Delta_k \\ \Delta^*_k & -\varepsilon_k \end{bmatrix}, \hspace{0.5cm} F^\dagger_k = \begin{bmatrix} f_k^\dagger & f_{-k} \end{bmatrix}, 
\end{equation}

and we see that up to a constant it is on the same form as the Kitaev Hamiltonian defined in equation \eqref{eq.HKitaevpre}. Hence, we can use the diagonalization made there modified by the constant $-\sum_k \Delta_k\langle f^\dagger_k f^\dagger_{-k}\rangle $ and get a result completely analogous to equation \eqref{eq.Kitaev.H_diagonalpre}: 
\begin{equation}
H_{FF} = \frac{1}{2}\sum_k (\varepsilon_k-2E_{F,k}-2\Delta_k\langle f^\dagger_k f^\dagger_{-k}\rangle) + \sum_k E_{F,k} \zeta^\dagger_k \zeta_k, \hspace{0.5cm} E_{F,k} = \sqrt{\varepsilon_k^2 + |\Delta_k|^2}.
\label{eq.Kitaev.HFF_diagonal}
\end{equation}
The new quasiparticle fermionic operators are defined in equation \eqref{eq.fermionquasiparticledef} with $u_{F,k},v_{F,k}$ defined in equation \eqref{eq.Kitaev.uk_vk}. From this we see, that the ground state of the system at $T=0$ is defined by having no quasiparticles $\zeta$ present: $\zeta_k \ket{g.s.}_0 = 0$ for all $k$. This means, that all the averages met in this section is to be taken with respect to the ground state $\ket{g.s.}$ at nonzero temperatures. This in turn can be calculated using Fermi statistics, and so the program is (in principle) now clear. 

\section{The pairing potential} \label{sec.pairingpotential}
The first step is to calculate the pairing potential $\Delta_k$. Inspecting the definition in equation \eqref{eq.pairingpotentialdef} we see, that we need to calculate $\langle f_k f_{-k} \rangle$, which in turn specifies another term in the Hamiltonian: $\langle f^\dagger_k f^\dagger_{-k} \rangle$. From the transformation defined in equation \eqref{eq.fermionquasiparticledef}, we get that $f_k = u^*_{F,k}\zeta_k - v_{F,k}\zeta^\dagger_{-k}, f_{-k} = v_{F,k}\zeta^\dagger_k + u^*_{F,k}\zeta_{-k}$ and so:
\begin{align}
\langle f_k f_{-k} \rangle &= \left \langle (u^*_{F,k}\zeta_k - v_{F,k}\zeta^\dagger_{-k}) (v_{F,k}\zeta^\dagger_k + u^*_{F,k}\zeta_{-k}) \right \rangle = u^*_{F,k}v_{F,k}\left[ \left \langle \zeta_k \zeta^\dagger_{k} \right \rangle - \left \langle \zeta^\dagger_{-k} \zeta_{-k} \right \rangle \right]  \nonumber \\
& =  u^*_{F,k}v_{F,k}\left[ 1 - \left \langle \zeta^\dagger_{k} \zeta_k \right \rangle - \left \langle \zeta^\dagger_{-k} \zeta_{-k} \right \rangle \right] = u^*_{F,k}v_{F,k}\left[1 - 2f(E_{F,k})\right]
\end{align}

