% Chapter 3

\chapter{The one dimensional Fermi Hamiltonian} % Main chapter title

\label{Chapter4} % For referencing the chapter elsewhere, use \ref{Chapter3} 

\lhead{Chapter 4. \emph{1D Fermi Hamiltonian}} % This is for the header on each page - perhaps a shortened title

%----------------------------------------------------------------------------------------
In this chapter we will finally arrive at the promised Kitaev Hamiltonian promised in chapter \ref{Chapter2}. We start by calculating the pair interaction Hamiltonian for the fermions in section \ref{sec.HFFint}. This is then inserted into the full Hamiltonian in section \ref{sec.HFFfull}.

\section{The fermion interacting Hamiltonian} \label{sec.HFFint}

We start out in real space. The interaction Hamiltonian for pair interactions is given by:
\begin{equation}
H^\text{int}_{FF} = \int dxdx' \hat{\psi}^\dagger_F(x)\hat{\psi}^\dagger_F(x')\tilde{V}^\text{ind}_{FF}(x'-x) \hat{\psi}_F(x') \hat{\psi}_F(x).
\label{eq.HFFintdef}
\end{equation}
Here the induced interaction in real space is achieved by a Fourier transform: $\tilde{V}^\text{ind}_{FF}(x'-x) = \int dq\; \text{e}^{-iq(x'-x)} V^\text{ind}_{FF}(q)$, with $V^\text{ind}_{FF}(q)$ written in equation \eqref{eq.VFFq} above. Further we make a mean field approximation to get the Hamiltonian on a solvable quadratic form. We write
\begin{equation}
\hat{\psi}_F(x') \hat{\psi}_F(x) = \langle \hat{\psi}_F(x') \hat{\psi}_F(x) \rangle + \left(\hat{\psi}_F(x') \hat{\psi}_F(x)-\langle \hat{\psi}_F(x') \hat{\psi}_F(x) \rangle \right) = \langle \hat{\psi}_F(x') \hat{\psi}_F(x) \rangle + A_F(x',x)
\end{equation}
and treat the last part $A_F$ as a small quantity. The meaning of this will become clearer later on. The mean in the end has to be taken with respect to the specific state of the system, we will also see how this plays out later on. We now only keep terms of $A_F$ and $A^\dagger_F$ up to first order, so that we get the following three terms for the interaction Hamiltonian:\footnote{Formally we are decoupling the Hamiltonian, so that it so no longer really describing any interaction.}
\begin{align}
H^\text{int}_{FF} = &\int dxdx' \langle \hat{\psi}^\dagger_F(x) \hat{\psi}^\dagger_F(x') \rangle \tilde{V}^\text{ind}_{FF}(x'-x) \langle \hat{\psi}_F(x') \hat{\psi}_F(x) \rangle + \nonumber \\
&\int dxdx' A^\dagger_F(x,x') \tilde{V}^\text{ind}_{FF}(x'-x) \langle \hat{\psi}_F(x') \hat{\psi}_F(x) \rangle + \nonumber \\
&\int dxdx' \langle \hat{\psi}^\dagger_F(x) \hat{\psi}^\dagger_F(x') \rangle \tilde{V}^\text{ind}_{FF}(x'-x) A_F(x',x). \nonumber
\end{align} 
To get back to momentum space we expand the field operators in terms of plane waves: $\hat{\psi}_F(x) = \frac{1}{\sqrt{\mathcal{L}}}\sum_k \text{e}^{ikx} f_k$, where as usual $f_k$ is the fermion annihilation operator. Further we assume, that only states belonging to opposite momenta couples.\footnote{At least they are assumed to be absolutely dominant.} This means, that $f_kf_{k'} = f_kf_{k'}\delta_{k,-k'}$, which is the usual BCS assumption. With this, we get the following expression:
\begin{align}
\hat{\psi}_F(x') \hat{\psi}_F(x) &= \frac{1}{\mathcal{L}}\sum_{k} \text{e}^{ik(x-x')}f_kf_{-k} = \frac{1}{2\mathcal{L}}\sum_{k} \left(\text{e}^{ik(x-x')}f_{k}f_{-k}+\text{e}^{-ik(x-x')}f_{-k}f_{k}\right) \nonumber \\
&= \frac{i}{\mathcal{L}}\sum_{k} \sin(k(x-x'))f_{k}f_{-k}, 
\end{align}
where we in the last equality used, that the operators anticommute. Let us plug this into the first term of the interaction Hamiltonian to see how it plays out: 
\begin{align}
H^\text{int}_{FF,1} &= \int dxdx' \langle \hat{\psi}^\dagger_F(x) \hat{\psi}^\dagger_F(x') \rangle \tilde{V}^\text{ind}_{FF}(x'-x) \langle \hat{\psi}_F(x') \hat{\psi}_F(x) \rangle \nonumber \\
&= - \frac{1}{\mathcal{L}^2}\sum_{k,k'}\langle f^\dagger_{k}f^\dagger_{-k} \rangle \langle f_{k'}f_{-k'} \rangle \int dx dx' \sin(k(x'-x))\sin(k'(x'-x))\tilde{V}^\text{ind}_{FF}(x'-x) \nonumber \\
&= - \frac{1}{\mathcal{L}}\sum_{k,k'}\langle f^\dagger_{k}f^\dagger_{-k} \rangle \langle f_{k'}f_{-k'} \rangle \int du \sin(ku)\sin(k'u)\tilde{V}^\text{ind}_{FF}(u) \nonumber \\
&= - \frac{1}{2\mathcal{L}}\sum_{k,k'}\langle f^\dagger_{k}f^\dagger_{-k} \rangle \langle f_{k'}f_{-k'} \rangle W^\text{ind}_{FF}(k,k')
\end{align}
where we use the substitution $u = x'-x$ and use that $\int dx = \mathcal{L}$. Further we have defined the coupling potential $W^\text{ind}_{FF}(k,k')$. The two last terms are analogous and yield:
\begin{align}
H^\text{int}_{FF,2} &= - \frac{1}{2\mathcal{L}}\sum_{k,k'}\left(f^\dagger_k f^\dagger_{-k} - \langle f^\dagger_k f^\dagger_{-k}\rangle\right)\langle f_{k'}f_{-k'} \rangle W^\text{ind}_{FF}(k,k'), \nonumber \\
H^\text{int}_{FF,3} &= - \frac{1}{2\mathcal{L}}\sum_{k,k'}\left(f_k f_{-k} - \langle f_k f_{-k}\rangle\right)\langle f^\dagger_{k'}f^\dagger_{-k'} \rangle W^\text{ind}_{FF}(k,k'), \nonumber
\end{align}
Now seeing, that the coupling potential enters in all of the terms, we better calculate, what it actually is. We get, that it can be expressed simply in terms of the induced interaction calculated in chapter \ref{Chapter3}:
\begin{equation}
W^\text{ind}_{FF}(k,k') = 2\int du \sin(ku)\sin(k'u)\tilde{V}^\text{ind}_{FF}(u) = V^\text{ind}_{FF}(k-k') - V^\text{ind}_{FF}(k+k').
\end{equation}
Since $V^\text{ind}_{FF}(q)$ only depends on $q^2$, we get the following important properties of the coupling potential: 
\begin{align}
W^\text{ind}_{FF}(k,k') &= W^\text{ind}_{FF}(k',k), \hspace{0.5cm} \text{Symmetry in arguments}, \nonumber \\
W^\text{ind}_{FF}(-k,k') &= -W^\text{ind}_{FF}(k,k'), \hspace{0.5cm} \text{Uneven in single argument}, \nonumber \\
W^\text{ind}_{FF}(-k,-k') &= W^\text{ind}_{FF}(k,k'), \hspace{0.5cm} \text{Even in double argument}.
\label{eq.CouplingPotentialSymmetries}
\end{align}
These symmmetries will be important later on. From here on and out the analysis is very similar to the historical Bardeen-Cooper-Schrieffer treatment. However, we will not be assuming any constancy of the coupling potential over a range of momentum, as is the case in traditional BCS-theory\cite{Tinkham,LandauStatPhys2,PlischkeStatPhys}. Adding the three terms yields:
\begin{equation}
H^\text{int}_{FF} = -\frac{1}{2\mathcal{L}}\sum_{k,k'} W^\text{ind}_{FF}(k,k')\left[ \langle f_{k'}f_{-k'} \rangle f^\dagger_k f^\dagger_{-k} + \langle f^\dagger_{k'}f^\dagger_{-k'} \rangle f_k f_{-k} - \langle f_{k}f_{-k} \rangle  \langle f^\dagger_{k'}f^\dagger_{-k'} \rangle \right] \nonumber
\end{equation}

We simplify the expression by defining the pairing potential:
\begin{equation}
\Delta_k = -\frac{1}{\mathcal{L}}\sum_{k'}W^\text{ind}_{FF}(k,k')\langle f_{k'}f_{-k'} \rangle.
\label{eq.pairingpotentialdef}
\end{equation}

Then the interaction Hamiltonian can finally be written as:
\begin{equation}
H^\text{int}_{FF} = \frac{1}{2}\sum_{k} \left[ \Delta_k f^\dagger_k f^\dagger_{-k} - \Delta^*_k f_k f_{-k} - \Delta_k \langle f^\dagger_{k'}f^\dagger_{-k'} \rangle \right]
\label{eq.HFFintfinal}
\end{equation}
The minus sign in front of $\Delta^*_k$ comes from anticommuting the operators. Since the coupling potential is uneven in a single argument $\Delta_{-k} = -\Delta_k$: the pairing potential is uneven as well. We are now ready to study the full Hamiltonian.

\section{The full fermion Hamiltonian} \label{sec.HFFfull}
The free particle Hamiltonian for the fermions is of course $H_0 = \sum_k \frac{k^2}{2m_F} f^\dagger_k f_k$. Further we inforce, that the number of fermionic particles is conserved. This is achieved by subtracting $\mu N = \mu \sum_k f^\dagger_k f_k$, where $\mu$ is the chemical potential. We hereby obtain the full fermion (grand) Hamiltonian:
\begin{equation}
H_{FF} = H_0-\mu N + H^\text{int}_{FF} = \sum_k \varepsilon_k f^\dagger_k f_k + \frac{1}{2}\sum_{k} \left[ \Delta_k f^\dagger_k f^\dagger_{-k} - \Delta^*_k f_k f_{-k} - \Delta_k \langle f^\dagger_{k'}f^\dagger_{-k'} \rangle \right], 
\label{eq.HFFdef}
\end{equation} 
where $\varepsilon_k = \frac{k^2}{2m_F}-\mu$ is the kinetic energy relative to the chemical potential. We can bring the above into a matrix form:
\begin{equation}
H_{FF} = -\sum_k \Delta_k\langle f^\dagger_k f^\dagger_{-k}\rangle + \frac{1}{2}\sum_{k} F_k^\dagger \mathcal{H}_{FF,k} F_k, \hspace{0.5cm} \mathcal{H}_{FF,k} = \begin{bmatrix} \varepsilon_k & \Delta_k \\ \Delta^*_k & -\varepsilon_k \end{bmatrix}, \hspace{0.5cm} F^\dagger_k = \begin{bmatrix} f_k^\dagger & f_{-k} \end{bmatrix}, 
\end{equation}

and we see that up to a constant it is on the same form as the Kitaev Hamiltonian defined in equation \eqref{eq.HKitaevpre}. Hence, we can use the diagonalization made there modified by the constant $-\sum_k \Delta_k\langle f^\dagger_k f^\dagger_{-k}\rangle $ and get a result completely analogous to equation \eqref{eq.Kitaev.H_diagonalpre}: 
\begin{equation}
H_{FF} = \frac{1}{2}\sum_k (\varepsilon_k-2E_{F,k}-2\Delta_k\langle f^\dagger_k f^\dagger_{-k}\rangle) + \sum_k E_{F,k} \zeta^\dagger_k \zeta_k, \hspace{0.5cm} E_{F,k} = \sqrt{\varepsilon_k^2 + |\Delta_k|^2}.
\label{eq.Kitaev.HFF_diagonal}
\end{equation}
The new quasiparticle fermionic operators are defined in equation \eqref{eq.fermionquasiparticledef} with $u_{F,k},v_{F,k}$ defined in equation \eqref{eq.Kitaev.uk_vk}. From this we see, that the ground state of the system at $T=0$ is defined by having no quasiparticles $\zeta$ present: $\zeta_k \ket{g.s.}_0 = 0$ for all $k$. Further we now know have to take the averages met in this section. They are to be taken with respect to the state thermalized state of the system, which in turn can be calculated using Fermi statistics, as we shall see in the following section. 

\section{The distribution of the quasiparticles}
In the previous section we inforced, that the number of physical ($f$) fermions in the system is fixed. However, we do not fix the number of quasiparticles ($\zeta$). This means, that the partition function takes the form of the \textit{canonical} partition function: $Z = \tr\left[\text{e}^{-\beta H_{FF}}\right]$. In terms of thermodynamics this means, that every single quasiparticle is in thermal (and not diffusive) equilibrium with all the others, hence working as a heat reservoir. Since the fermion Hamiltonian is diagonal in the quasiparticles $\zeta_k$, we can calculate the partition function for each $k$ by replacing $H_{FF}$ with the $k$'th (diagonal) term. Dropping the unimportant ground state energy $E_0 = \frac{1}{2}\sum_k (\varepsilon_k-2E_{F,k}-2\Delta_k\langle f^\dagger_k f^\dagger_{-k}\rangle)$, we get:
\begin{equation}
Z_k = \tr\left[\text{e}^{-\beta H_{FF,k}}\right] = \tr\left[\text{e}^{-\beta E_{F,k}\zeta^\dagger_k\zeta_k }\right] = 1 + \text{e}^{-\beta E_{F,k}}. 
\end{equation}     
For the calculation of the trace, one can for example use the the single particle complete basis $\{\ket{g.s.}_0, \zeta^\dagger_k\ket{g.s.}_0 \}$. This is all a rather involved way of saying, that the quasiparticle can either be absent $\ket{g.s.}_0$ and have zero energy or present $\zeta^\dagger_k\ket{g.s.}_0$ and have energy $E_{F,k}$. Finally we get the Fermi-Dirac distribution of the quasiparticles:
\begin{equation}
f(E_{F,k}) = \left\langle \zeta^\dagger_k\zeta_k \right\rangle = \sum_n n P_k(n) = \frac{0\cdot 1 + 1 \cdot \text{e}^{-\beta E_{F,k}} }{Z_k} = \frac{1}{\text{e}^{\beta E_{F,k}} + 1}, 
\end{equation}
where $n=0,1$ is the possible number of fermions in the state, and $P_k(n) = \text{e}^{-\beta n E_{F,k}}/Z_k$ is the probability of the state being occupied\cite{PlischkeStatPhys,SchroederThermal}. This is the average number of $\zeta_k$ particles in the thermalized state of the system. We now know how to take all the averages met so far, and so the program is (in principle) clear.  

\section{The pairing potential} \label{sec.pairingpotential}
\subsection{The integral equation} \label{subsec.pairingpotential.integralequation}
In this section we find an integral equation for the pairing potential $\Delta_k$. Inspecting the definition in equation \eqref{eq.pairingpotentialdef} we see, that we need to calculate $\langle f_k f_{-k} \rangle$, which in turn specifies another term in the Hamiltonian: $\langle f^\dagger_k f^\dagger_{-k} \rangle$. From the transformation defined in equation \eqref{eq.fermionquasiparticledef}, we get that $f_k = u^*_{F,k}\zeta_k - v_{F,k}\zeta^\dagger_{-k}, f_{-k} = v_{F,k}\zeta^\dagger_k + u^*_{F,k}\zeta_{-k}$ and so:
\begin{align}
\langle f_k f_{-k} \rangle &= \left \langle (u^*_{F,k}\zeta_k - v_{F,k}\zeta^\dagger_{-k}) (v_{F,k}\zeta^\dagger_k + u^*_{F,k}\zeta_{-k}) \right \rangle = u^*_{F,k}v_{F,k}\left[ \left \langle \zeta_k \zeta^\dagger_{k} \right \rangle - \left \langle \zeta^\dagger_{-k} \zeta_{-k} \right \rangle \right]  \nonumber \\
& =  u^*_{F,k}v_{F,k}\left[ 1 - \left \langle \zeta^\dagger_{k} \zeta_k \right \rangle - \left \langle \zeta^\dagger_{-k} \zeta_{-k} \right \rangle \right] = u^*_{F,k}v_{F,k}\left[1 - 2f(E_{F,k})\right], \nonumber
\end{align}
where we in the last equality use, that $\left \langle \zeta^\dagger_{k} \zeta_{k} \right \rangle = f(E_{F,k})=(\exp(\beta E_{F,k})+1)^{-1} $ from the previous section. Inserting this into the expression for $\Delta_k$ and using the relation $\frac{v_{F,k}\Delta^*_k}{u_{F,k}}=E_{F,k}-\varepsilon_k$ from equation \eqref{eq.Kitaev.uk_vk} we get:
\begin{equation}
\Delta_k = - \frac{1}{\mathcal{L}}\sum_{k'} W^\text{ind}_{FF}(k,k')\frac{\Delta_{k'}}{2E_{F,k'}}\tanh\left(\frac{\beta E_{F,k}}{2}\right).
\label{eq.PairingpotentialSumEquation}
\end{equation} 
Now since the wire has a finite length, the momenta are quantized according to $k = n\frac{\pi}{\mathcal{L}}$, and so $dk = \frac{\pi}{\mathcal{L}}$. This means, that we can transform the above into the following integral:
\begin{equation}
\Delta_k = - \int \frac{dk'}{2\pi} W^\text{ind}_{FF}(k,k')\frac{\Delta_{k'}}{E_{F,k'}}\tanh\left(\frac{\beta E_{F,k}}{2}\right),
\label{eq.PairingpotentialIntegralEquation}
\end{equation} 
and hereby canceling the length $\mathcal{L}$. Because of the symmetries of the coupling potential summarized in equation \eqref{eq.CouplingPotentialSymmetries}, we see, that $W^\text{ind}_{FF}(0,k') = 0$. Hence, it is immediately clear from both equation \eqref{eq.PairingpotentialSumEquation} and \eqref{eq.PairingpotentialIntegralEquation}, that $\Delta_{k=0} = 0$ for any temperature. This means, that there is actually no energy gap at $k=0$. Landau has argued, that if the minimum $k_c = \min_k \frac{E_{F,k}}{|k|}$ does not occur at $k=0$, then quasiparticles propagating with $|k|< k_c$ will do so without friction. $k_c > 0$ is the socalled Landau condition for superfluidity\cite{LandauStatPhys2,PlischkeStatPhys}. Since the energy $E_{F,k}$ goes to 0 for $k\to 0$, it is not clear whether the condition is satisfied or not, and a further analysis is needed. 

\subsection{Numerical calculation}
In this subsection we will describe a numerical method of calculating the pairing potential. The numerical calculation is based on the following idea. We start at $T=0$ and make an initial guess for $\Delta_k$. Then we plug this into the integral equation \eqref{eq.PairingpotentialIntegralEquation} for $\beta = \infty$ and hereby obtain a (hopefully) better result for $\Delta_k(T=0)$. This procedure is repeated succesively until it converges. Afterwards we increase the temperature with a small increment $dT$ and use $\Delta_k(T=0)$ as an initial guess repeating the process for $T=0$. This is repeated for rising temperatures until it will probably reach some critial temperature, where $\Delta_k(T_c)=0$ for all $k$, somewhat equivalent to the original BCS-theory\cite{Tinkham,LandauStatPhys2,PlischkeStatPhys}.    

