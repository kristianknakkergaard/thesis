% Chapter 4

\chapter{The Fermi Hamiltonian} % Main chapter title

\label{Chapter4} % For referencing the chapter elsewhere, use \ref{Chapter3} 

\lhead{Part II. \emph{One wire}}
\chead{Chapter 4. \emph{Fermi Hamiltonian}} % This is for the header on each page - perhaps a shortened title

%----------------------------------------------------------------------------------------
In this chapter we will finally arrive at the Kitaev Hamiltonian promised in chapter \ref{Chapter2}. Firstly, we calculate the pair interaction Hamiltonian for the fermions in section \ref{sec.HFFint}. This is done under the assumption, that we can restrict the induced interaction to have a vanishing Matsubara frequency: $\omega_q = 0$. See section \ref{sec.RetardationEffects} for details. The result for the interaction Hamiltonian is then inserted into the full Hamiltonian in section \ref{sec.HFFfull}. The approach is an example of a Bardeen-Cooper-Schrieffer (BCS) theory, see e.g. \cite{Tinkham,LandauStatPhys2,PlischkeStatPhys}. 

\section{The fermion interacting Hamiltonian} \label{sec.HFFint}

We start out in real space. The interaction Hamiltonian for pair interactions is given by:
\begin{equation}
H^\text{int}_{FF} = \frac{1}{2}\int dxdx' \hat{\psi}^\dagger_F(x)\hat{\psi}^\dagger_F(x')\tilde{V}^\text{ind}_{FF}(x'-x,0) \hat{\psi}_F(x') \hat{\psi}_F(x).
\label{eq.HFFintdef}
\end{equation}
With $\tilde{V}^\text{ind}_{FF}(x'-x,0)$ being the induced interaction at zero frequency in real space (see equation \eqref{eq.VFFx_exact}). The factor of $1/2$ is present, since the particles are identical. We make a mean field approximation to get the Hamiltonian on a solvable quadratic form. We write:
\begin{equation}
\hat{\psi}_F(x') \hat{\psi}_F(x) = \braket{\hat{\psi}_F(x') \hat{\psi}_F(x)} + \left(\hat{\psi}_F(x') \hat{\psi}_F(x) - \braket{\hat{\psi}_F(x') \hat{\psi}_F(x)} \right) = \braket{\hat{\psi}_F(x') \hat{\psi}_F(x)} + A_F(x',x)
\end{equation}
and treat the last part $A_F$ as a small quantity. The meaning of this will become clearer later on. The mean in the end has to be taken with respect to the specific state of the system, we will also see how this plays out later on as well. We now only keep terms of $A_F$ and $A^\dagger_F$ up to first order, so that we get the following three terms for the interaction Hamiltonian:\footnote{Formally we are decoupling the Hamiltonian, so that it so no longer really describing any interaction.}
\begin{align}
H^\text{int}_{FF} = & \frac{1}{2}\int dxdx' \braket {\hat{\psi}^\dagger_F(x) \hat{\psi}^\dagger_F(x')}  \tilde{V}^\text{ind}_{FF}(x'-x,0) \braket{\hat{\psi}_F(x') \hat{\psi}_F(x)} + \nonumber \\
& \frac{1}{2}\int dxdx' A^\dagger_F(x,x') \tilde{V}^\text{ind}_{FF}(x'-x,0) \braket{\hat{\psi}_F(x') \hat{\psi}_F(x)} + \nonumber \\
& \frac{1}{2}\int dxdx' \braket{\hat{\psi}^\dagger_F(x) \hat{\psi}^\dagger_F(x')} \tilde{V}^\text{ind}_{FF}(x'-x,0) A_F(x',x). \nonumber
\end{align} 
To get back to momentum space we expand the field operators in terms of plane waves: $\hat{\psi}_F(x) = \frac{1}{\sqrt{\mathcal{L}}}\sum_k \text{e}^{ikx} f_k$, where as usual $f_k$ is the fermion annihilation operator. Further we assume, that only states belonging to opposite momenta couples.\footnote{At least they are assumed to be absolutely dominant.} This means, that $f_kf_{k'} = f_kf_{k'}\delta_{k,-k'}$, which is the usual BCS assumption. With this, we get the following expression:
\begin{align}
\hat{\psi}_F(x') \hat{\psi}_F(x) &= \frac{1}{\mathcal{L}}\sum_{k} \text{e}^{ik(x-x')}f_kf_{-k} = \frac{1}{2\mathcal{L}}\sum_{k} \left(\text{e}^{ik(x-x')}f_{k}f_{-k}+\text{e}^{-ik(x-x')}f_{-k}f_{k}\right) \nonumber \\
&= \frac{i}{\mathcal{L}}\sum_{k} \sin(k(x-x'))f_{k}f_{-k}, 
\label{eq.correlation}
\end{align}
where we in the last equality used, that the operators anticommute. Let us plug this into the first term of the interaction Hamiltonian to see how it plays out: 
\begin{align}
H^\text{int}_{FF,1} &= \frac{1}{2}\int dxdx' \braket {\hat{\psi}^\dagger_F(x) \hat{\psi}^\dagger_F(x')} \tilde{V}^\text{ind}_{FF}(x'-x,0) \braket {\hat{\psi}_F(x') \hat{\psi}_F(x)} \nonumber \\
&= - \frac{1}{2\mathcal{L}^2}\sum_{k,k'} \braket {f^\dagger_{k}f^\dagger_{-k} } \braket {f_{k'}f_{-k'}} \int dx dx' \sin(k(x'-x))\sin(k'(x'-x))\tilde{V}^\text{ind}_{FF}(x'-x,0) \nonumber \\
&= - \frac{1}{2\mathcal{L}}\sum_{k,k'} \braket {f^\dagger_{k}f^\dagger_{-k}} \braket{f_{k'}f_{-k'}} \int du \sin(ku)\sin(k'u)\tilde{V}^\text{ind}_{FF}(u,0) \nonumber \\
&= - \frac{1}{2\mathcal{L}}\sum_{k,k'} \braket {f^\dagger_{k}f^\dagger_{-k}} \braket {f_{k'}f_{-k'}} W^\text{ind}_{FF}(k,k')
\end{align}
where we use the substitution $u = x'-x$ and use that $\int dx = \mathcal{L}$. Further we have defined the effective induced interaction $W^\text{ind}_{FF}(k,k')$. The two last terms are analogous and yield:
\begin{align}
H^\text{int}_{FF,2} &= - \frac{1}{2\mathcal{L}}\sum_{k,k'}\left(f^\dagger_k f^\dagger_{-k} - \braket {f^\dagger_k f^\dagger_{-k}}\right) \braket {f_{k'}f_{-k'}} W^\text{ind}_{FF}(k,k'), \nonumber \\
H^\text{int}_{FF,3} &= - \frac{1}{2\mathcal{L}}\sum_{k,k'}\left(f_k f_{-k} - \braket{f_k f_{-k}} \right)\braket {f^\dagger_{k'}f^\dagger_{-k'} } W^\text{ind}_{FF}(k,k'), \nonumber
\end{align}
Now seeing, that the effective interaction enters in all of the terms, we better calculate, what it actually is. We get, that it can be expressed simply in terms of the induced interaction in momentum space calculated in chapter \ref{Chapter3}:
\begin{equation}
W^\text{ind}_{FF}(k,k') = \int du \sin(ku)\sin(k'u)\tilde{V}^\text{ind}_{FF}(u,0) = \frac{1}{2}\left[V^\text{ind}_{FF}(k-k',0) - V^\text{ind}_{FF}(k+k',0)\right].
\label{eq.EffectiveInteraction}
\end{equation}
Since $V^\text{ind}_{FF}(q,0)$ only depends on $q^2$, we get the following important properties of the effective induced interaction: 
\begin{align}
W^\text{ind}_{FF}(k,k') &= W^\text{ind}_{FF}(k',k), \hspace{0.5cm} \text{Symmetry in arguments}, \nonumber \\
W^\text{ind}_{FF}(-k,k') &= -W^\text{ind}_{FF}(k,k'), \hspace{0.5cm} \text{Odd in single argument}, \nonumber \\
W^\text{ind}_{FF}(-k,-k') &= W^\text{ind}_{FF}(k,k'), \hspace{0.5cm} \text{Even in double argument}.
\label{eq.EffectiveInteractionSymmetries}
\end{align}
These symmmetries will be important later on. The analysis is a BCS treatment as mentioned. However, we will not be assuming any constancy of the effective interaction over a range of momentum, as is the case in traditional BCS-theory\cite{Tinkham,LandauStatPhys2,PlischkeStatPhys}. Adding the three terms yields:
\begin{equation}
H^\text{int}_{FF} = -\frac{1}{2\mathcal{L}}\sum_{k,k'} W^\text{ind}_{FF}(k,k')\left[ \braket{ f_{k'}f_{-k'}} f^\dagger_k f^\dagger_{-k} + \braket {f^\dagger_{k'}f^\dagger_{-k'}} f_k f_{-k} - \braket {f_{k}f_{-k}} \braket {f^\dagger_{k'}f^\dagger_{-k'}} \right] \nonumber
\end{equation}

We simplify the expression by defining the pairing potential:
\begin{equation}
\Delta_k = -\frac{1}{\mathcal{L}}\sum_{k'}W^\text{ind}_{FF}(k,k')\braket {f_{k'}f_{-k'}}.
\label{eq.pairingpotentialdef}
\end{equation}

Then the interaction Hamiltonian can finally be written as:
\begin{equation}
H^\text{int}_{FF} = \frac{1}{2}\sum_{k} \left[ \Delta_k f^\dagger_k f^\dagger_{-k} - \Delta^*_k f_k f_{-k} - \Delta_k \braket {f^\dagger_{k'}f^\dagger_{-k'}} \right]
\label{eq.HFFintfinal}
\end{equation}
The minus sign in front of $\Delta^*_k$ comes from anticommuting the operators. Since the effective interaction is odd in a single argument $\Delta_{-k} = -\Delta_k$: the pairing potential is odd as well. We are now ready to study the full Hamiltonian.

\section{The full fermion Hamiltonian} \label{sec.HFFfull}
The free particle Hamiltonian for the fermions is of course $H_0 = \sum_k \frac{k^2}{2m_F} f^\dagger_k f_k$. Further, we will have terms like $f^\dagger f^\dagger$. This means, that the Hamiltonian is not particle number conserving. However, we inforce that the system is in diffusive equilibrium and thus has a well-defined chemical potential $\mu(T)$. This is achieved by subtracting $\mu N_F = \mu \sum_k f^\dagger_k f_k$, where $\mu$ is the chemical potential. We hereby obtain the grand fermion Hamiltonian:
\begin{equation}
H_{FF} = H_0-\mu N_F + H^\text{int}_{FF} = \sum_k \varepsilon_k f^\dagger_k f_k + \frac{1}{2}\sum_{k} \left[ \Delta_k f^\dagger_k f^\dagger_{-k} - \Delta^*_k f_k f_{-k} - \Delta_k \braket {f^\dagger_{k'}f^\dagger_{-k'}} \right], 
\label{eq.HFFdef}
\end{equation} 
where $\varepsilon_k = \frac{k^2}{2m_F}-\mu$ is the kinetic energy relative to the chemical potential. We can bring the above into a standard Bogoliubov-de Gennes (BdG) form:

\begin{equation}
H_{FF} = \frac{1}{2}\sum_k \left[\varepsilon_k - \Delta_k \braket {f^\dagger_k f^\dagger_{-k}}\right] + \frac{1}{2}\sum_{k} \begin{bmatrix} f_k^\dagger & f_{-k} \end{bmatrix} \mathcal{H}_{FF,k} \begin{bmatrix} f_k \\ f^\dagger_{-k} \end{bmatrix}, \hspace{0.5cm} \mathcal{H}_{FF,k} = \begin{bmatrix} \varepsilon_k & \Delta_k \\ \Delta^*_k & -\varepsilon_k \end{bmatrix}, 
\end{equation}

and we see that up to a constant it is on the same form as the Kitaev Hamiltonian defined in equation \eqref{eq.HKitaevpre}. Hence, we can use the diagonalization made there modified by the constant $-\sum_k \Delta_k\braket {f^\dagger_k f^\dagger_{-k}} $ and get a result completely analogous to equation \eqref{eq.Kitaev.H_diagonalpre}: 
\begin{equation}
H_{FF} = \frac{1}{2}\sum_k \left[\varepsilon_k-E_{F,k}-\Delta_k\braket {f^\dagger_k f^\dagger_{-k}}\right] + \sum_k E_{F,k} \zeta^\dagger_k \zeta_k, \hspace{0.5cm} E_{F,k} = \sqrt{\varepsilon_k^2 + |\Delta_k|^2}.
\label{eq.Kitaev.HFF_diagonal}
\end{equation}
The new quasiparticle fermionic operators are defined in equation \eqref{eq.fermionquasiparticledef} with $u_{F,k},v_{F,k}$ defined in equation \eqref{eq.Kitaev.uk_vk}. From this we see, that the ground state of the system at $T=0$ is defined by having no quasiparticles $\zeta$ present: $\zeta_k \ket{\text{S}}_0 = 0$ for all $k$.\footnote{S is short for \text{S}uperfluid ground state.} Further we now know have to take the averages met in this section. They are to be taken with respect to the thermalized state of the system, which in turn can be calculated using Fermi statistics, as we shall see in the following section. 

The expected temperature dependence of the system is inferred from the original BCS-theory \cite{Tinkham,LandauStatPhys2,PlischkeStatPhys}. The pairing potential $\Delta_k(T)$ is expected to have its maximum for $T = 0$ and then monotonically decrease to 0, when the temperature is increased to a critical temperature $T_c$. This, we will see, defines a phase transition from a superfluid phase below $T_c$ to the normal phase above $T_c$. Let us therefore shortly review the Landau criterion for superfluidity. 

%The phase transition is expected to be a first order transition. This is again inferred from the original BCS-theory, where the heat capacity is \textit{discontinuous} at the critical temperature \cite{Tinkham}. It further means, that we expect the 

\section{Superfluidity} \label{sec.Superfluidity}
In this section we show the Landau criterion for superfluidity. It is heavily based on the argument in \cite{LandauStatPhys2}. It is a classical argument based solely on the transformation laws of energy and momentum in classical mechanics. 

Let us imagine a fluid of total mass $M$ moving in a capillary with constant velocity $\mathbf{v}$. This reference frame we denote $S_0$. Because of friction with the walls of the capillary we expect, that the fluid will gradually slow down. Let us then switch to the reference $S$, where the fluid initially lies still and the capillary moves with velocity $-\mathbf{v}$. The capillary will then, because of the viscosity, make a drag on the liquid. Let us assume, that this drag makes an elementary excitation of the liquid with momentum $\mathbf{k}$. The corresponding energy is then $E_{\mathbf{k}}$. The whole liquid now has a momentum $\mathbf{K}_0 = \mathbf{k}$ and an energy $\frac{K_0^2}{2M} = E_0 = E_{\mathbf{k}}$. Let us now switch back to the original frame $S_0$. Then the liquid has the momentum $K = K_0 + M\mathbf{v}$, and the energy is:
\begin{equation}
E = \frac{K^2}{2M} = \frac{(\mathbf{K}_0 + M\mathbf{v})^2}{2M} = E_0 + \mathbf{K}_0\cdot \mathbf{v} + \frac{1}{2}Mv^2 = E_{\mathbf{k}} + \mathbf{k} \cdot \mathbf{v} + \frac{1}{2}Mv^2 . \nonumber
\end{equation}
$\frac{1}{2}Mv^2$ is the initial energy of the moving liquid. Since the energy of the liquid most be lowered by the drag, we have $E_{\mathbf{k}} + \mathbf{k} \cdot \mathbf{v} < 0 $. For a given value of $\mathbf{k}$ it is clear, that the left hand side is a minimum, when $\mathbf{k}$ and $\mathbf{v}$ are antiparallel, such that: $ \mathbf{k}\cdot \mathbf{v} = -kv$. In total we get the condition:
\begin{equation}
v > \frac{E_{\mathbf{k}}}{k}. \nonumber
\end{equation}

This inequality states, that for a specific speed $v$ of the fluid, it is only momentum states obeying the inequality that can be excited. Now let us assume, that the minimum of the right hand side is nonzero: $v_c = \inf_{\mathbf{k}}\left[\frac{E_{\mathbf{k}}}{k} \right] > 0$. The inequality above then means, that for $v < v_c$, the liquid cannot be excited and no drag can occur. This is the defining quality of a superfluid, and we arrive at the Landau criterion for superfluidity:
\begin{equation}
\inf_{\mathbf{k}}\left[\frac{E_{\mathbf{k}}}{k} \right] > 0.
\end{equation}

Caution should be taken however. The argument only goes one way. It may be possible for a liquid not obeying this inequality to be a superfluid, but the reverse, a liquid obeying the inequality and being a normal fluid, is not possible as now shown. 

\section{The distribution of the quasiparticles}
In the previous section we inforced, that the number of physical ($f$) fermions in the system is fixed. However, we do not fix the number of quasiparticles ($\zeta$). This means, that the partition function takes the form of the \textit{canonical} partition function: $Z = \tr\left[\text{e}^{-\beta H_{FF}}\right]$. In terms of thermodynamics this means, that every single quasiparticle is in thermal (and not diffusive) equilibrium with all the others, hence working as a heat reservoir. Since the fermion Hamiltonian is diagonal in the quasiparticles $\zeta_k$, we can calculate the partition function for each $k$ by replacing $H_{FF}$ with the $k$'th (diagonal) term. Dropping the unimportant ground state energy $E_0 = \frac{1}{2}\sum_k (\varepsilon_k - E_{F,k} - \Delta_k\braket {f^\dagger_k f^\dagger_{-k}})$, we get:
\begin{equation}
Z_k = \tr\left[\text{e}^{-\beta H_{FF,k}}\right] = \tr\left[\text{e}^{-\beta E_{F,k}\zeta^\dagger_k\zeta_k }\right] = 1 + \text{e}^{-\beta E_{F,k}}. 
\end{equation}     
For the calculation of the trace, one can for example use the single particle complete basis $\{\ket{\text{S}}_0, \zeta^\dagger_k\ket{\text{S}}_0\}$. This is all a rather involved way of saying, that the quasiparticle can either be absent $\ket{\text{S}}_0$ and have zero energy or present $\zeta^\dagger_k\ket{\text{S}}_0$ and have energy $E_{F,k}$. Finally we get the Fermi-Dirac distribution of the quasiparticles:
\begin{equation}
f(E_{F,k}) = \braket {\zeta^\dagger_k\zeta_k} = \sum_N N P_k(N) = \frac{0\cdot 1 + 1 \cdot \text{e}^{-\beta E_{F,k}} }{Z_k} = \frac{1}{\text{e}^{\beta E_{F,k}} + 1}, 
\end{equation}
where $N=0,1$ is the possible number of fermions in the state, and $P_k(N) = \text{e}^{-\beta N E_{F,k}}/Z_k$ is the probability of the state being occupied\cite{PlischkeStatPhys,SchroederThermal}. This is the average number of $\zeta_k$ particles in the thermalized state of the system. We now know how to take all the averages met so far, and so the program is (in principle) clear. 

\section{The ground state and Cooper pairs}
The ground state $\ket{\text{S}}_0$ can be expressed simply in terms of the original $f$-operators as:
\begin{equation}
\ket{\text{S}}_0 = \prod_{k > 0} \left(u_{F,k} + v_{F,k}f^\dagger_{-k}f^\dagger_k\right)\ket{0},
\end{equation} 
where $\ket{0}$ is the vacuum: $f_k\ket{0} = 0$. This can be verified explicitly by checking, that $\zeta_k\ket{\text{S}}_0 = 0$ for all $k$, and that $_{0}\!\braket{\text{S}|\text{S}}_0 = 1$. In this it is important to notice, that the terms $(u_{F,k} + v_{F,k}f^\dagger_{-k}f^\dagger_k)$ commute with each other, since all daggered operators anticommute. The appearance of $f^\dagger_{-k}f^\dagger_k$ for each $k$ is a manifestation of the fact, that the fermions pair up in socalled Cooper pairs with opposite momenta. In this line of thinking it is also interesting to see, how the singly excited states look. A simple calculation shows:
\begin{equation}
\zeta^\dagger_k\ket{\text{S}}_0 = \prod_{k' > 0, k' \neq k }\left(u_{F,k'} + v_{F,k'}f^\dagger_{-k'}f^\dagger_{k'}\right)f^\dagger_k\ket{0}.
\end{equation}
The action of $\zeta^\dagger_k$ is thus to break up the $k$'th Cooper pair. 

\section{Vanishing trapping width limit}
In this section we will finally see, how we now can obtain a $l_t\to 0$ limit. From equation \eqref{eq.EffectiveInteraction} and the alternative form of the induced interaction in momentum space written after equation \eqref{eq.VFF(q,0)}, we get that:
\begin{align}
W^\text{ind}_{FF}(k,k') &= \frac{1}{2}\left[V^\text{ind}_{FF}(k-k',0) - V^\text{ind}_{FF}(k+k',0)\right] \nonumber \\
&= -\frac{m_Bg_{BF}^2n_B}{2\pi}\left[ \text{e}^{F(k-k')} E_1(F(k-k')) - \text{e}^{F(k+k')} E_1(F(k+k')) \right], \nonumber
\end{align}
where $E_1(x) = \int_1^\infty du \frac{\text{e}^{-xu}}{u}$ is the exponential integral. Separately $V^\text{ind}_{FF}(k-k',0)$ and $V^\text{ind}_{FF}(k+k',0)$ do not have a well-defined $l_t \to 0$ limit. However, it turns out that the effective interaction $W^\text{ind}_{FF}(k,k')$ does. To see this, we use the asymptotic behaviour of $E_1(x)$ for $0 < x \ll 1$: $E_1(x) \approx C -\ln(x)$, where $C$ is a constant.\footnote{The constant is minus the Euler-Mascheroni constant $\gamma$. To ten digits precision $\gamma = 0.5772156649$. This expansion is found in Maple 16.} The exponentials $\text{e}^{F(k-k')}$ and $\text{e}^{F(k+k')}$ just give $1$ in the $l_t \to 0$ limit, and so we are left with the expression:
\begin{equation}
\lim_{l_t \to 0} \; W^\text{ind}_{FF}(k,k') = -\frac{m_Bg_{BF}^2n_B}{2\pi} \ln\left[\frac{(k+k')^2+2/\xi^2}{(k-k')^2+2/\xi^2}\right].
\label{eq.EffectiveInteractionlt=0} 
\end{equation}
We finally have a closed form expression for the effective interaction. This will be used extensively in the numerical analyses in the next two chapters. We notice that along the lines $k = k'$ and $k = -k'$ the effective interaction diverges logarithmically for $k\to \infty$. Away from these lines the function quickly goes to zero. The dimensionless effective induced interaction for $l_t = 0$ is hereby:
\begin{equation}
\frac{2m_F}{k_F} W^\text{ind}_{FF}(k,k') = - 4\left( \frac{m_F}{m_B} + \frac{m_B}{m_F} + 2 \right) \frac{n_B^{1/3}}{n_F}(n_Ba_{BF}^3)^{2/3} \ln\left[\frac{(\tilde{k}+\tilde{k}')^2+2/\tilde{\xi}^2}{(\tilde{k}-\tilde{k}')^2+2/\tilde{\xi}^2}\right],
\label{eq.EffectiveInteractionlt=0dimensionless} 
\end{equation}

with $\tilde{k} = k/k_F$ and $\tilde{\xi} = k_F\xi = \sqrt{ \frac{ \pi }{ 8(n_Ba_B^3)^{1/3} } } \frac{n_F}{n_B^{1/3}}$. Hence, we notice that the effective induced interaction is proportional to the ratio of interparticle distances $n_B^{1/3}/n_F$ and the square of the Bose-Fermi gas parameter $(n_Ba_{BF}^3)^{1/3}$. 




